\documentclass[twoside]{article}
\usepackage[paperwidth=210mm, paperheight=297mm, margin=2cm]{geometry}
\usepackage[utf8]{inputenc}
\usepackage{ctex}

% Formatting Packages ——————————————————————————————————————
\usepackage{multicol}
\usepackage{multirow}
\usepackage{enumitem}
\usepackage{indentfirst}
\usepackage[toc]{multitoc}

% Math & Physics Packages ————————————————————————————
\usepackage{amsmath, amsthm, amsfonts, amssymb}
\usepackage{amssymb}
\usepackage{setspace}
\usepackage{physics}
\usepackage{cancel}
\usepackage{nicefrac}

% Image-related Packages —————————————————————————————
\usepackage{graphics, graphicx}
\usepackage{tikz}
\usetikzlibrary{arrows.meta}
\usepackage{pgfplots}
\pgfplotsset{compat=1.18}
\usepackage{xcolor}
\usepackage{fourier-orns}
\usepackage{lipsum}


%% Silakan otak-atik judul di sini
\title{\texorpdfstring{\vspace{-1.5em}}{}\textbf{概率论与数理统计}}
\author{张梓卫 10235101526}
\date{\the\day \ \monthyear}

% Kalau mau bab pertama nomornya 0, ganti 0 jadi -1
\setcounter{section}{0}

\renewcommand*\contentsname{第十三周概率论作业}
\renewcommand*{\multicolumntoc}{2}
\setlength{\columnseprule}{0.5pt}

% Layouting Packages ————————————————————————————————————————
\usepackage{titlesec}
\usepackage{fancyhdr}
\pagestyle{fancy}
\setlength{\headheight}{14.39996pt}
\fancyfoot[C]{\textit{By Deralive (10235101526)}}
\fancyfoot[R]{\thepage}

\fancyhead[LE, RO]{\textsl{\rightmark}}
\fancyhead[LO, RE]{\textsl{\leftmark}}

\renewcommand\headrule{
	\vspace{-6pt}
	\hrulefill
	\raisebox{-2.1pt}
	{\quad\floweroneleft\decoone\floweroneright\quad}
	\hrulefill}

\renewcommand\footrule{
	\hrulefill
	\raisebox{-2.1pt}
	{\quad\floweroneleft\decoone\floweroneright\quad}
	\hrulefill}
  
\newcommand{\fpb}[2]{\mathrm{FPB}(#1, #2)}
\newcommand{\kpk}[2]{\mathrm{KPK}(#1, #2)}

% Reference and Bibliography Packages ————————————————————
\usepackage{hyperref}
\hypersetup{
    colorlinks=true,
    linkcolor={black},
    citecolor={biru!70!black},
    urlcolor={biru!80!black}
}

\numberwithin{equation}{section}

% Silakan lihat dokumentasi package biblatex
% untuk format sitasi yang diperlukan
\usepackage[backend=biber]{biblatex}
\addbibresource{ref.bib}
\makeatletter
\def\@biblabel#1{}
\makeatother

\renewcommand{\mod}{\mathrm{mod} \ }

% Titling
\newcommand{\garis} [3] []{
	\begin{center}
		\begin{tikzpicture}
			\draw[#2-#3, ultra thick, #1] (0,0) to (1\linewidth,0);
		\end{tikzpicture}
	\end{center}
}

\newcommand{\monthyear}{
  \ifcase\month\or January\or February\or March\or April\or May\or June\or
  July\or August\or September\or October\or November\or
  December\fi\space\number\year
}

% Colour Palette ——————————————————————————————————————
\definecolor{merah}{HTML}{F4564E}
\definecolor{merahtua}{HTML}{89313E}
\definecolor{biru}{HTML}{60BBE5}
\definecolor{birutua}{HTML}{412F66}
\definecolor{hijau}{HTML}{59CC78}
\definecolor{hijautua}{HTML}{366D5B}
\definecolor{kuning}{HTML}{FFD56B}
\definecolor{jingga}{HTML}{FBA15F}
\definecolor{ungu}{HTML}{8C5FBF}
\definecolor{lavender}{HTML}{CBA5E8}
\definecolor{merjamb}{HTML}{FFB6E0}
\definecolor{mygray}{HTML}{E6E6E6}

% Theorems ————————————————————————————————————————————
\usepackage{tcolorbox}
\tcbuselibrary{skins,breakable,theorems}
\usepackage{changepage}

\newcounter{hitung}
\setcounter{hitung}{\thesection}

\makeatletter
	% Proof 证明如下
	\def\tcb@theo@widetitle#1#2#3{\hbox to \textwidth{\textsc{\large#1}\normalsize\space#3\hfil(#2)}}
	\tcbset{
		theorem style/theorem wide name and number/.code={ \let\tcb@theo@title=\tcb@theo@widetitle},
		proofbox/.style={skin=enhancedmiddle,breakable,parbox=false,boxrule=0mm,
			check odd page, toggle left and right, colframe=black!20!white!92!hijau,
			leftrule=8pt, rightrule=0mm, boxsep=0mm,arc=0mm, outer arc=0mm,
			left=3mm,right=3mm,top=0mm,bottom=0mm, toptitle=0mm,
			bottomtitle=0mm,colback=gray!3!white!98!biru, before skip=8pt, after skip=8pt,
			before={\par\vskip-2pt},after={\par\smallbreak},
		},
	}
	\newtcolorbox{ProofBox}{proofbox}
	\makeatother
	
	\let\realproof\proof
	\let\realendproof\endproof
	\renewenvironment{proof}[1][Prove:]{\ProofBox\strut\textsc{#1}\space}{\endProofBox}
        \AtEndEnvironment{proof}{\null\hfill$\blacksquare$}
        % Definition 定义环境
	\newtcbtheorem[use counter=hitung, number within=section]{dfn}{定义}
	{theorem style=theorem wide name and number,breakable,enhanced,arc=3.5mm,outer arc=3.5mm,
		boxrule=0pt,toprule=1pt,leftrule=0pt,bottomrule=1pt, rightrule=0pt,left=0.2cm,right=0.2cm,
		titlerule=0.5em,toptitle=0.1cm,bottomtitle=-0.1cm,top=0.2cm,
		colframe=white!10!biru,
		colback=white!90!biru,
		coltitle=white,
		shadow={1.3mm}{-1.3mm}{0mm}{gray!50!white}, % 添加阴影
        coltext=birutua!60!gray, title style={white!10!biru}, rbefoe skip=8pt, after skip=8pt,
		fonttitle=\bfseries,fontupper=\normalsize}{dfn}

	% 答题卡
	\newtcbtheorem[use counter=hitung, number within=section]{ans}{解答}
	{theorem style=theorem wide name and number,breakable,enhanced,arc=3.5mm,outer arc=3.5mm,
		boxrule=0pt,toprule=1pt,leftrule=0pt,bottomrule=1pt, rightrule=0pt,left=0.2cm,right=0.2cm,
		titlerule=0.5em,toptitle=0.1cm,bottomtitle=-0.1cm,top=0.2cm,
		colframe=white!10!biru,
		colback=white!90!biru,
		coltitle=white,
		shadow={1.3mm}{-1.3mm}{0mm}{gray!50!white}, % 添加阴影
        coltext=birutua!60!gray, title style={white!10!biru}, before skip=8pt, after skip=8pt,
		fonttitle=\bfseries,fontupper=\normalsize}{ans}

	% Axiom
	\newtcbtheorem[use counter=hitung, number within=section]{axm}{公理}
	{theorem style=theorem wide name and number,breakable,enhanced,arc=3.5mm,outer arc=3.5mm,
		boxrule=0pt,toprule=1pt,leftrule=0pt,bottomrule=1pt, rightrule=0pt,left=0.2cm,right=0.2cm,
		titlerule=0.5em,toptitle=0.1cm,bottomtitle=-0.1cm,top=0.2cm,
		colframe=white!10!biru,colback=white!90!biru,coltitle=white,
		shadow={1.3mm}{-1.3mm}{0mm}{gray!50!white!90}, % 添加阴影
        coltext=birutua!60!gray,title style={white!10!biru},before skip=8pt, after skip=8pt,
		fonttitle=\bfseries,fontupper=\normalsize}{axm}
 
	% Theorem
	\newtcbtheorem[use counter=hitung, number within=section]{thm}{定理}
	{theorem style=theorem wide name and number,breakable,enhanced,arc=3.5mm,outer arc=3.5mm,
		boxrule=0pt,toprule=1pt,leftrule=0pt,bottomrule=1pt, rightrule=0pt,left=0.2cm,right=0.2cm,
		titlerule=0.5em,toptitle=0.1cm,bottomtitle=-0.1cm,top=0.2cm,
		colframe=white!10!merah,colback=white!75!pink,coltitle=white, coltext=merahtua!80!merah,
		shadow={1.3mm}{-1.3mm}{0mm}{gray!50!white!90}, % 添加阴影
		title style={white!10!merah}, before skip=8pt, after skip=8pt,
		fonttitle=\bfseries,fontupper=\normalsize}{thm}
	
	% Proposition
	\newtcbtheorem[use counter=hitung, number within=section]{prp}{命题}
	{theorem style=theorem wide name and number,breakable,enhanced,arc=3.5mm,outer arc=3.5mm,
		boxrule=0pt,toprule=1pt,leftrule=0pt,bottomrule=1pt, rightrule=0pt,left=0.2cm,right=0.2cm,
		titlerule=0.5em,toptitle=0.1cm,bottomtitle=-0.1cm,top=0.2cm,
		colframe=white!10!hijau,colback=white!90!hijau,coltitle=white, coltext=hijautua!80!brown,
		shadow={1.3mm}{-1.3mm}{0mm}{gray!50!white}, % 添加阴影
		title style={white!10!hijau}, before skip=8pt, after skip=8pt,
		fonttitle=\bfseries,fontupper=\normalsize}{prp}


	% Example
	\newtcolorbox[use counter=hitung, number within=section]{cth}[1][]{breakable,
		colframe=white!10!jingga, coltitle=white!90!jingga, colback=white!85!jingga, coltext=black!10!brown!50!jingga, colbacktitle=white!10!jingga, enhanced, fonttitle=\bfseries,fontupper=\normalsize, attach boxed title to top left={yshift=-2mm}, before skip=8pt, after skip=8pt,
		title=Contoh~\thetcbcounter \ \ #1}

	% Catatan/Note
	\newtcolorbox{ctt}[1][]{enhanced, 
		left=4.1mm, borderline west={8pt}{0pt}{white!10!kuning}, 
		before skip=6pt, after skip=6pt, 
		colback=white!85!kuning, colframe= white!85!kuning, coltitle=orange!60!kuning!25!brown, coltext=orange!60!kuning!25!brown,
		fonttitle=\bfseries,fontupper=\normalsize, before skip=8pt, after skip=8pt,
		title=\underline{Catatan}  #1}
	
	% Komentar/Remark
	\newtcolorbox{rmr}[1][]{
		,arc=0mm,outer arc=0mm,
		boxrule=0pt,toprule=1pt,leftrule=0pt,bottomrule=5pt, rightrule=0pt,left=0.2cm,right=0.2cm,
		titlerule=0.5em,toptitle=0.1cm,bottomtitle=-0.1cm,top=0.2cm,
		colframe=white!10!kuning,colback=white!85!kuning,coltitle=white, coltext=orange!60!kuning,
		fonttitle=\bfseries,fontupper=\normalsize, before skip=8pt, after skip=8pt,
		title=Komentar  #1}

% ————————————————————————————————————————————————————————————————————————
\begin{document}

\maketitle
\vspace{-3.5em}
\garis{Kite}{Kite}

\tableofcontents

\section{第一章习题1、3、4}

\subsection{习题 1}

1. 写出下列随机试验的样本空间 \( S \):

(1) 记录一个班一次数学考试的平均分数 (设以百分制记分)。  

\begin{ans}{1}{1}

样本空间 \( S \) 是从 0 到 100 的所有可能分数组成的集合。
\[
S = [0, 100]
\]
\end{ans}

(2) 生产产品直到有 10 件正品为止,记录生产产品的总件数。  

\begin{ans}{2}{2}
    
样本空间 \( S \) 是所有可能的生产件数,因为必须生产 10 件正品,所以最少生产 10 件,最多可以无限生产。
\[
S = \{10, 11, 12, \dots, n, n = +\infty \}
\]

\end{ans}

(3) 对某工厂生产的产品进行检查,合格的记上“正品”,不合格的记上“次品”,如连续查出了 2 件次品就停止检查,若检查了 4 件产品就停止检查,记录检查的结果。  
样本空间 \( S \) 是所有可能的检查结果。假设“正品”记为 \( P \),“次品”记为 \( F \),

\begin{ans}{3}{3}

    样本空间可以包含如下元素:
\[
S = \{PPPP, PPPF, PPFP, PFPP, FPFP, FPPF, PFPF, PPFF, PFF, FPF, FF \}
\]

\end{ans}

(4) 在单位圆内任意取一点,记录它的坐标。

\begin{ans}{4}{4}

样本空间 \( S \) 是单位圆内的所有点的坐标,坐标形式为 \( (x, y) \),其中满足:
\[
x^2 + y^2 \leq 1
\]
即:
\[
S = \{(x, y) | x^2 + y^2 \leq 1 \}
\]

\end{ans}

\subsection{习题3}

1. 已知 \( A, B, C \) 是三事件,且 \( P(A) = P(B) = P(C) = \frac{1}{4} \),\( P(AB) = P(BC) = 0 \),\( P(AC) = \frac{1}{8} \),求 \( A, B, C \) 至少有一个发生的概率。

求A、B、C至少有一个发生的概率,即计算三个事件的并集概率:

\begin{ans}{5}{5}

\[
P(A \cup B \cup C) = P(A) + P(B) + P(C) - P(A \cap B) - P(A \cap C) - P(B \cap C) + P(A \cap B \cap C)
\]

其中,由于$ ABC \subset AB$,故 $ P(ABC) <= P(AB) = 0 $, 由于 $ P(ABC) >= 0 $, 故 $ P(ABC) = 0$。

则所求 $ P(ABC) = \frac{5}{8 }$.

\end{ans}

2. 已知 \( P(A) = \frac{1}{2}, P(B) = \frac{1}{3}, P(C) = \frac{1}{5}, P(AB) = \frac{1}{10}, P(AC) = \frac{1}{15}, P(BC) = \frac{1}{20}, P(ABC) = \frac{1}{30} \),求 \( A \cup B, A \cup B \cup C, \overline{A} \overline{B} C, \overline{A} B \cup C \) 的概率。

\begin{ans}{6}{6}

解答:

\[
P(A \cup B) = P(A) + P(B) - P(AB) = \frac{1}{2} + \frac{1}{3} - \frac{1}{10} = \frac{11}{15}
\]

\[
P(\overline{A} \cap \overline{B}) = P(A \cup B) = 1 - P(A \cup B) = \frac{4}{15}
\]

\[
P(A \cup B \cup C) = P(A) + P(B) + P(C) - P(A \cap B) - P(A \cap C) - P(B \cap C) + P(A \cap B \cap C)
\]
\[
= \frac{1}{2} + \frac{1}{3} + \frac{1}{5} - \frac{1}{10} - \frac{1}{15} - \frac{1}{20} + \frac{1}{30}
\]
\[
= \frac{51}{60} = \frac{17}{20}
\]


\[
P(\overline{A} \cap \overline{B} \cap \overline{C}) = P(A \cup B \cup C) = 1 - P(A \cup B \cup C) = \frac{3}{20}
\]

\[
P(\overline{A} \cap \overline{B} \cap C) = P(\overline{A} \cap \overline{B} \cap (\overline{S} - C)) = P(\overline{A} \cap \overline{B}) - P(\overline{A} \cap \overline{B} \cap C)
= \frac{4}{15} - \frac{3}{20} = \frac{16}{60} - \frac{9}{60} = \frac{7}{60}
\]

\[
p = P(\overline{A} \cap \overline{B} \cap C)
\]

\[
p = P(\overline{A} \cap \overline{B}) + P(C) - P(\overline{A} \cap \overline{B} \cap C)
\]

\[
p = \frac{4}{15} + \frac{1}{5} - \frac{7}{60} = \frac{7}{20}
\]

\end{ans}

(3) 已知 $P(A) = \frac{1}{2}$,(i) 若 $A, B$ 互不相容,求 $P(AB^c)$,(ii) 若 $P(AB) = \frac{1}{8}$,求 $P(AB^c)$。

\begin{ans}{7}{7}
不妨设 $ \Omega $ 为全集.

(i) $ P(AB) = P(A(\Omega - B)) = P(A - AB) = P(A) - P(AB) = \frac{1}{2} $

(ii)$ P(AB) = P(A(\Omega - B)) = P(A - AB) = P(A) - P(AB) = \frac{1}{2} - \frac{1}{8} = \frac{3}{8} $

\end{ans}

\subsection{习题4}

(1) 证明 $\overline{AB} = \overline{A}B$ 等价于 $A = B$:
\begin{ans}{8}{8}
    因为 $\overline{AB}$ 表示 $A$ 和 $B$ 都不发生的事件,且 $\overline{A}B$ 表示 $A$ 不发生但 $B$ 发生的事件。
    
    \[
    \overline{AB} = \overline{A}B \implies A = B
    \]
    
    根据集合的性质,这说明 $A$ 与 $B$ 是相同的事件。
\end{ans}

(2) 验证事件 $A$ 和事件 $B$ 恰有一个发生的概率:

\begin{ans}{9}{9}
    
    设事件 $A$ 或事件 $B$ 发生但不同时发生的概率为:
    
    \[
    P(A \cup B) - P(A \cap B) = P(A) + P(B) - 2P(AB)
    \]
    
    根据集合间的基本概率公式,可以得出:
    
    \[
    P(A \text{ 恰有一个发生}) = P(A) + P(B) - 2P(AB).
    \]
    
    证毕。
\end{ans}

\section{附加题}

\begin{itemize}
    \item 补充习题 1: 设 $A, B, C$ 为三个事件。已知 $P(A) = a, P(B) = 2a, P(C) = 3a, P(AB) = P(AC) = P(BC) = b$。证明:$a \leq \frac{1}{4}, b \leq \frac{1}{4}$。
    
    \item 补充习题 2: 证明 $\lvert P(AB) - P(A)P(B) \rvert \leq \frac{1}{4}$。
\end{itemize}

\subsection{附加题1}

\begin{proof}
    
    因为 $AB \subseteq A$,则 $P(AB) \leq P(A)$,即 $b \leq a$。

    由于 $1 \geq P(B \cup C) = P(B) + P(C) - P(BC) = 5a - b \geq 4a$,则 $a \leq \frac{1}{4}$。
    
    由 $b \leq a$ 且 $a \leq \frac{1}{4}$,得 $b \leq \frac{1}{4}$。

\end{proof}

\subsection{附加题2}

\begin{proof}

\noindent
设 $P(A) \geq P(B)$,则

\[
P(AB) - P(A)P(B) \leq P(B) - P(B)P(B) = P(B)[1 - P(B)] \leq (\frac{[P(B) + 1 - P(B)]}{2})^2 = \frac{1}{4}.
\]

\end{proof}
\newpage{}

\end{document}