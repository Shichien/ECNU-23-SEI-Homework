\documentclass[twoside]{article}
\usepackage[paperwidth=210mm, paperheight=297mm, margin=2cm]{geometry}
\usepackage[utf8]{inputenc}
\usepackage{ctex}

% Formatting Packages ——————————————————————————————————————
\usepackage{multicol}
\usepackage{multirow}
\usepackage{enumitem}
\usepackage{indentfirst}
\usepackage[toc]{multitoc}

% Math & Physics Packages ————————————————————————————
\usepackage{amsmath, amsthm, amsfonts, amssymb}
\usepackage{amssymb}
\usepackage{setspace}
\usepackage{physics}
\usepackage{cancel}
\usepackage{nicefrac}

% Image-related Packages —————————————————————————————
\usepackage{graphics, graphicx}
\usepackage{tikz}
\usetikzlibrary{arrows.meta}
\usepackage{pgfplots}
\pgfplotsset{compat=1.18}
\usepackage{xcolor}
\usepackage{fourier-orns}
\usepackage{lipsum}


%% Silakan otak-atik judul di sini
\title{\texorpdfstring{\vspace{-1.5em}}{}\textbf{概率论与数理统计}}
\author{张梓卫 10235101526}
\date{\the\day \ \monthyear}

% Kalau mau bab pertama nomornya 0, ganti 0 jadi -1
\setcounter{section}{0}

\renewcommand*\contentsname{第十三周概率论作业}
\renewcommand*{\multicolumntoc}{2}
\setlength{\columnseprule}{0.5pt}

% Layouting Packages ————————————————————————————————————————
\usepackage{titlesec}
\usepackage{fancyhdr}
\pagestyle{fancy}
\setlength{\headheight}{14.39996pt}
\fancyfoot[C]{\textit{By Deralive (10235101526)}}
\fancyfoot[R]{\thepage}

\fancyhead[LE, RO]{\textsl{\rightmark}}
\fancyhead[LO, RE]{\textsl{\leftmark}}

\renewcommand\headrule{
	\vspace{-6pt}
	\hrulefill
	\raisebox{-2.1pt}
	{\quad\floweroneleft\decoone\floweroneright\quad}
	\hrulefill}

\renewcommand\footrule{
	\hrulefill
	\raisebox{-2.1pt}
	{\quad\floweroneleft\decoone\floweroneright\quad}
	\hrulefill}
  
\newcommand{\fpb}[2]{\mathrm{FPB}(#1, #2)}
\newcommand{\kpk}[2]{\mathrm{KPK}(#1, #2)}

% Reference and Bibliography Packages ————————————————————
\usepackage{hyperref}
\hypersetup{
    colorlinks=true,
    linkcolor={black},
    citecolor={biru!70!black},
    urlcolor={biru!80!black}
}

\numberwithin{equation}{section}

% Silakan lihat dokumentasi package biblatex
% untuk format sitasi yang diperlukan
\usepackage[backend=biber]{biblatex}
\addbibresource{ref.bib}
\makeatletter
\def\@biblabel#1{}
\makeatother

\renewcommand{\mod}{\mathrm{mod} \ }

% Titling
\newcommand{\garis} [3] []{
	\begin{center}
		\begin{tikzpicture}
			\draw[#2-#3, ultra thick, #1] (0,0) to (1\linewidth,0);
		\end{tikzpicture}
	\end{center}
}

\newcommand{\monthyear}{
  \ifcase\month\or January\or February\or March\or April\or May\or June\or
  July\or August\or September\or October\or November\or
  December\fi\space\number\year
}

% Colour Palette ——————————————————————————————————————
\definecolor{merah}{HTML}{F4564E}
\definecolor{merahtua}{HTML}{89313E}
\definecolor{biru}{HTML}{60BBE5}
\definecolor{birutua}{HTML}{412F66}
\definecolor{hijau}{HTML}{59CC78}
\definecolor{hijautua}{HTML}{366D5B}
\definecolor{kuning}{HTML}{FFD56B}
\definecolor{jingga}{HTML}{FBA15F}
\definecolor{ungu}{HTML}{8C5FBF}
\definecolor{lavender}{HTML}{CBA5E8}
\definecolor{merjamb}{HTML}{FFB6E0}
\definecolor{mygray}{HTML}{E6E6E6}

% Theorems ————————————————————————————————————————————
\usepackage{tcolorbox}
\tcbuselibrary{skins,breakable,theorems}
\usepackage{changepage}

\newcounter{hitung}
\setcounter{hitung}{\thesection}

\makeatletter
	% Proof 证明如下
	\def\tcb@theo@widetitle#1#2#3{\hbox to \textwidth{\textsc{\large#1}\normalsize\space#3\hfil(#2)}}
	\tcbset{
		theorem style/theorem wide name and number/.code={ \let\tcb@theo@title=\tcb@theo@widetitle},
		proofbox/.style={skin=enhancedmiddle,breakable,parbox=false,boxrule=0mm,
			check odd page, toggle left and right, colframe=black!20!white!92!hijau,
			leftrule=8pt, rightrule=0mm, boxsep=0mm,arc=0mm, outer arc=0mm,
			left=3mm,right=3mm,top=0mm,bottom=0mm, toptitle=0mm,
			bottomtitle=0mm,colback=gray!3!white!98!biru, before skip=8pt, after skip=8pt,
			before={\par\vskip-2pt},after={\par\smallbreak},
		},
	}
	\newtcolorbox{ProofBox}{proofbox}
	\makeatother
	
	\let\realproof\proof
	\let\realendproof\endproof
	\renewenvironment{proof}[1][Prove:]{\ProofBox\strut\textsc{#1}\space}{\endProofBox}
        \AtEndEnvironment{proof}{\null\hfill$\blacksquare$}
        % Definition 定义环境
	\newtcbtheorem[use counter=hitung, number within=section]{dfn}{定义}
	{theorem style=theorem wide name and number,breakable,enhanced,arc=3.5mm,outer arc=3.5mm,
		boxrule=0pt,toprule=1pt,leftrule=0pt,bottomrule=1pt, rightrule=0pt,left=0.2cm,right=0.2cm,
		titlerule=0.5em,toptitle=0.1cm,bottomtitle=-0.1cm,top=0.2cm,
		colframe=white!10!biru,
		colback=white!90!biru,
		coltitle=white,
		shadow={1.3mm}{-1.3mm}{0mm}{gray!50!white}, % 添加阴影
        coltext=birutua!60!gray, title style={white!10!biru}, rbefoe skip=8pt, after skip=8pt,
		fonttitle=\bfseries,fontupper=\normalsize}{dfn}

	% 答题卡
	\newtcbtheorem[use counter=hitung, number within=section]{ans}{解答}
	{theorem style=theorem wide name and number,breakable,enhanced,arc=3.5mm,outer arc=3.5mm,
		boxrule=0pt,toprule=1pt,leftrule=0pt,bottomrule=1pt, rightrule=0pt,left=0.2cm,right=0.2cm,
		titlerule=0.5em,toptitle=0.1cm,bottomtitle=-0.1cm,top=0.2cm,
		colframe=white!10!biru,
		colback=white!90!biru,
		coltitle=white,
		shadow={1.3mm}{-1.3mm}{0mm}{gray!50!white}, % 添加阴影
        coltext=birutua!60!gray, title style={white!10!biru}, before skip=8pt, after skip=8pt,
		fonttitle=\bfseries,fontupper=\normalsize}{ans}

	% Axiom
	\newtcbtheorem[use counter=hitung, number within=section]{axm}{公理}
	{theorem style=theorem wide name and number,breakable,enhanced,arc=3.5mm,outer arc=3.5mm,
		boxrule=0pt,toprule=1pt,leftrule=0pt,bottomrule=1pt, rightrule=0pt,left=0.2cm,right=0.2cm,
		titlerule=0.5em,toptitle=0.1cm,bottomtitle=-0.1cm,top=0.2cm,
		colframe=white!10!biru,colback=white!90!biru,coltitle=white,
		shadow={1.3mm}{-1.3mm}{0mm}{gray!50!white!90}, % 添加阴影
        coltext=birutua!60!gray,title style={white!10!biru},before skip=8pt, after skip=8pt,
		fonttitle=\bfseries,fontupper=\normalsize}{axm}
 
	% Theorem
	\newtcbtheorem[use counter=hitung, number within=section]{thm}{定理}
	{theorem style=theorem wide name and number,breakable,enhanced,arc=3.5mm,outer arc=3.5mm,
		boxrule=0pt,toprule=1pt,leftrule=0pt,bottomrule=1pt, rightrule=0pt,left=0.2cm,right=0.2cm,
		titlerule=0.5em,toptitle=0.1cm,bottomtitle=-0.1cm,top=0.2cm,
		colframe=white!10!merah,colback=white!75!pink,coltitle=white, coltext=merahtua!80!merah,
		shadow={1.3mm}{-1.3mm}{0mm}{gray!50!white!90}, % 添加阴影
		title style={white!10!merah}, before skip=8pt, after skip=8pt,
		fonttitle=\bfseries,fontupper=\normalsize}{thm}
	
	% Proposition
	\newtcbtheorem[use counter=hitung, number within=section]{prp}{命题}
	{theorem style=theorem wide name and number,breakable,enhanced,arc=3.5mm,outer arc=3.5mm,
		boxrule=0pt,toprule=1pt,leftrule=0pt,bottomrule=1pt, rightrule=0pt,left=0.2cm,right=0.2cm,
		titlerule=0.5em,toptitle=0.1cm,bottomtitle=-0.1cm,top=0.2cm,
		colframe=white!10!hijau,colback=white!90!hijau,coltitle=white, coltext=hijautua!80!brown,
		shadow={1.3mm}{-1.3mm}{0mm}{gray!50!white}, % 添加阴影
		title style={white!10!hijau}, before skip=8pt, after skip=8pt,
		fonttitle=\bfseries,fontupper=\normalsize}{prp}


	% Example
	\newtcolorbox[use counter=hitung, number within=section]{cth}[1][]{breakable,
		colframe=white!10!jingga, coltitle=white!90!jingga, colback=white!85!jingga, coltext=black!10!brown!50!jingga, colbacktitle=white!10!jingga, enhanced, fonttitle=\bfseries,fontupper=\normalsize, attach boxed title to top left={yshift=-2mm}, before skip=8pt, after skip=8pt,
		title=Contoh~\thetcbcounter \ \ #1}

	% Catatan/Note
	\newtcolorbox{ctt}[1][]{enhanced, 
		left=4.1mm, borderline west={8pt}{0pt}{white!10!kuning}, 
		before skip=6pt, after skip=6pt, 
		colback=white!85!kuning, colframe= white!85!kuning, coltitle=orange!60!kuning!25!brown, coltext=orange!60!kuning!25!brown,
		fonttitle=\bfseries,fontupper=\normalsize, before skip=8pt, after skip=8pt,
		title=\underline{Catatan}  #1}
	
	% Komentar/Remark
	\newtcolorbox{rmr}[1][]{
		,arc=0mm,outer arc=0mm,
		boxrule=0pt,toprule=1pt,leftrule=0pt,bottomrule=5pt, rightrule=0pt,left=0.2cm,right=0.2cm,
		titlerule=0.5em,toptitle=0.1cm,bottomtitle=-0.1cm,top=0.2cm,
		colframe=white!10!kuning,colback=white!85!kuning,coltitle=white, coltext=orange!60!kuning,
		fonttitle=\bfseries,fontupper=\normalsize, before skip=8pt, after skip=8pt,
		title=Komentar  #1}

% ————————————————————————————————————————————————————————————————————————
\begin{document}

\maketitle
\vspace{-3.5em}
\garis{Kite}{Kite}

\tableofcontents

\section{第三章习题 20}

设 $X$ 和 $Y$ 是相互独立的随机变量,其概率密度分别为
\[
f_X(x) = 
\begin{cases} 
ae^{-ax}, & x > 0, \\ 
0, & x \leq 0 
\end{cases}, \quad 
f_Y(y) = 
\begin{cases} 
\mu e^{-\mu y}, & y > 0, \\ 
0, & y \leq 0 
\end{cases}
\]
其中 $a > 0, \mu > 0$ 是常数。引入随机变量
\[
Z = 
\begin{cases} 
1, & \text{当 } X \leq Y, \\ 
0, & \text{当 } X > Y 
\end{cases}
\]

\begin{enumerate}
    \item 求条件概率密度 $f_{X|Y}(x|y)$。
    \item 求 $Z$ 的分布律和分布函数。
\end{enumerate}


\begin{ans}{20}{20}
    \textbf{解:} 由于 $X$ 和 $Y$ 相互独立,$(X,Y)$ 的联合概率密度为 
\[
f(x,y) = f_X(x)f_Y(y) =
\begin{cases} 
\mu ae^{-ax-\mu y}, & x > 0, y > 0, \\ 
0, & \text{其他情况} 
\end{cases}
\]

\begin{enumerate}
    \item 当 $y > 0$ 时,有
    \[
    f_{X|Y}(x|y) = f_X(x) = 
    \begin{cases} 
    ae^{-ax}, & x > 0, \\ 
    0, & \text{其他情况} 
    \end{cases}
    \]
    
    \item 
    \[
    P(X \leq Y) = \int_{G: x \leq y} f(x,y)dxdy = \int_{0}^{\infty} \int_{0}^{y} f(x,y)dxdy 
    \]
    \[
    = \int_{0}^{\infty} \left[-\lambda e^{-ax-\mu y}\right]_{x=0}^{x=y} dy = \int_{0}^{\infty} -\lambda e^{-ay-\mu y} dy
    \]
    \[
    = \int_{0}^{\infty} ae^{-(\lambda + \mu)y} dy = \frac{\lambda}{\lambda + \mu}
    \]

    而且 
    \[
    P(X > Y) = 1 - P(X \leq Y) = \frac{\mu}{\lambda + \mu}
    \]

    故 $Z$ 的分布律为
    \[
    \begin{array}{|c|c|}
    \hline
    Z & P_Z \\
    \hline
    0 & \frac{\mu}{\lambda + \mu} \\
    1 & \frac{\lambda}{\lambda + \mu} \\
    \hline
    \end{array}
    \]

    $Z$ 的分布函数为
    \[
    F_Z(z) = 
    \begin{cases} 
    0, & z < 0, \\ 
    \frac{\mu}{\lambda + \mu}, & 0 \leq z < 1, \\ 
    1, & z \geq 1 
    \end{cases}
    \]
\end{enumerate}

\end{ans}

\section{第二章习题 22}

\section*{22. 设 $X$ 和 $Y$ 是两个相互独立的随机变量,其概率密度分别为}
\[
f_X(x) = 
\begin{cases} 
1, & 0 \leq x \leq 1, \\ 
0, & \text{其他情况} 
\end{cases}, \quad 
f_Y(y) = 
\begin{cases} 
e^{-y}, & y > 0, \\ 
0, & \text{其他情况} 
\end{cases}
\]

\textbf{求随机变量} $Z = X + Y$ \textbf{的概率密度}。

\begin{ans}{22}{22}

    \textbf{解:} 利用公示:
\[
f_{Z}(z) = \int_{-\infty}^{\infty} f_X(x) f_Y(z - x) dx
\]

\textbf{因此,} $f_{Z}(z)$ 的定义如下:
\[
f_Z(z) = \int_{0}^{1} f_X(x) f_Y(z - x) dx = \int_{0}^{1} f_X(x) e^{-(z-x)} dx
\]
\[
= \int_{0}^{1} 1 \cdot e^{-(z-x)} dx = \int_{0}^{1} e^{-z+x} dx
\]

\textbf{则有:}
\[
f_Z(z) = 
\begin{cases} 
\left(1 - e^{-z}\right), & 0 < z < 1, \\ 
0, & z \leq 0 \text{ 或 } z \geq 1 
\end{cases}
\]

若利用公式
\[
f_Z(z) = \int_{0}^{1} f_X(x) f_Y(z - x) dx
\]
可以得出:
\[
f_Z(z) = \int_{0}^{1} f_Y(z-x)dx
\]

\textbf{条件:}
\begin{itemize}
    \item 当 $0 < z < 1$ 时,
    \[
    F_Z(z) = P(Z \leq z) = P(X + Y \leq z) = \int_{0}^{z} f_X(x)dxdy = \int_{0}^{z} 1 \cdot dy = z
    \]
    
    \item 当 $z \geq 1$ 时,
    \[
    F_Z(z) = P(Z \leq z) = P(X + Y \leq z) = \int_{0}^{z} e^{-(z-x)}dx
    \]
\end{itemize}

\textbf{得到:}
\[
F_Z(z) = 
\begin{cases} 
0, & z < 0, \\ 
1 - e^{-z}, & 0 < z < 1, \\ 
1 - (1 - e^{-1}), & z \geq 1 
\end{cases}
\]

\end{ans}

\section{第一章习题 23}

某种商品周围的需求量是一个随机变量,其概率密度为
\[
f(t) = 
\begin{cases} 
te^{-t}, & t > 0, \\ 
0, & t \leq 0 
\end{cases}
\]

设各周的需求量是相互独立的。求 (1) 两周,(2) 三周的需求量的概率密度。

\begin{ans}{23}{23}
    \textbf{解:} 设某种商品在第 $i$ 周的需求量为 $X_i\,(i=1,2,3)$,由题设 $X_1, X_2, X_3$ 相互独立,并且

    \[
    f_{X_i}(t) = f(t) =
    \begin{cases} 
    te^{-t}, & t > 0, \\ 
    0, & t \leq 0 
    \end{cases}
    \]
    
    (1) 记两周的需求量 $Z$,即 $Z = X_1 + X_2$,则 $Z$ 的概率密度为
    \[
    f_Z(z) = \int_0^{\infty} f(x) f(z-x) dx.
    \]
    
    由 $f(t)$ 的定义,知
    \[
    f_Z(z) = 
    \begin{cases} 
    \int_0^{z} f(x) f(z-x) dx, & z > 0, \\ 
    0, & \text{其他情况} 
    \end{cases}
    \]
    
    计算 $f_Z(z)$ 时,上述积分的被积函数不等于零,于是
    \[
    f_Z(z) = \int_0^{z} xe^{-(z-x)}e^{-x} dx = e^{-z} \int_0^{z} x e^{-2x} dx.
    \]
    
    此时,可以利用分部积分法求解,得到
    \[
    f_Z(z) = e^{-z} \left[-\frac{1}{2}e^{-2x}\right]_{0}^{z} = e^{-z} \cdot \frac{1}{2}(1 - e^{-2z}).
    \]
    
    (2) 记三周的需求量 $W$,即 $W = Z + X_3$,因 $X_1, X_2, X_3$ 相互独立,故 $Z = X_1 + X_2$ 也相互独立,从而 $W$ 的概率密度为
    \[
    f_W(u) = \int_0^{\infty} f_Z(z) f_{X_3}(u - z) dz.
    \]
    
    由上述 $f_Z(z)$ 及 $f(t)$ 的定义,知
    \[
    f_W(u) = 
    \begin{cases} 
    \int_0^{u} f_Z(z) f_{X_3}(u - z) dz, & u > 0, \\ 
    0, & \text{其他情况} 
    \end{cases}
    \]
    
    在计算时,上述积分的被积函数不等于零,因此 $W$ 的概率密度为
    \[
    f_W(u) = \int_0^{u} \left[\frac{1}{2}e^{-z}(1 - e^{-2z})\right]e^{-(u-z)} dz = e^{-u} \int_0^{u} \frac{1}{2} (1 - e^{-2z}) dz.
    \]
    
    计算这个积分,得到:
    \[
    f_W(u) = e^{-u} \left[\frac{u}{2} + \frac{1}{4}(1 - e^{-2u})\right] = e^{-u} \cdot \frac{u}{2}, \quad u > 0.
    \]
\end{ans}

\section{第三章习题 30}

设某种型号的电子元件的寿命(以 h 计)近似地服从正态分布 $N(160, 20^2)$,随机地选择 4 只,求其中没有一只寿命小于 180 h 的概率。

\textbf{解:} 以 $X_i\,(i = 1,2,3,4)$ 记所选取的第 $i$ 只元件的寿命,由题设一只元件寿命小于 180 h 的概率为
\[
P\{X_i \leq 180\} = P\left(\frac{X_i - 160}{20} \leq \frac{180 - 160}{20}\right) = \Phi\left(\frac{180 - 160}{20}\right) = \Phi(1) \approx 0.8413.
\]

\begin{ans}{30}{30}
可认为 $X_1, X_2, X_3, X_4$ 相互独立,故选取的 4 只元件没有一只寿命小于 180 h 的概率为
\[
P\left(\bigcap_{i=1}^{4} \{X_i > 180\}\right) = \prod_{i=1}^{4} (1 - P\{X_i \leq 180\}) = (1 - 0.8413)^4 \approx 0.000063.
\]
\end{ans}
\end{document}
