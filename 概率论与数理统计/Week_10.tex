\documentclass[twoside]{article}
\usepackage[paperwidth=210mm, paperheight=297mm, margin=2cm]{geometry}
\usepackage[utf8]{inputenc}
\usepackage{ctex}

% Formatting Packages ——————————————————————————————————————
\usepackage{multicol}
\usepackage{multirow}
\usepackage{enumitem}
\usepackage{indentfirst}
\usepackage[toc]{multitoc}

% Math & Physics Packages ————————————————————————————
\usepackage{amsmath, amsthm, amsfonts, amssymb}
\usepackage{amssymb}
\usepackage{setspace}
\usepackage{physics}
\usepackage{cancel}
\usepackage{nicefrac}

% Image-related Packages —————————————————————————————
\usepackage{graphics, graphicx}
\usepackage{tikz}
\usetikzlibrary{arrows.meta}
\usepackage{pgfplots}
\pgfplotsset{compat=1.18}
\usepackage{xcolor}
\usepackage{fourier-orns}
\usepackage{lipsum}


%% Silakan otak-atik judul di sini
\title{\texorpdfstring{\vspace{-1.5em}}{}\textbf{概率论与数理统计}}
\author{张梓卫 10235101526}
\date{\the\day \ \monthyear}

% Kalau mau bab pertama nomornya 0, ganti 0 jadi -1
\setcounter{section}{0}

\renewcommand*\contentsname{第十三周概率论作业}
\renewcommand*{\multicolumntoc}{2}
\setlength{\columnseprule}{0.5pt}

% Layouting Packages ————————————————————————————————————————
\usepackage{titlesec}
\usepackage{fancyhdr}
\pagestyle{fancy}
\setlength{\headheight}{14.39996pt}
\fancyfoot[C]{\textit{By Deralive (10235101526)}}
\fancyfoot[R]{\thepage}

\fancyhead[LE, RO]{\textsl{\rightmark}}
\fancyhead[LO, RE]{\textsl{\leftmark}}

\renewcommand\headrule{
	\vspace{-6pt}
	\hrulefill
	\raisebox{-2.1pt}
	{\quad\floweroneleft\decoone\floweroneright\quad}
	\hrulefill}

\renewcommand\footrule{
	\hrulefill
	\raisebox{-2.1pt}
	{\quad\floweroneleft\decoone\floweroneright\quad}
	\hrulefill}
  
\newcommand{\fpb}[2]{\mathrm{FPB}(#1, #2)}
\newcommand{\kpk}[2]{\mathrm{KPK}(#1, #2)}

% Reference and Bibliography Packages ————————————————————
\usepackage{hyperref}
\hypersetup{
    colorlinks=true,
    linkcolor={black},
    citecolor={biru!70!black},
    urlcolor={biru!80!black}
}

\numberwithin{equation}{section}

% Silakan lihat dokumentasi package biblatex
% untuk format sitasi yang diperlukan
\usepackage[backend=biber]{biblatex}
\addbibresource{ref.bib}
\makeatletter
\def\@biblabel#1{}
\makeatother

\renewcommand{\mod}{\mathrm{mod} \ }

% Titling
\newcommand{\garis} [3] []{
	\begin{center}
		\begin{tikzpicture}
			\draw[#2-#3, ultra thick, #1] (0,0) to (1\linewidth,0);
		\end{tikzpicture}
	\end{center}
}

\newcommand{\monthyear}{
  \ifcase\month\or January\or February\or March\or April\or May\or June\or
  July\or August\or September\or October\or November\or
  December\fi\space\number\year
}

% Colour Palette ——————————————————————————————————————
\definecolor{merah}{HTML}{F4564E}
\definecolor{merahtua}{HTML}{89313E}
\definecolor{biru}{HTML}{60BBE5}
\definecolor{birutua}{HTML}{412F66}
\definecolor{hijau}{HTML}{59CC78}
\definecolor{hijautua}{HTML}{366D5B}
\definecolor{kuning}{HTML}{FFD56B}
\definecolor{jingga}{HTML}{FBA15F}
\definecolor{ungu}{HTML}{8C5FBF}
\definecolor{lavender}{HTML}{CBA5E8}
\definecolor{merjamb}{HTML}{FFB6E0}
\definecolor{mygray}{HTML}{E6E6E6}

% Theorems ————————————————————————————————————————————
\usepackage{tcolorbox}
\tcbuselibrary{skins,breakable,theorems}
\usepackage{changepage}

\newcounter{hitung}
\setcounter{hitung}{\thesection}

\makeatletter
	% Proof 证明如下
	\def\tcb@theo@widetitle#1#2#3{\hbox to \textwidth{\textsc{\large#1}\normalsize\space#3\hfil(#2)}}
	\tcbset{
		theorem style/theorem wide name and number/.code={ \let\tcb@theo@title=\tcb@theo@widetitle},
		proofbox/.style={skin=enhancedmiddle,breakable,parbox=false,boxrule=0mm,
			check odd page, toggle left and right, colframe=black!20!white!92!hijau,
			leftrule=8pt, rightrule=0mm, boxsep=0mm,arc=0mm, outer arc=0mm,
			left=3mm,right=3mm,top=0mm,bottom=0mm, toptitle=0mm,
			bottomtitle=0mm,colback=gray!3!white!98!biru, before skip=8pt, after skip=8pt,
			before={\par\vskip-2pt},after={\par\smallbreak},
		},
	}
	\newtcolorbox{ProofBox}{proofbox}
	\makeatother
	
	\let\realproof\proof
	\let\realendproof\endproof
	\renewenvironment{proof}[1][Prove:]{\ProofBox\strut\textsc{#1}\space}{\endProofBox}
        \AtEndEnvironment{proof}{\null\hfill$\blacksquare$}
        % Definition 定义环境
	\newtcbtheorem[use counter=hitung, number within=section]{dfn}{定义}
	{theorem style=theorem wide name and number,breakable,enhanced,arc=3.5mm,outer arc=3.5mm,
		boxrule=0pt,toprule=1pt,leftrule=0pt,bottomrule=1pt, rightrule=0pt,left=0.2cm,right=0.2cm,
		titlerule=0.5em,toptitle=0.1cm,bottomtitle=-0.1cm,top=0.2cm,
		colframe=white!10!biru,
		colback=white!90!biru,
		coltitle=white,
		shadow={1.3mm}{-1.3mm}{0mm}{gray!50!white}, % 添加阴影
        coltext=birutua!60!gray, title style={white!10!biru}, rbefoe skip=8pt, after skip=8pt,
		fonttitle=\bfseries,fontupper=\normalsize}{dfn}

	% 答题卡
	\newtcbtheorem[use counter=hitung, number within=section]{ans}{解答}
	{theorem style=theorem wide name and number,breakable,enhanced,arc=3.5mm,outer arc=3.5mm,
		boxrule=0pt,toprule=1pt,leftrule=0pt,bottomrule=1pt, rightrule=0pt,left=0.2cm,right=0.2cm,
		titlerule=0.5em,toptitle=0.1cm,bottomtitle=-0.1cm,top=0.2cm,
		colframe=white!10!biru,
		colback=white!90!biru,
		coltitle=white,
		shadow={1.3mm}{-1.3mm}{0mm}{gray!50!white}, % 添加阴影
        coltext=birutua!60!gray, title style={white!10!biru}, before skip=8pt, after skip=8pt,
		fonttitle=\bfseries,fontupper=\normalsize}{ans}

	% Axiom
	\newtcbtheorem[use counter=hitung, number within=section]{axm}{公理}
	{theorem style=theorem wide name and number,breakable,enhanced,arc=3.5mm,outer arc=3.5mm,
		boxrule=0pt,toprule=1pt,leftrule=0pt,bottomrule=1pt, rightrule=0pt,left=0.2cm,right=0.2cm,
		titlerule=0.5em,toptitle=0.1cm,bottomtitle=-0.1cm,top=0.2cm,
		colframe=white!10!biru,colback=white!90!biru,coltitle=white,
		shadow={1.3mm}{-1.3mm}{0mm}{gray!50!white!90}, % 添加阴影
        coltext=birutua!60!gray,title style={white!10!biru},before skip=8pt, after skip=8pt,
		fonttitle=\bfseries,fontupper=\normalsize}{axm}
 
	% Theorem
	\newtcbtheorem[use counter=hitung, number within=section]{thm}{定理}
	{theorem style=theorem wide name and number,breakable,enhanced,arc=3.5mm,outer arc=3.5mm,
		boxrule=0pt,toprule=1pt,leftrule=0pt,bottomrule=1pt, rightrule=0pt,left=0.2cm,right=0.2cm,
		titlerule=0.5em,toptitle=0.1cm,bottomtitle=-0.1cm,top=0.2cm,
		colframe=white!10!merah,colback=white!75!pink,coltitle=white, coltext=merahtua!80!merah,
		shadow={1.3mm}{-1.3mm}{0mm}{gray!50!white!90}, % 添加阴影
		title style={white!10!merah}, before skip=8pt, after skip=8pt,
		fonttitle=\bfseries,fontupper=\normalsize}{thm}
	
	% Proposition
	\newtcbtheorem[use counter=hitung, number within=section]{prp}{命题}
	{theorem style=theorem wide name and number,breakable,enhanced,arc=3.5mm,outer arc=3.5mm,
		boxrule=0pt,toprule=1pt,leftrule=0pt,bottomrule=1pt, rightrule=0pt,left=0.2cm,right=0.2cm,
		titlerule=0.5em,toptitle=0.1cm,bottomtitle=-0.1cm,top=0.2cm,
		colframe=white!10!hijau,colback=white!90!hijau,coltitle=white, coltext=hijautua!80!brown,
		shadow={1.3mm}{-1.3mm}{0mm}{gray!50!white}, % 添加阴影
		title style={white!10!hijau}, before skip=8pt, after skip=8pt,
		fonttitle=\bfseries,fontupper=\normalsize}{prp}


	% Example
	\newtcolorbox[use counter=hitung, number within=section]{cth}[1][]{breakable,
		colframe=white!10!jingga, coltitle=white!90!jingga, colback=white!85!jingga, coltext=black!10!brown!50!jingga, colbacktitle=white!10!jingga, enhanced, fonttitle=\bfseries,fontupper=\normalsize, attach boxed title to top left={yshift=-2mm}, before skip=8pt, after skip=8pt,
		title=Contoh~\thetcbcounter \ \ #1}

	% Catatan/Note
	\newtcolorbox{ctt}[1][]{enhanced, 
		left=4.1mm, borderline west={8pt}{0pt}{white!10!kuning}, 
		before skip=6pt, after skip=6pt, 
		colback=white!85!kuning, colframe= white!85!kuning, coltitle=orange!60!kuning!25!brown, coltext=orange!60!kuning!25!brown,
		fonttitle=\bfseries,fontupper=\normalsize, before skip=8pt, after skip=8pt,
		title=\underline{Catatan}  #1}
	
	% Komentar/Remark
	\newtcolorbox{rmr}[1][]{
		,arc=0mm,outer arc=0mm,
		boxrule=0pt,toprule=1pt,leftrule=0pt,bottomrule=5pt, rightrule=0pt,left=0.2cm,right=0.2cm,
		titlerule=0.5em,toptitle=0.1cm,bottomtitle=-0.1cm,top=0.2cm,
		colframe=white!10!kuning,colback=white!85!kuning,coltitle=white, coltext=orange!60!kuning,
		fonttitle=\bfseries,fontupper=\normalsize, before skip=8pt, after skip=8pt,
		title=Komentar  #1}

% ————————————————————————————————————————————————————————————————————————
\begin{document}

\maketitle
\vspace{-3.5em}
\garis{Kite}{Kite}

\tableofcontents

\section{第四章习题 6}

Given the following probability distribution for the random variable \( X \):

\[
\begin{array}{c|ccc}
X & -2 & 0 & 2 \\
\hline
P(X) & 0.4 & 0.3 & 0.3 \\
\end{array}
\]

1. Calculate \( E(X) \), \( E(X^2) \), and \( E(3X^2 + 5) \).
2. If \( X \sim \pi(\lambda) \), find \( E\left(\frac{1}{X+1}\right) \).

\begin{ans}{6}{6}

    1.
    \[
    E(X) = \sum_{i} x_i \cdot P(X = x_i)
    \]
    代入值计算:
    \[
    E(X) = (-2) \cdot 0.4 + 0 \cdot 0.3 + 2 \cdot 0.3 = -0.8 + 0 + 0.6 = -0.2
    \]
 
    2.
    \[
    E(X^2) = \sum_{i} x_i^2 \cdot P(X = x_i)
    \]
    代入值计算:
    \[
    E(X^2) = (-2)^2 \cdot 0.4 + 0^2 \cdot 0.3 + 2^2 \cdot 0.3 = 4 \cdot 0.4 + 0 + 4 \cdot 0.3 = 1.6 + 0 + 1.2 = 2.8
    \]

    3.
    根据期望的线性性质,有:
    \[
    E(3X^2 + 5) = 3E(X^2) + 5
    \]
    代入 \( E(X^2) = 2.8 \):
    \[
    E(3X^2 + 5) = 3 \cdot 2.8 + 5 = 8.4 + 5 = 13.4
    \]
\end{ans}

\section{第四章习题 7}

7. (1) 设随机变量 \( X \) 的概率密度为:
   \[
   f(x) = 
   \begin{cases}
      e^{-x}, & x > 0, \\
      0, & x \leq 0.
   \end{cases}
   \]
   求以下的数学期望:
   \begin{itemize}
       \item[(i)] \( Y = 2X \)
       \item[(ii)] \( Y = e^{-2X} \)
   \end{itemize}

   (2) 设随机变量 \( X_1, X_2, \dots, X_n \) 相互独立,且都服从 \( (0,1) \) 上的均匀分布。
   \begin{itemize}
       \item[(i)] 求 \( U = \max\{X_1, X_2, \dots, X_n\} \) 的数学期望。
       \item[(ii)] 求 \( V = \min\{X_1, X_2, \dots, X_n\} \) 的数学期望。
   \end{itemize}

\begin{ans}{7}{7}

    1. 计算 \( Y = 2X \) 的数学期望:

   由于 \( Y = 2X \),我们有:
   \[
   E(Y) = E(2X) = 2E(X)
   \]
   对于参数为 \( \lambda = 1 \) 的指数分布,期望值 \( E(X) = \frac{1}{\lambda} = 1 \)。因此,
   \[
   E(Y) = 2 \cdot 1 = 2
   \]

2. 计算 \( Y = e^{-2X} \) 的数学期望:

   要计算 \( E(Y) = E(e^{-2X}) \),我们可以这样求解:
   \[
   E(e^{-2X}) = \int_{0}^{\infty} e^{-2x} f(x) \, dx = \int_{0}^{\infty} e^{-2x} \cdot e^{-x} \, dx = \int_{0}^{\infty} e^{-3x} \, dx
   \]
   计算该积分:
   \[
   E(e^{-2X}) = \int_{0}^{\infty} e^{-3x} \, dx = \left[ -\frac{1}{3} e^{-3x} \right]_{0}^{\infty} = \frac{1}{3}
   \]

\subsection*{第(2)问}

1. 计算 \( U = \max\{X_1, X_2, \dots, X_n\} \) 的数学期望:

   对于 \( X_i \sim \text{Uniform}(0,1) \), \( U = \max\{X_1, X_2, \dots, X_n\} \) 的累积分布函数(CDF)为:
   \[
   P(U \leq u) = P(X_1 \leq u, X_2 \leq u, \dots, X_n \leq u) = P(X_i \leq u)^n = u^n
   \]
   因此,\( U \) 的概率密度函数(PDF)为:
   \[
   f_U(u) = \frac{d}{du} P(U \leq u) = \frac{d}{du} u^n = n u^{n-1}, \quad 0 < u < 1
   \]
   \( U \) 的数学期望为:
   \[
   E(U) = \int_{0}^{1} u \cdot f_U(u) \, du = \int_{0}^{1} u \cdot n u^{n-1} \, du = n \int_{0}^{1} u^n \, du
   \]
   计算该积分:
   \[
   E(U) = n \cdot \frac{1}{n+1} = \frac{n}{n+1}
   \]

2. 计算 \( V = \min\{X_1, X_2, \dots, X_n\} \) 的数学期望:

   对于 \( X_i \sim \text{Uniform}(0,1) \),\( V = \min\{X_1, X_2, \dots, X_n\} \) 的累积分布函数为:
   \[
   P(V > v) = P(X_1 > v, X_2 > v, \dots, X_n > v) = P(X_i > v)^n = (1 - v)^n
   \]
   因此,\( V \) 的概率密度函数为:
   \[
   f_V(v) = \frac{d}{dv} (1 - v)^n = -n (1 - v)^{n-1}, \quad 0 < v < 1
   \]
   \( V \) 的数学期望为:
   \[
   E(V) = \int_{0}^{1} v \cdot f_V(v) \, dv = \int_{0}^{1} v \cdot n (1 - v)^{n-1} \, dv
   \]
   使用分部积分或 Beta 函数的已知结果,我们得到:
   \[
   E(V) = \frac{1}{n+1}
   \]
\end{ans}

\section{第四章习题 9}

9. (1) 设随机变量 \( (X, Y) \) 的概率密度为
   \[
   f(x, y) = 
   \begin{cases}
      12y^2, & 0 \leq y \leq x \leq 1, \\
      0, & \text{其他}.
   \end{cases}
   \]
   求 \( E(X) \), \( E(Y) \), \( E(XY) \), \( E(X^2 + Y^2) \)。

   (2) 设随机变量 \( X, Y \) 的联合概率密度为
   \[
   f(x, y) = 
   \begin{cases}
      \frac{1}{y} e^{-(x + \frac{y}{y})}, & x > 0, y > 0, \\
      0, & \text{其他}.
   \end{cases}
   \]
   求 \( E(X) \), \( E(Y) \), \( E(XY) \)。

\begin{ans}{9}{9}

   根据期望的定义,\( E(X) \) 的表达式为:
   \[
   E(X) = \iint_{D} x f(x, y) \, dx \, dy,
   \]
   其中区域 \( D \) 满足 \( 0 \leq y \leq x \leq 1 \)。将密度函数 \( f(x, y) = 12y^2 \) 代入,得到:
   \[
   E(X) = \int_{0}^{1} \int_{y}^{1} x \cdot 12y^2 \, dx \, dy = \frac{4}{5}
   \]


   同理,\( E(Y) \), \( E(XY) \) 的计算公式为:
   \[
   E(Y) = \iint_{D} y f(x, y) \, dx \, dy = \int_{0}^{1} \int_{y}^{1} y \cdot 12y^2 \, dx \, dy = \frac{3}{5}
   \]

   \[
   E(XY) = \iint_{D} x y f(x, y) \, dx \, dy = \int_{0}^{1} \int_{y}^{1} x y \cdot 12y^2 \, dx \, dy = \frac{1}{2}
   \]

   使用期望的线性性质,\( E(X^2 + Y^2) = E(X^2) + E(Y^2) \),其中:
   \[
   E(X^2) = \iint_{D} x^2 f(x, y) \, dx \, dy = \int_{0}^{1} \int_{y}^{1} x^2 \cdot 12y^2 \, dx \, dy,
   \]
   \[
   E(Y^2) = \iint_{D} y^2 f(x, y) \, dx \, dy = \int_{0}^{1} \int_{y}^{1} y^2 \cdot 12y^2 \, dx \, dy.
   \]
   代入相加可得:
   \[
   E(X^2 + Y^2) = E(X^2) + E(Y^2) = \frac{16}{15}
   \]

\subsection*{第(2)问}

    1.
   对于联合密度函数 \( f(x, y) = \frac{1}{y} e^{-(x + \frac{y}{y})} \),我们可以计算 \( E(X) \):
   \[
   E(X) = \int_{0}^{\infty} \int_{0}^{\infty} x f(x, y) \, dx \, dy = \int_{0}^{\infty} \int_{0}^{\infty} x \cdot \frac{1}{y} e^{-(y + \frac{x}{y})} \, dx \, dy.
   \]

    2.与上同理
   \[
   E(Y) = \int_{0}^{\infty} \int_{0}^{\infty} y f(x, y) \, dx \, dy = \int_{0}^{\infty} \int_{0}^{\infty} y \cdot \frac{1}{y} e^{-(y + \frac{x}{y})} \, dx \, dy.
   \]

    3.
   \[
   E(XY) = \int_{0}^{\infty} \int_{0}^{\infty} x y f(x, y) \, dx \, dy = \int_{0}^{\infty} \int_{0}^{\infty} x y \cdot \frac{1}{y} e^{-(x + \frac{y}{y})} \, dx \, dy.
   \]
\end{ans}

\section{第四章习题 12}

某车间生产的圆盘直径在区间 (a,b) 上服从均匀分布,试求圆盘面积的数学期望。

\begin{ans}{12}{12}

   设圆盘的直径为随机变量 \( D \),其在区间 \( (a, b) \) 上服从均匀分布。因此,概率密度函数 \( f(D) \) 为:
   \[
   f(D) = \frac{1}{b - a}, \quad a \leq D \leq b.
   \]

   圆盘的面积 \( A \) 可以表示为:
   \[
   A = \pi \left( \frac{D}{2} \right)^2 = \frac{\pi}{4} D^2.
   \]

   则:
   \[
   E(A) = E\left( \frac{\pi}{4} D^2 \right) = \frac{\pi}{4} E(D^2).
   \]


   因为 \( D \) 在 \( (a, b) \) 上均匀分布,所以:
   \[
   E(D^2) = \int_{a}^{b} D^2 \cdot f(D) \, dD = \int_{a}^{b} D^2 \cdot \frac{1}{b - a} \, dD.
   \]
   计算该积分:
   \[
   E(D^2) = \frac{1}{b - a} \int_{a}^{b} D^2 \, dD.
   \]
   对 \( D^2 \) 积分,得到:
   \[
   \int D^2 \, dD = \frac{D^3}{3},
   \]
   因此,
   \[
   E(D^2) = \frac{1}{b - a} \left[ \frac{D^3}{3} \right]_{a}^{b} = \frac{1}{b - a} \cdot \frac{b^3 - a^3}{3} = \frac{b^3 - a^3}{3(b - a)}.
   \]


   代入 \( E(D^2) \) 的值,我们得到:
   \[
   E(A) = \frac{\pi}{4} \cdot \frac{b^3 - a^3}{3(b - a)} = \frac{\pi (b^3 - a^3)}{12(b - a)}.
   \]

   通过分解 \( b^3 - a^3 \) 为 \( (b - a)(b^2 + ab + a^2) \),可以进一步简化:
   \[
   E(A) = \frac{\pi}{12} (b^2 + ab + a^2).
   \]
\end{ans}

\section{第四章习题 14}

设随机变量 \( X_1, X_2 \) 的概率密度分别为
\[
f_1(x) = 
\begin{cases}
2e^{-2x}, & x > 0, \\
0, & x \leq 0,
\end{cases}
\]
\[
f_2(x) = 
\begin{cases}
4e^{-4x}, & x > 0, \\
0, & x \leq 0.
\end{cases}
\]

\begin{ans}{14}{14}

    1. 计算 \( E(X_1 + X_2) \):

    根据期望的线性性质,\( E(X_1 + X_2) = E(X_1) + E(X_2) \)。
 
    - \( X_1 \) 服从指数分布,概率密度为 \( f_1(x) = 2e^{-2x} \),因此 \( X_1 \) 的期望为:
      \[
      E(X_1) = \int_0^{\infty} x \cdot 2e^{-2x} \, dx = \frac{1}{2}.
      \]
 
    - \( X_2 \) 也服从指数分布,概率密度为 \( f_2(x) = 4e^{-4x} \),因此 \( X_2 \) 的期望为:
      \[
      E(X_2) = \int_0^{\infty} x \cdot 4e^{-4x} \, dx = \frac{1}{4}.
      \]
 
    所以,
    \[
    E(X_1 + X_2) = E(X_1) + E(X_2) = \frac{1}{2} + \frac{1}{4} = \frac{3}{4}.
    \]
 
 2. 计算 \( E(2X_1 - 3X_2^2) \):
 
    根据期望的线性性质,\( E(2X_1 - 3X_2^2) = 2E(X_1) - 3E(X_2^2) \)。
 
    - 已知 \( E(X_1) = \frac{1}{2} \),所以 \( 2E(X_1) = 2 \times \frac{1}{2} = 1 \)。
    
    - 计算 \( E(X_2^2) \)。对于指数分布的 \( X_2 \)(参数为 4),我们有:
      \[
      E(X_2^2) = \int_0^{\infty} x^2 \cdot 4e^{-4x} \, dx = \frac{2}{16} = \frac{1}{8}.
      \]
 
    所以,
    \[
    E(2X_1 - 3X_2^2) = 1 - 3 \cdot \frac{1}{8} = 1 - \frac{3}{8} = \frac{5}{8}.
    \]
 
 \subsection*{第(2)问}
 
 若 \( X_1 \) 和 \( X_2 \) 相互独立,则 \( E(X_1 X_2) = E(X_1) \cdot E(X_2) \)。
 
 - 已知 \( E(X_1) = \frac{1}{2} \) 和 \( E(X_2) = \frac{1}{4} \)。
 
 因此,
 \[
 E(X_1 X_2) = E(X_1) \cdot E(X_2) = \frac{1}{2} \cdot \frac{1}{4} = \frac{1}{8}.
 \]

\end{ans}

\end{document}