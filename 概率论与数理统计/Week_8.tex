\documentclass[twoside]{article}
\usepackage[paperwidth=210mm, paperheight=297mm, margin=2cm]{geometry}
\usepackage[utf8]{inputenc}
\usepackage{ctex}

% Formatting Packages ——————————————————————————————————————
\usepackage{multicol}
\usepackage{multirow}
\usepackage{enumitem}
\usepackage{indentfirst}
\usepackage[toc]{multitoc}

% Math & Physics Packages ————————————————————————————
\usepackage{amsmath, amsthm, amsfonts, amssymb}
\usepackage{amssymb}
\usepackage{setspace}
\usepackage{physics}
\usepackage{cancel}
\usepackage{nicefrac}

% Image-related Packages —————————————————————————————
\usepackage{graphics, graphicx}
\usepackage{tikz}
\usetikzlibrary{arrows.meta}
\usepackage{pgfplots}
\pgfplotsset{compat=1.18}
\usepackage{xcolor}
\usepackage{fourier-orns}
\usepackage{lipsum}


%% Silakan otak-atik judul di sini
\title{\texorpdfstring{\vspace{-1.5em}}{}\textbf{概率论与数理统计}}
\author{张梓卫 10235101526}
\date{\the\day \ \monthyear}

% Kalau mau bab pertama nomornya 0, ganti 0 jadi -1
\setcounter{section}{0}

\renewcommand*\contentsname{第十三周概率论作业}
\renewcommand*{\multicolumntoc}{2}
\setlength{\columnseprule}{0.5pt}

% Layouting Packages ————————————————————————————————————————
\usepackage{titlesec}
\usepackage{fancyhdr}
\pagestyle{fancy}
\setlength{\headheight}{14.39996pt}
\fancyfoot[C]{\textit{By Deralive (10235101526)}}
\fancyfoot[R]{\thepage}

\fancyhead[LE, RO]{\textsl{\rightmark}}
\fancyhead[LO, RE]{\textsl{\leftmark}}

\renewcommand\headrule{
	\vspace{-6pt}
	\hrulefill
	\raisebox{-2.1pt}
	{\quad\floweroneleft\decoone\floweroneright\quad}
	\hrulefill}

\renewcommand\footrule{
	\hrulefill
	\raisebox{-2.1pt}
	{\quad\floweroneleft\decoone\floweroneright\quad}
	\hrulefill}
  
\newcommand{\fpb}[2]{\mathrm{FPB}(#1, #2)}
\newcommand{\kpk}[2]{\mathrm{KPK}(#1, #2)}

% Reference and Bibliography Packages ————————————————————
\usepackage{hyperref}
\hypersetup{
    colorlinks=true,
    linkcolor={black},
    citecolor={biru!70!black},
    urlcolor={biru!80!black}
}

\numberwithin{equation}{section}

% Silakan lihat dokumentasi package biblatex
% untuk format sitasi yang diperlukan
\usepackage[backend=biber]{biblatex}
\addbibresource{ref.bib}
\makeatletter
\def\@biblabel#1{}
\makeatother

\renewcommand{\mod}{\mathrm{mod} \ }

% Titling
\newcommand{\garis} [3] []{
	\begin{center}
		\begin{tikzpicture}
			\draw[#2-#3, ultra thick, #1] (0,0) to (1\linewidth,0);
		\end{tikzpicture}
	\end{center}
}

\newcommand{\monthyear}{
  \ifcase\month\or January\or February\or March\or April\or May\or June\or
  July\or August\or September\or October\or November\or
  December\fi\space\number\year
}

% Colour Palette ——————————————————————————————————————
\definecolor{merah}{HTML}{F4564E}
\definecolor{merahtua}{HTML}{89313E}
\definecolor{biru}{HTML}{60BBE5}
\definecolor{birutua}{HTML}{412F66}
\definecolor{hijau}{HTML}{59CC78}
\definecolor{hijautua}{HTML}{366D5B}
\definecolor{kuning}{HTML}{FFD56B}
\definecolor{jingga}{HTML}{FBA15F}
\definecolor{ungu}{HTML}{8C5FBF}
\definecolor{lavender}{HTML}{CBA5E8}
\definecolor{merjamb}{HTML}{FFB6E0}
\definecolor{mygray}{HTML}{E6E6E6}

% Theorems ————————————————————————————————————————————
\usepackage{tcolorbox}
\tcbuselibrary{skins,breakable,theorems}
\usepackage{changepage}

\newcounter{hitung}
\setcounter{hitung}{\thesection}

\makeatletter
	% Proof 证明如下
	\def\tcb@theo@widetitle#1#2#3{\hbox to \textwidth{\textsc{\large#1}\normalsize\space#3\hfil(#2)}}
	\tcbset{
		theorem style/theorem wide name and number/.code={ \let\tcb@theo@title=\tcb@theo@widetitle},
		proofbox/.style={skin=enhancedmiddle,breakable,parbox=false,boxrule=0mm,
			check odd page, toggle left and right, colframe=black!20!white!92!hijau,
			leftrule=8pt, rightrule=0mm, boxsep=0mm,arc=0mm, outer arc=0mm,
			left=3mm,right=3mm,top=0mm,bottom=0mm, toptitle=0mm,
			bottomtitle=0mm,colback=gray!3!white!98!biru, before skip=8pt, after skip=8pt,
			before={\par\vskip-2pt},after={\par\smallbreak},
		},
	}
	\newtcolorbox{ProofBox}{proofbox}
	\makeatother
	
	\let\realproof\proof
	\let\realendproof\endproof
	\renewenvironment{proof}[1][Prove:]{\ProofBox\strut\textsc{#1}\space}{\endProofBox}
        \AtEndEnvironment{proof}{\null\hfill$\blacksquare$}
        % Definition 定义环境
	\newtcbtheorem[use counter=hitung, number within=section]{dfn}{定义}
	{theorem style=theorem wide name and number,breakable,enhanced,arc=3.5mm,outer arc=3.5mm,
		boxrule=0pt,toprule=1pt,leftrule=0pt,bottomrule=1pt, rightrule=0pt,left=0.2cm,right=0.2cm,
		titlerule=0.5em,toptitle=0.1cm,bottomtitle=-0.1cm,top=0.2cm,
		colframe=white!10!biru,
		colback=white!90!biru,
		coltitle=white,
		shadow={1.3mm}{-1.3mm}{0mm}{gray!50!white}, % 添加阴影
        coltext=birutua!60!gray, title style={white!10!biru}, rbefoe skip=8pt, after skip=8pt,
		fonttitle=\bfseries,fontupper=\normalsize}{dfn}

	% 答题卡
	\newtcbtheorem[use counter=hitung, number within=section]{ans}{解答}
	{theorem style=theorem wide name and number,breakable,enhanced,arc=3.5mm,outer arc=3.5mm,
		boxrule=0pt,toprule=1pt,leftrule=0pt,bottomrule=1pt, rightrule=0pt,left=0.2cm,right=0.2cm,
		titlerule=0.5em,toptitle=0.1cm,bottomtitle=-0.1cm,top=0.2cm,
		colframe=white!10!biru,
		colback=white!90!biru,
		coltitle=white,
		shadow={1.3mm}{-1.3mm}{0mm}{gray!50!white}, % 添加阴影
        coltext=birutua!60!gray, title style={white!10!biru}, before skip=8pt, after skip=8pt,
		fonttitle=\bfseries,fontupper=\normalsize}{ans}

	% Axiom
	\newtcbtheorem[use counter=hitung, number within=section]{axm}{公理}
	{theorem style=theorem wide name and number,breakable,enhanced,arc=3.5mm,outer arc=3.5mm,
		boxrule=0pt,toprule=1pt,leftrule=0pt,bottomrule=1pt, rightrule=0pt,left=0.2cm,right=0.2cm,
		titlerule=0.5em,toptitle=0.1cm,bottomtitle=-0.1cm,top=0.2cm,
		colframe=white!10!biru,colback=white!90!biru,coltitle=white,
		shadow={1.3mm}{-1.3mm}{0mm}{gray!50!white!90}, % 添加阴影
        coltext=birutua!60!gray,title style={white!10!biru},before skip=8pt, after skip=8pt,
		fonttitle=\bfseries,fontupper=\normalsize}{axm}
 
	% Theorem
	\newtcbtheorem[use counter=hitung, number within=section]{thm}{定理}
	{theorem style=theorem wide name and number,breakable,enhanced,arc=3.5mm,outer arc=3.5mm,
		boxrule=0pt,toprule=1pt,leftrule=0pt,bottomrule=1pt, rightrule=0pt,left=0.2cm,right=0.2cm,
		titlerule=0.5em,toptitle=0.1cm,bottomtitle=-0.1cm,top=0.2cm,
		colframe=white!10!merah,colback=white!75!pink,coltitle=white, coltext=merahtua!80!merah,
		shadow={1.3mm}{-1.3mm}{0mm}{gray!50!white!90}, % 添加阴影
		title style={white!10!merah}, before skip=8pt, after skip=8pt,
		fonttitle=\bfseries,fontupper=\normalsize}{thm}
	
	% Proposition
	\newtcbtheorem[use counter=hitung, number within=section]{prp}{命题}
	{theorem style=theorem wide name and number,breakable,enhanced,arc=3.5mm,outer arc=3.5mm,
		boxrule=0pt,toprule=1pt,leftrule=0pt,bottomrule=1pt, rightrule=0pt,left=0.2cm,right=0.2cm,
		titlerule=0.5em,toptitle=0.1cm,bottomtitle=-0.1cm,top=0.2cm,
		colframe=white!10!hijau,colback=white!90!hijau,coltitle=white, coltext=hijautua!80!brown,
		shadow={1.3mm}{-1.3mm}{0mm}{gray!50!white}, % 添加阴影
		title style={white!10!hijau}, before skip=8pt, after skip=8pt,
		fonttitle=\bfseries,fontupper=\normalsize}{prp}


	% Example
	\newtcolorbox[use counter=hitung, number within=section]{cth}[1][]{breakable,
		colframe=white!10!jingga, coltitle=white!90!jingga, colback=white!85!jingga, coltext=black!10!brown!50!jingga, colbacktitle=white!10!jingga, enhanced, fonttitle=\bfseries,fontupper=\normalsize, attach boxed title to top left={yshift=-2mm}, before skip=8pt, after skip=8pt,
		title=Contoh~\thetcbcounter \ \ #1}

	% Catatan/Note
	\newtcolorbox{ctt}[1][]{enhanced, 
		left=4.1mm, borderline west={8pt}{0pt}{white!10!kuning}, 
		before skip=6pt, after skip=6pt, 
		colback=white!85!kuning, colframe= white!85!kuning, coltitle=orange!60!kuning!25!brown, coltext=orange!60!kuning!25!brown,
		fonttitle=\bfseries,fontupper=\normalsize, before skip=8pt, after skip=8pt,
		title=\underline{Catatan}  #1}
	
	% Komentar/Remark
	\newtcolorbox{rmr}[1][]{
		,arc=0mm,outer arc=0mm,
		boxrule=0pt,toprule=1pt,leftrule=0pt,bottomrule=5pt, rightrule=0pt,left=0.2cm,right=0.2cm,
		titlerule=0.5em,toptitle=0.1cm,bottomtitle=-0.1cm,top=0.2cm,
		colframe=white!10!kuning,colback=white!85!kuning,coltitle=white, coltext=orange!60!kuning,
		fonttitle=\bfseries,fontupper=\normalsize, before skip=8pt, after skip=8pt,
		title=Komentar  #1}

% ————————————————————————————————————————————————————————————————————————
\begin{document}

\maketitle
\vspace{-3.5em}
\garis{Kite}{Kite}

\tableofcontents

\section{第三章习题 4}

设 \( X, Y \) 都是非负的连续型随机变量,它们相互独立。
\begin{enumerate}
    \item 证明 \( P(X < Y) = \int_0^{\infty} F_X(x) f_Y(x) \, dx \),其中 \( F_X(x) \) 是 \( X \) 的分布函数,\( f_Y(y) \) 是 \( Y \) 的概率密度。
    \item 设 \( X, Y \) 相互独立,其概率密度分别为
    \[
    f_X(x) = 
    \begin{cases} 
        \lambda_1 e^{-\lambda_1 x}, & x > 0, \\
        0, & \text{其他},
    \end{cases}
    \quad
    f_Y(y) = 
    \begin{cases} 
        \lambda_2 e^{-\lambda_2 y}, & y > 0, \\
        0, & \text{其他}.
    \end{cases}
    \]
    求 \( P(X < Y) \)。
\end{enumerate}

\begin{ans}{4}{4}
    (1) 因为 \( X \) 和 \( Y \) 是相互独立的非负连续型随机变量,且 \( F_X(x) \) 是 \( X \) 的分布函数,\( f_Y(y) \) 是 \( Y \) 的概率密度函数,因此:
    \[
    P(X < Y) = \iint_{x < y} f_X(x) f_Y(y) \, dx \, dy
    \]
    
    通过改变积分顺序,我们得到:
    \[
    P(X < Y) = \int_0^{\infty} \int_0^y f_X(x) \, dx \, f_Y(y) \, dy = \int_0^{\infty} F_X(x) f_Y(x) \, dx
    \]
    
    (2) 已知 \( f_X(x) = \lambda_1 e^{-\lambda_1 x} \) 和 \( f_Y(y) = \lambda_2 e^{-\lambda_2 y} \),根据 (1) 的结果,有:
    \[
    P(X < Y) = \int_0^{\infty} F_X(x) f_Y(y) \, dx
    \]
    
    将 \( F_X(x) = 1 - e^{-\lambda_1 x} \) 和 \( f_Y(y) = \lambda_2 e^{-\lambda_2 y} \) 代入,得到:
    \[
    P(X < Y) = \int_0^{\infty} (1 - e^{-\lambda_1 x}) \lambda_2 e^{-\lambda_2 x} \, dx
    \]
    
    经过积分计算,结果为:
    \[
    P(X < Y) = \frac{\lambda_1}{\lambda_1 + \lambda_2}
    \]
\end{ans}

\section{第三章习题 6}
将一枚硬币掷 3 次,以 \( X \) 表示前 2 次中出现正面(H)的次数,以 \( Y \) 表示 3 次中出现正面的次数。求 \( X, Y \) 的联合分布律以及 \( (X, Y) \) 的边缘分布律。

\begin{ans}{6}{6}
    样本空间如下,共有 8 种情况:
    \[
    \{ \text{HHH, HHT, HTH, THH, HTT, THT, TTH, TTT} \}
    \]
    
    构建联合概率表如下:
    
    \[
    \begin{array}{|c|c|c|c|c|c|}
    \hline
    X \backslash Y & 0 & 1 & 2 & 3 & \text{边缘分布 } P(Y = j) \\
    \hline
    0 & \frac{1}{8} & 0 & \frac{1}{8} & 0 & \frac{1}{4} \\
    1 & 0 & \frac{1}{8} & \frac{2}{8} & 0 & \frac{3}{8} \\
    2 & 0 & 0 & \frac{1}{8} & \frac{1}{8} & \frac{1}{4} \\
    \hline
    \text{边缘分布 } P(X = i) & \frac{1}{4} & \frac{3}{8} & \frac{1}{4} & 0 & 1 \\
    \hline
    \end{array}
    \]
\end{ans}

\section{第三章习题 7}

设二维随机变量 \( (X, Y) \) 的概率密度为
\[
f(x, y) = 
\begin{cases} 
4.8y(2 - x), & 0 \leq x \leq 1, 0 \leq y \leq x, \\
0, & \text{其他}.
\end{cases}
\]
求边缘概率密度。

\begin{ans}{7}{7}
    由于 \( (X, Y) \) 的概率密度 \( f(x, y) \) 在区域 \( G = \{ (x, y) | 0 \leq x \leq 1, 0 \leq y \leq x \} \) 外取零,因此我们可以计算边缘密度 \( f_X(x) \) 和 \( f_Y(y) \)。

    \[
    f_X(x) = \int_{-\infty}^{\infty} f(x, y) \, dy = \int_0^x 4.8y(2 - x) \, dy
    \]
    \[
    = 4.8 \int_0^x y(2 - x) \, dy = 2.4 (2 - x) x^2, \quad 0 \leq x \leq 1.
    \]
    
    同理,
    \[
    f_Y(y) = \int_{-\infty}^{\infty} f(x, y) \, dx = \int_y^1 4.8y(2 - x) \, dx
    \]
    \[
    = 4.8y \int_y^1 (2 - x) \, dx = 2.4y(3 - 4y + y^2), \quad 0 \leq y \leq 1.
    \]
    
    因此,边缘密度函数为
    \[
    f_X(x) = 
    \begin{cases} 
    2.4 (2 - x) x^2, & 0 \leq x \leq 1, \\
    0, & \text{其他}.
    \end{cases}
    \]
    和
    \[
    f_Y(y) = 
    \begin{cases} 
    2.4y(3 - 4y + y^2), & 0 \leq y \leq 1, \\
    0, & \text{其他}.
    \end{cases}
    \]
    
\end{ans}

\section{第三章习题 9}

设二维随机变量 \( (X, Y) \) 的概率密度为
\[
f(x, y) = 
\begin{cases} 
cx^2 y, & x^2 \leq y \leq 1, \\
0, & \text{其他}.
\end{cases}
\]

\begin{enumerate}
    \item 确定常数 \( c \)。
    \item 求边缘概率密度。
\end{enumerate}

\begin{ans}{9}{9}
    (1) 确定常数 \( c \)

    \[
    1 = \iint f(x, y) \, dx \, dy = \int_0^1 \int_{x^2}^1 cx^2 y \, dy \, dx
    \]
    
    计算内层积分:
    \[
    = \int_0^1 c x^2 \int_{x^2}^1 y \, dy \, dx = \int_0^1 c x^2 \left[ \frac{y^2}{2} \right]_{x^2}^1 \, dx
    \]
    \[
    = \int_0^1 c x^2 \left( \frac{1}{2} - \frac{x^4}{2} \right) \, dx = c \int_0^1 \frac{x^2}{2} (1 - x^4) \, dx
    \]
    \[
    = c \int_0^1 \frac{x^2}{2} - \frac{x^6}{2} \, dx = c \left[ \frac{x^3}{6} - \frac{x^7}{14} \right]_0^1 = c \left( \frac{1}{6} - \frac{1}{14} \right)
    \]
    \[
    = c \cdot \frac{4}{21} = 1 \Rightarrow c = \frac{21}{4}
    \]
    
    (2) 求边缘概率密度
    
    \[
    f_X(x) = \int_{-\infty}^{\infty} f(x, y) \, dy = \int_{x^2}^1 \frac{21}{4} x^2 y \, dy
    \]
    \[
    = \frac{21}{4} x^2 \int_{x^2}^1 y \, dy = \frac{21}{4} x^2 \left[ \frac{y^2}{2} \right]_{x^2}^1
    \]
    \[
    = \frac{21}{4} x^2 \cdot \frac{1 - x^4}{2} = \frac{21}{8} x^2 (1 - x^4), \quad -1 \leq x \leq 1.
    \]
    
    同理,
    \[
    f_Y(y) = \int_{-\infty}^{\infty} f(x, y) \, dx = \int_{-\sqrt{y}}^{\sqrt{y}} \frac{21}{4} x^2 y \, dx
    \]
    \[
    = \frac{21}{4} y \int_{-\sqrt{y}}^{\sqrt{y}} x^2 \, dx = \frac{21}{4} y \cdot \frac{2y^{3/2}}{3} = \frac{7}{4} y^{5/2}, \quad 0 \leq y \leq 1.
    \]
    
    因此,边缘密度函数为
    \[
    f_X(x) = 
    \begin{cases} 
    \frac{21}{8} x^2 (1 - x^4), & -1 \leq x \leq 1, \\
    0, & \text{其他}.
    \end{cases}
    \]
    和
    \[
    f_Y(y) = 
    \begin{cases} 
    \frac{7}{4} y^{5/2}, & 0 \leq y \leq 1, \\
    0, & \text{其他}.
    \end{cases}
    \]
\end{ans}

\section{第三章习题 17}

(1) 设随机变量 \( (X, Y) \) 具有分布函数
\[
F(x, y) = 
\begin{cases} 
(1 - e^{-ax})y, & x \geq 0, 0 \leq y \leq 1, a > 0, \\
1 - e^{-ax}, & x \geq 0, y > 1, \\
0, & \text{其他}.
\end{cases}
\]
证明 \( X, Y \) 相互独立。

(2) 设随机变量 \( (X, Y) \) 具有分布律
\[
P(X = x, Y = y) = p^2 (1 - p)^{x + y - 2}, \quad 0 < p < 1, x, y \text{均为正整数},
\]
问 \( X, Y \) 是否相互独立?


\begin{ans}{17}{17}
    (1) \( F_X(x) = F(x, \infty) \) 和 \( F_Y(y) = F(\infty, y) \) 为
    \[
    F_X(x) = 
    \begin{cases}
    1 - e^{-ax}, & x \geq 0, \\
    0, & \text{其他}.
    \end{cases}
    \]
    \[
    F_Y(y) = 
    \begin{cases}
    y, & 0 \leq y \leq 1, \\
    1, & y > 1, \\
    0, & \text{其他}.
    \end{cases}
    \]
    
    因为对于所有 \( x, y \),都有 \( F(x, y) = F_X(x) F_Y(y) \),故 \( X, Y \) 相互独立。
    
    (2) \( P(X = x) \) 的计算如下:
    \[
    P(X = x) = \sum_{j=1}^{\infty} P(X = x, Y = j) = p^2 (1 - p)^{x - 1} \sum_{j=1}^{\infty} (1 - p)^{j - 1}
    \]
    \[
    = p^2 (1 - p)^{x - 1} \cdot \frac{1}{1 - (1 - p)} = p(1 - p)^{x - 1}, \quad x = 1, 2, \dots, \text{其中 } 0 < p < 1.
    \]
    
    同理,
    \[
    P(Y = y) = p(1 - p)^{y - 1}, \quad y = 1, 2, \dots, \text{其中 } 0 < p < 1.
    \]
    
    因为对于所有正整数 \( x, y \) 都有
    \[
    P(X = x, Y = y) = P(X = x) P(Y = y),
    \]
    故 \( X, Y \) 相互独立。

\end{ans}

\section{第三章习题 18}

设 \( X \) 和 \( Y \) 是两个相互独立的随机变量,\( X \) 在区间 \( (0, 1) \) 上服从均匀分布,\( Y \) 的概率密度为
\[
f_Y(y) = 
\begin{cases} 
\frac{1}{2} e^{-y/2}, & y > 0, \\
0, & y \leq 0.
\end{cases}
\]
\begin{enumerate}
    \item 求 \( X \) 和 \( Y \) 的联合概率密度。
    \item 设含有 a 的二次方程为 \( a^2 + 2Xa + Y = 0 \),试求 \( a \) 有实根的概率。
\end{enumerate}

\begin{ans}{18}{18}
    因为 \( X \) 和 \( Y \) 相互独立,故 \( (X, Y) \) 的联合概率密度为
    \[
    f(x, y) = f_X(x) f_Y(y) = 
    \begin{cases}
    \frac{1}{2} e^{-y/2}, & 0 < x < 1, y > 0, \\
    0, & \text{其他}.
    \end{cases}
    \]
    
    (2) 方程 \( a^2 + 2Xa + Y = 0 \) 有实根的充要条件为其判别式 \( 4X^2 - 4Y \geq 0 \),即 \( X^2 \geq Y \)。所以
    \[
    P(X^2 \geq Y) = \iint_G f(x, y) \, dx \, dy = \int_0^1 \int_0^{x^2} \frac{1}{2} e^{-y/2} \, dy \, dx
    \]
    \[
    = \int_0^1 \left[ -e^{-y/2} \right]_0^{x^2} \, dx = \int_0^1 \left( 1 - e^{-x^2/2} \right) \, dx
    \]
    
    令 \( u = x^2/2 \),则积分结果为:
    \[
    P(X^2 \geq Y) = 1 - \sqrt{\frac{\pi}{2}} \approx 0.1445.
    \]
\end{ans}

\section{第三章习题 19}

进行打靶,设弹着点 \( A(X, Y) \) 的坐标 \( X \) 和 \( Y \) 相互独立,且都服从 \( N(0, 1) \) 分布,规定
\begin{itemize}
    \item 点 \( A \) 落在区域 \( D_1 = \{(x, y) | x^2 + y^2 \leq 1\} \) 得 2 分;
    \item 点 \( A \) 落在 \( D_2 = \{(x, y) | 1 < x^2 + y^2 \leq 4\} \) 得 1 分;
    \item 点 \( A \) 落在 \( D_3 = \{(x, y) | x^2 + y^2 > 4\} \) 得 0 分。
\end{itemize}
以 \( Z \) 记打靶的得分,写出 \( X, Y \) 的联合概率密度,并求 \( Z \) 的分布律。

\begin{ans}{19}{19}
    由题意知 \( X, Y \) 的概率密度为
    \[
    f_X(x) = \frac{1}{\sqrt{2\pi}} e^{-x^2/2}, \quad -\infty < x < \infty,
    \]
    \[
    f_Y(y) = \frac{1}{\sqrt{2\pi}} e^{-y^2/2}, \quad -\infty < y < \infty.
    \]
    
    且 \( X \) 和 \( Y \) 相互独立,故 \( X \) 和 \( Y \) 的联合概率密度为
    \[
    f(x, y) = f_X(x) f_Y(y) = \frac{1}{2\pi} e^{-(x^2 + y^2)/2}, \quad -\infty < x < \infty, -\infty < y < \infty.
    \]
    
    计算得分分布:
    
    \[
    P(Z = 2) = P((X, Y) \in D_1) = \iint_{D_1} f(x, y) \, dx \, dy = 1 - e^{-1/2}.
    \]
    
    \[
    P(Z = 1) = P((X, Y) \in D_2) = \iint_{D_2} f(x, y) \, dx \, dy = e^{-1/2} - e^{-2}.
    \]
    
    \[
    P(Z = 0) = P((X, Y) \in D_3) = 1 - (1 - e^{-1/2}) - (e^{-1/2} - e^{-2}) = e^{-2}.
    \]
    
    因此,\( Z \) 的分布律为
    
    \[
    \begin{array}{c|c|c|c}
    Z & 0 & 1 & 2 \\
    \hline
    P(Z = z) & e^{-2} & e^{-1/2} - e^{-2} & 1 - e^{-1/2} \\
    \end{array}
    \]
\end{ans}

\end{document}
