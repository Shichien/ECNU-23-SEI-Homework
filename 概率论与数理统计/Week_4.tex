\documentclass[twoside]{article}
\usepackage[paperwidth=210mm, paperheight=297mm, margin=2cm]{geometry}
\usepackage[utf8]{inputenc}
\usepackage{ctex}

% Formatting Packages ——————————————————————————————————————
\usepackage{multicol}
\usepackage{multirow}
\usepackage{enumitem}
\usepackage{indentfirst}
\usepackage[toc]{multitoc}

% Math & Physics Packages ————————————————————————————
\usepackage{amsmath, amsthm, amsfonts, amssymb}
\usepackage{amssymb}
\usepackage{setspace}
\usepackage{physics}
\usepackage{cancel}
\usepackage{nicefrac}

% Image-related Packages —————————————————————————————
\usepackage{graphics, graphicx}
\usepackage{tikz}
\usetikzlibrary{arrows.meta}
\usepackage{pgfplots}
\pgfplotsset{compat=1.18}
\usepackage{xcolor}
\usepackage{fourier-orns}
\usepackage{lipsum}


%% Silakan otak-atik judul di sini
\title{\texorpdfstring{\vspace{-1.5em}}{}\textbf{概率论与数理统计}}
\author{张梓卫 10235101526}
\date{\the\day \ \monthyear}

% Kalau mau bab pertama nomornya 0, ganti 0 jadi -1
\setcounter{section}{0}

\renewcommand*\contentsname{第十三周概率论作业}
\renewcommand*{\multicolumntoc}{2}
\setlength{\columnseprule}{0.5pt}

% Layouting Packages ————————————————————————————————————————
\usepackage{titlesec}
\usepackage{fancyhdr}
\pagestyle{fancy}
\setlength{\headheight}{14.39996pt}
\fancyfoot[C]{\textit{By Deralive (10235101526)}}
\fancyfoot[R]{\thepage}

\fancyhead[LE, RO]{\textsl{\rightmark}}
\fancyhead[LO, RE]{\textsl{\leftmark}}

\renewcommand\headrule{
	\vspace{-6pt}
	\hrulefill
	\raisebox{-2.1pt}
	{\quad\floweroneleft\decoone\floweroneright\quad}
	\hrulefill}

\renewcommand\footrule{
	\hrulefill
	\raisebox{-2.1pt}
	{\quad\floweroneleft\decoone\floweroneright\quad}
	\hrulefill}
  
\newcommand{\fpb}[2]{\mathrm{FPB}(#1, #2)}
\newcommand{\kpk}[2]{\mathrm{KPK}(#1, #2)}

% Reference and Bibliography Packages ————————————————————
\usepackage{hyperref}
\hypersetup{
    colorlinks=true,
    linkcolor={black},
    citecolor={biru!70!black},
    urlcolor={biru!80!black}
}

\numberwithin{equation}{section}

% Silakan lihat dokumentasi package biblatex
% untuk format sitasi yang diperlukan
\usepackage[backend=biber]{biblatex}
\addbibresource{ref.bib}
\makeatletter
\def\@biblabel#1{}
\makeatother

\renewcommand{\mod}{\mathrm{mod} \ }

% Titling
\newcommand{\garis} [3] []{
	\begin{center}
		\begin{tikzpicture}
			\draw[#2-#3, ultra thick, #1] (0,0) to (1\linewidth,0);
		\end{tikzpicture}
	\end{center}
}

\newcommand{\monthyear}{
  \ifcase\month\or January\or February\or March\or April\or May\or June\or
  July\or August\or September\or October\or November\or
  December\fi\space\number\year
}

% Colour Palette ——————————————————————————————————————
\definecolor{merah}{HTML}{F4564E}
\definecolor{merahtua}{HTML}{89313E}
\definecolor{biru}{HTML}{60BBE5}
\definecolor{birutua}{HTML}{412F66}
\definecolor{hijau}{HTML}{59CC78}
\definecolor{hijautua}{HTML}{366D5B}
\definecolor{kuning}{HTML}{FFD56B}
\definecolor{jingga}{HTML}{FBA15F}
\definecolor{ungu}{HTML}{8C5FBF}
\definecolor{lavender}{HTML}{CBA5E8}
\definecolor{merjamb}{HTML}{FFB6E0}
\definecolor{mygray}{HTML}{E6E6E6}

% Theorems ————————————————————————————————————————————
\usepackage{tcolorbox}
\tcbuselibrary{skins,breakable,theorems}
\usepackage{changepage}

\newcounter{hitung}
\setcounter{hitung}{\thesection}

\makeatletter
	% Proof 证明如下
	\def\tcb@theo@widetitle#1#2#3{\hbox to \textwidth{\textsc{\large#1}\normalsize\space#3\hfil(#2)}}
	\tcbset{
		theorem style/theorem wide name and number/.code={ \let\tcb@theo@title=\tcb@theo@widetitle},
		proofbox/.style={skin=enhancedmiddle,breakable,parbox=false,boxrule=0mm,
			check odd page, toggle left and right, colframe=black!20!white!92!hijau,
			leftrule=8pt, rightrule=0mm, boxsep=0mm,arc=0mm, outer arc=0mm,
			left=3mm,right=3mm,top=0mm,bottom=0mm, toptitle=0mm,
			bottomtitle=0mm,colback=gray!3!white!98!biru, before skip=8pt, after skip=8pt,
			before={\par\vskip-2pt},after={\par\smallbreak},
		},
	}
	\newtcolorbox{ProofBox}{proofbox}
	\makeatother
	
	\let\realproof\proof
	\let\realendproof\endproof
	\renewenvironment{proof}[1][Prove:]{\ProofBox\strut\textsc{#1}\space}{\endProofBox}
        \AtEndEnvironment{proof}{\null\hfill$\blacksquare$}
        % Definition 定义环境
	\newtcbtheorem[use counter=hitung, number within=section]{dfn}{定义}
	{theorem style=theorem wide name and number,breakable,enhanced,arc=3.5mm,outer arc=3.5mm,
		boxrule=0pt,toprule=1pt,leftrule=0pt,bottomrule=1pt, rightrule=0pt,left=0.2cm,right=0.2cm,
		titlerule=0.5em,toptitle=0.1cm,bottomtitle=-0.1cm,top=0.2cm,
		colframe=white!10!biru,
		colback=white!90!biru,
		coltitle=white,
		shadow={1.3mm}{-1.3mm}{0mm}{gray!50!white}, % 添加阴影
        coltext=birutua!60!gray, title style={white!10!biru}, rbefoe skip=8pt, after skip=8pt,
		fonttitle=\bfseries,fontupper=\normalsize}{dfn}

	% 答题卡
	\newtcbtheorem[use counter=hitung, number within=section]{ans}{解答}
	{theorem style=theorem wide name and number,breakable,enhanced,arc=3.5mm,outer arc=3.5mm,
		boxrule=0pt,toprule=1pt,leftrule=0pt,bottomrule=1pt, rightrule=0pt,left=0.2cm,right=0.2cm,
		titlerule=0.5em,toptitle=0.1cm,bottomtitle=-0.1cm,top=0.2cm,
		colframe=white!10!biru,
		colback=white!90!biru,
		coltitle=white,
		shadow={1.3mm}{-1.3mm}{0mm}{gray!50!white}, % 添加阴影
        coltext=birutua!60!gray, title style={white!10!biru}, before skip=8pt, after skip=8pt,
		fonttitle=\bfseries,fontupper=\normalsize}{ans}

	% Axiom
	\newtcbtheorem[use counter=hitung, number within=section]{axm}{公理}
	{theorem style=theorem wide name and number,breakable,enhanced,arc=3.5mm,outer arc=3.5mm,
		boxrule=0pt,toprule=1pt,leftrule=0pt,bottomrule=1pt, rightrule=0pt,left=0.2cm,right=0.2cm,
		titlerule=0.5em,toptitle=0.1cm,bottomtitle=-0.1cm,top=0.2cm,
		colframe=white!10!biru,colback=white!90!biru,coltitle=white,
		shadow={1.3mm}{-1.3mm}{0mm}{gray!50!white!90}, % 添加阴影
        coltext=birutua!60!gray,title style={white!10!biru},before skip=8pt, after skip=8pt,
		fonttitle=\bfseries,fontupper=\normalsize}{axm}
 
	% Theorem
	\newtcbtheorem[use counter=hitung, number within=section]{thm}{定理}
	{theorem style=theorem wide name and number,breakable,enhanced,arc=3.5mm,outer arc=3.5mm,
		boxrule=0pt,toprule=1pt,leftrule=0pt,bottomrule=1pt, rightrule=0pt,left=0.2cm,right=0.2cm,
		titlerule=0.5em,toptitle=0.1cm,bottomtitle=-0.1cm,top=0.2cm,
		colframe=white!10!merah,colback=white!75!pink,coltitle=white, coltext=merahtua!80!merah,
		shadow={1.3mm}{-1.3mm}{0mm}{gray!50!white!90}, % 添加阴影
		title style={white!10!merah}, before skip=8pt, after skip=8pt,
		fonttitle=\bfseries,fontupper=\normalsize}{thm}
	
	% Proposition
	\newtcbtheorem[use counter=hitung, number within=section]{prp}{命题}
	{theorem style=theorem wide name and number,breakable,enhanced,arc=3.5mm,outer arc=3.5mm,
		boxrule=0pt,toprule=1pt,leftrule=0pt,bottomrule=1pt, rightrule=0pt,left=0.2cm,right=0.2cm,
		titlerule=0.5em,toptitle=0.1cm,bottomtitle=-0.1cm,top=0.2cm,
		colframe=white!10!hijau,colback=white!90!hijau,coltitle=white, coltext=hijautua!80!brown,
		shadow={1.3mm}{-1.3mm}{0mm}{gray!50!white}, % 添加阴影
		title style={white!10!hijau}, before skip=8pt, after skip=8pt,
		fonttitle=\bfseries,fontupper=\normalsize}{prp}


	% Example
	\newtcolorbox[use counter=hitung, number within=section]{cth}[1][]{breakable,
		colframe=white!10!jingga, coltitle=white!90!jingga, colback=white!85!jingga, coltext=black!10!brown!50!jingga, colbacktitle=white!10!jingga, enhanced, fonttitle=\bfseries,fontupper=\normalsize, attach boxed title to top left={yshift=-2mm}, before skip=8pt, after skip=8pt,
		title=Contoh~\thetcbcounter \ \ #1}

	% Catatan/Note
	\newtcolorbox{ctt}[1][]{enhanced, 
		left=4.1mm, borderline west={8pt}{0pt}{white!10!kuning}, 
		before skip=6pt, after skip=6pt, 
		colback=white!85!kuning, colframe= white!85!kuning, coltitle=orange!60!kuning!25!brown, coltext=orange!60!kuning!25!brown,
		fonttitle=\bfseries,fontupper=\normalsize, before skip=8pt, after skip=8pt,
		title=\underline{Catatan}  #1}
	
	% Komentar/Remark
	\newtcolorbox{rmr}[1][]{
		,arc=0mm,outer arc=0mm,
		boxrule=0pt,toprule=1pt,leftrule=0pt,bottomrule=5pt, rightrule=0pt,left=0.2cm,right=0.2cm,
		titlerule=0.5em,toptitle=0.1cm,bottomtitle=-0.1cm,top=0.2cm,
		colframe=white!10!kuning,colback=white!85!kuning,coltitle=white, coltext=orange!60!kuning,
		fonttitle=\bfseries,fontupper=\normalsize, before skip=8pt, after skip=8pt,
		title=Komentar  #1}

% ————————————————————————————————————————————————————————————————————————
\begin{document}

\maketitle
\vspace{-3.5em}
\garis{Kite}{Kite}

\tableofcontents

\section{第一章习题37}

37.设第一只盒子中装有3只蓝色球、2只绿色球、2只白色球,第二只盒子中装有2只
蓝色球、3只绿色球、4只白色球,独立地分别在两只盒子中各取一只球.

\begin{itemize}
    \item (1)求至少有一只蓝色球的概率.
    \item (2)求有一只蓝色球、一只白色球的概率.
    \item (3)已知至少有一只蓝色球,求有一只蓝色球、一只白色球的概率
\end{itemize}

\subsection*{Problem 1:至少有一个蓝色球的概率}

\begin{ans}{37(1)}{37(1)}
    从第一个盒子中取到蓝球的概率为
    \[
    P(A_1) = \frac{3}{7}.
    \]
    从第二个盒子中取到蓝球的概率为
    \[
    P(A_2) = \frac{2}{9}.
    \]
    两只盒子都没有蓝色球的概率为
    \[
    P(\overline{A_1} \cap \overline{A_2}) = \frac{4}{7} \times \frac{7}{9} = \frac{4}{9}.
    \]
    则至少有一个蓝色球的概率为
    \[
    P(\text{至少有一个蓝色球}) = 1 - \frac{4}{9} = \frac{5}{9}.
    \]
\end{ans}
    
\subsection*{Problem 2: 有一只蓝色球和一只白色球的概率}
    
\begin{ans}{37(2)}{37(2)}
    此处求的是一只蓝色球和一只白色球的概率,可以有以下两种情况:
    - 第一只盒子取出蓝色球,第二只盒子取出白色球;
    - 第一只盒子取出白色球,第二只盒子取出蓝色球。
    
    根据独立事件的乘法规则:
    
    \[
    P(\text{蓝色,白色}) = P(\text{第一盒蓝色,第二盒白色}) + P(\text{第一盒白色,第二盒蓝色})
    \]
    
    \[
    P(\text{第一盒蓝色,第二盒白色}) = P(A) \times \frac{4}{9} = \frac{3}{7} \times \frac{4}{9} = \frac{12}{63} = \frac{4}{21}
    \]
    
    \[
    P(\text{第一盒白色,第二盒蓝色}) = \frac{2}{7} \times P(B) = \frac{2}{7} \times \frac{2}{9} = \frac{4}{63}
    \]
    
    故一只蓝色球和一只白色球的概率为:
    
    \[
    P(\text{蓝色,白色}) = \frac{4}{21} + \frac{4}{63} = \frac{12}{63} + \frac{4}{63} = \frac{16}{63}
    \]
    
\end{ans}
    
\subsection*{Problem 3: 已知至少有一只蓝色球,求有一只蓝色球和一只白色球的概率}
    
\begin{ans}{37(3)}{37(3)}
    至少有一个蓝色球的概率是:
    \[
    P(\text{至少一个蓝色球}) = \frac{5}{9}.
    \]
    根据条件概率公式有:
    \[
    P(\text{一个蓝色球和一个白色球} \mid \text{至少一个蓝色球}) = \frac{\frac{16}{63}}{\frac{5}{9}} = \frac{16}{35}.
    \]
\end{ans}


\section{第一章习题40}

将 $A, B, C$ 三个字母之一输入信道,输出为原字母的概率是 $\alpha$,而输出为其他某一字母的概率是 $\frac{1-\alpha}{2}$。输入字母串 $AAAA, BBBB, CCCC$ 之一,输入信道,输入 $AAAA, BBBB, CCCC$ 的概率分别为 $p_1, p_2, p_3$,且 $p_1 + p_2 + p_3 = 1$。

已知输出为 $ABCA$,求输入的是 $AAAA$ 的概率。

\begin{ans}{40}{40}

设事件 $X = AAAA$ 表示输入的是 $AAAA$,事件 $Y = ABCA$ 表示输出为 $ABCA$。要求的是 $P(X|Y)$,根据贝叶斯公式:

\[
P(X|Y) = \frac{P(Y|X)P(X)}{P(Y)}
\]

首先计算条件概率 $P(Y|X)$,即在输入为 $AAAA$ 的条件下,输出为 $ABCA$ 的概率。对每个字母的传输过程:
- 第一个字母 $A$ 被正确传输的概率为 $\alpha$;
- 第二个字母 $A$ 被错误传输为 $B$ 的概率为 $\frac{1-\alpha}{2}$;
- 第三个字母 $A$ 被错误传输为 $C$ 的概率为 $\frac{1-\alpha}{2}$;
- 第四个字母 $A$ 被正确传输的概率为 $\alpha$。

因此:
\[
P(Y|X) = \alpha \times \frac{1-\alpha}{2} \times \frac{1-\alpha}{2} \times \alpha = \alpha^2 \times \left(\frac{1-\alpha}{2}\right)^2
\]

接着,$P(Y)$ 表示输出为 $ABCA$ 的总概率,可以用全概率公式计算:

\[
P(Y) = P(Y|X)P(X) + P(Y|BBBB)P(BBBB) + P(Y|CCCC)P(CCCC)
\]

对于输入为 $BBBB$ 和 $CCCC$ 的情况,类似地可以计算 $P(Y|BBBB)$ 和 $P(Y|CCCC)$。例如,对于输入为 $BBBB$:
\[
P(Y|BBBB) = \frac{1-\alpha}{2} \times \alpha \times \frac{1-\alpha}{2} \times \frac{1-\alpha}{2}
\]
同样,对于输入为 $CCCC$:
\[
P(Y|CCCC) = \frac{1-\alpha}{2} \times \frac{1-\alpha}{2} \times \alpha \times \frac{1-\alpha}{2}
\]

因此,总概率 $P(Y)$ 为:
\[
P(Y) = \alpha^2 \times \left(\frac{1-\alpha}{2}\right)^2 \times p_1 + \frac{1-\alpha}{2} \times \alpha \times \left(\frac{1-\alpha}{2}\right)^2 \times p_2 + \left(\frac{1-\alpha}{2}\right)^3 \times \alpha \times p_3
\]

代入贝叶斯公式,得到:
\[
P(X|Y) = \frac{\alpha^2 \times \left(\frac{1-\alpha}{2}\right)^2 \times p_1}{P(Y)}
\]

\end{ans}

\section{补充习题 1}

甲乙两人轮流掷一颗骰子,甲先掷。每当某人掷出 1 点时,交给对方掷,否则此人继续掷。求第 $n$ 次仍然由甲方掷骰子的概率。

\subsection*{解答}

\begin{ans}{补充习题 1}{补充习题 1}
    
    甲(乙)掷到1点的概率为:
    \[
    \frac{1}{6}
    \]
    甲(乙)未掷到1点的概率为:
    \[
    \frac{5}{6}
    \]
    设第 \(k\) 次由甲掷的概率为 \(p_k\),则乙掷的概率为 \(1 - p_k\)。
    
    第一次由甲掷,故第二次由甲掷的概率为:
    \[
    p_2 = \frac{1}{6}
    \]
    于是,第 \(k+1\) 次由甲掷的概率为:
    \[
    p_{k+1} = \frac{1}{6} + \frac{5}{6} (1 - p_k) = \frac{5}{6} - \frac{2}{3} p_k
    \]
    即:
    \[
    p_{k+1} = \frac{5}{6} - \frac{2}{3} \left(p_k\right)
    \]
    因为:
    \[
    p_{2} = \frac{1}{6},p_{k+1} - \frac{1}{2} = - \frac{2}{3} \left( p_k - \frac{1}{2} \right)
    \]

    求得:
    \[
    p_n = \frac{1}{2} - \frac{1}{3} \left( -\frac{2}{3} \right)^{n-2}
    \]

\end{ans}

\end{document}