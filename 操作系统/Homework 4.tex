\documentclass[UTF8]{homework}
% \date{May 14, 2024}
% \author{Deralive / Shichien}
% \title{华东师范大学软件学院实验报告模板}
% 注意事项:编译两次,以确保目录、页码完整显示
% 模板文件:https://github.com/Shichien/ECNU-LateX-Template

\def\allfiles{}

% \documentclass[14pt,a4paper,UTF8,twoside]{article}

\usepackage{amsmath}
\usepackage{graphicx}
\usepackage{longtable}
% \usepackage{geometry} 
\usepackage{ctex}
\usepackage{booktabs} % 表格库
\usepackage{titlesec} % 标题库
\usepackage{fancyhdr} % 页眉页脚库
\usepackage{lastpage} % 页码数库
\usepackage{listings} % 代码块包
\usepackage{xcolor}
\usepackage[hidelinks]{hyperref}
\usepackage{tikz}
\usepackage{tikz-qtree}
\usepackage{fontspec} % 允许设置字体
\usepackage{unicode-math} % 允许数学公式使用特定字体
\usepackage{mwe}
\usepackage{zhlipsum} % 中文乱数文本
\usepackage{amsmath}
\usepackage{xcolor}
\usepackage{float} % 浮动体环境
\usepackage{subcaption} % 子图包
\usepackage{forest}
\usepackage{tikz}
\usepackage{pgfgantt}



\definecolor{mygreen}{rgb}{0,0.6,0}
\definecolor{mygray}{rgb}{0.5,0.5,0.5}
\definecolor{mymauve}{rgb}{0.58,0,0.82}

\date{} % 留空,以让编译时去除日期

%———————————————注意事项—————————————————%

% 1、如果编译显示失败,但没有错误信息,就是 filename.pdf 正在被占用
% 2、在文件夹中的终端使用 Windows > xelatex filename.tex 也可编译

%—————————————华东师范大学———————————————%

% 论文制作时须加页眉,页眉从中文摘要开始至论文末
% 偶数页码内容为:华东师范大学硕士学位论文,奇数页码内容为学位论文题目

%————————定义 \section 的标题样式————————%

% 注意:\chapter 等命令,内部使用的是 \thispagestyle{plain} 的排版格式
% 若需要自己加上页眉,实际是在用 \thispagestyle{fancy} 的排版格式
% 加上下面这一段指令,就能够让 \section 也使用 fancy 的排版格式
% 本质就是让目录、第一页也能够显示页眉、页脚

\fancypagestyle{plain}{
  \pagestyle{fancy}
}

\pagestyle{fancy} % 设置 plain style 的属性


\definecolor{mygreen}{rgb}{0,0.6,0}
\definecolor{mygray}{rgb}{0.5,0.5,0.5}
\definecolor{mymauve}{rgb}{0.58,0,0.82}

\newcommand{\hmwkTitle}{Homework\ \#1}
\newcommand{\hmwkDueDate}{October 04, 2021}
\newcommand{\hmwkClass}{Course Name}
\newcommand{\hmwkAuthorName}{Author Name}

\newcommand{\hmwkTitleCN}{作业\#3:\\ 241210 死锁}
\newcommand{\hmwkDueDateCN}{2024年12月11日}
\newcommand{\hmwkAuthorNameCN}{张梓卫}
\newcommand{\hmwkAuthorIDCN}{10235101526}

\if\hmwkCoverPage 1
    \title{
        \vspace{3in}
        \if\hmwkLanguage E
            \huge{\textbf{\hmwkClass}}\\
            \vspace{0.4in}
            \huge{\textbf{\hmwkTitle}}\\
            \vspace{0.4in}
            \normalsize{Due on \hmwkDueDate}\\
        \else
            \Huge{\textbf{\hmwkClassCN}}\\
            \vspace{0.3in}
            \Huge{\textbf{\hmwkTitleCN}}\\
            \vspace{0.4in}
            \normalsize{截止日期:\hmwkDueDateCN}\\
        \fi
        \vspace{2.4in}
    }

    \if\hmwkLanguage E
        \author{\hmwkAuthorName}
    \else
        \author{\hmwkAuthorNameCN (学号:\hmwkAuthorIDCN)}
    \fi
    \date{}
\else
    \title{
        \if\hmwkLanguage E
            \huge{\textbf{\hmwkClass}}\\
            \vspace{0.05in}
            \huge{\textbf{\hmwkTitle}}\\
            \vspace{0.2in}
            \normalsize{Due on \hmwkDueDate}\\
        \else
            \Huge{\textbf{\hmwkClassCN}}\\
            \vspace{0.05in}
            \Huge{\textbf{\hmwkTitleCN}}\\
            \vspace{0.2in}
            \normalsize{截止日期:\hmwkDueDateCN}\\
        \fi
    }

    \if\hmwkLanguage E
        \author{\hmwkAuthorName}
    \else
        \author{\hmwkAuthorNameCN (学号:\hmwkAuthorIDCN)}
    \fi
    \date{}
\fi

% 设置页脚:在每页的右下脚以斜体显示书名

\fancyfoot[RO,RE]{\it Lab Report By \LaTeX} % 使用意大利斜体显示
\renewcommand{\footrulewidth}{0.5pt} % 页脚水平线宽度

% 设置页码:在底部居中显示页码

\pagestyle{fancy}
\fancyfoot[C]{\kaishu 第 \thepage 页 \ 共 \pageref{LastPage} 页} % LastPage 需要二次编译以获取总页数

%——————————————代码块设置———————————————%

\lstset {
    backgroundcolor=\color{white},   % choose the background color; you must add \usepackage{color} or \usepackage{xcolor}
    basicstyle=\footnotesize,        % the size of the fonts that are used for the code
    breakatwhitespace=false,         % sets if automatic breaks should only happen at whitespace
    breaklines=true,                 % sets automatic line breaking
    captionpos=bl,                   % sets the caption-position to bottom
    commentstyle=\color{mygreen},    % comment style
    deletekeywords={...},            % if you want to delete keywords from the given language
    escapeinside={\%*}{*},           % if you want to add LaTeX within your code
    extendedchars=true,              % lets you use non-ASCII characters; for 8-bits encodings only, does not work with UTF-8
    frame=single,                    % adds a frame around the code
    keepspaces=true,                 % keeps spaces in text, useful for keeping indentation of code (possibly needs columns=flexible)
    keywordstyle=\color{blue},       % keyword style
    % language=Python,               % the language of the code
    morekeywords={*,...},            % if you want to add more keywords to the set
    numbers=left,                    % where to put the line-numbers; possible values are (none, left, right)
    numbersep=5pt,                   % how far the line-numbers are from the code
    numberstyle=\tiny\color{mygray}, % the style that is used for the line-numbers
    rulecolor=\color{black},         % if not set, the frame-color may be changed on line-breaks within not-black text (e.g. comments (green here))
    showspaces=false,                % show spaces everywhere adding particular underscores; it overrides 'showstringspaces'
    showstringspaces=false,          % underline spaces within strings only
    showtabs=false,                  % show tabs within strings adding particular underscores
    stepnumber=1,                    % the step between two line-numbers. If it's 1, each line will be numbered
    stringstyle=\color{orange},      % string literal style
    tabsize=2,                       % sets default tabsize to 2 spaces
    % title=Python Code              % show the filename of files included with \lstinputlisting; also try caption instead of title
}

\linespread{1.2}

%———————————————超链接设置——————————————%

\hypersetup{
    pdfstartview=FitH, % 设置PDF文档打开时的初始视图为页面宽度适应窗口宽度(即页面水平适应)
    CJKbookmarks=true, % 用对CJK(中文、日文、韩文)字符的书签支持,确保这些字符在书签中正确显示
    bookmarksnumbered=true, % 书签带有章节编号。这对有章节编号的文档很有用
    bookmarksopen=true, % 文档打开时,书签树是展开的,方便查看所有书签
    colorlinks, % 启用彩色链接。这样,链接在PDF中会显示为彩色,而不是默认的方框
    pdfborder=001, % 设置PDF文档中链接的边框样式。001 表示链接周围没有边框,仅在单击时显示一个矩形
    linkcolor=blue, % 设置文档内部链接(如目录中的章节链接)的颜色为蓝色
    anchorcolor=blue, % 设置锚点链接(即目标在同一文档内的链接)的颜色为蓝色
    citecolor=blue, % 设置引用(如文献引用)的颜色为蓝色
}

% 

\begin{document}

\maketitle
\if\hmwkCoverPage 1
    \pagebreak
\fi


%
% first problem, id is automatically set to 1
%
\begin{homeworkProblem}

\section{7.7}

假设一个系统有4个相同类型的资源,并由三个进程共享。每个进程最多需要2个资源,证明这个系统不会死锁。

\solution

四个资源,三个进程,必然有一个进程持有至少两个资源,该进程可以完成,完成后释放这两个资源,其余进程都可以完成。不会死锁。

数学证明:

\begin{itemize}
    \item 总资源数为 $R$;进程数为 $N$;每个进程最多需要的资源为 $K$。
\end{itemize}

系统不会死锁的条件是:
\[
R \geq (N - 1) \cdot K + 1
\]


$(N-1) \cdot K$ 是防止死锁的关键部分:即使 $N-1$ 个进程各占用最多的 $K$ 个资源,系统仍需要保留至少 1 个资源,来破坏循环等待条件。


在本题中:
\begin{itemize}
    \item $R = 4$(系统有 4 个资源);$N = 3$(系统有 3 个进程);$K = 2$(每个进程最多需要 2 个资源)。
\end{itemize}

\[
R \geq (3 - 1) \cdot 2 + 1 = 4
\]

$R = 4$ 满足该条件,因此系统不会死锁。

\section{7.12}

假设一个系统具有如下快照:

\begin{table}[H]
\centering
\renewcommand{\arraystretch}{1.5} % 设置行高
\begin{tabular}{|c|c|c|c|c|c|c|c|c|}
\hline
\textbf{进程} & \multicolumn{4}{|c|}{\textbf{Allocation}} & \multicolumn{4}{|c|}{\textbf{Max}} \\ \hline
 & A & B & C & D & A & B & C & D \\ \hline
$P_0$ & 3 & 0 & 1 & 4 & 5 & 1 & 1 & 7 \\ \hline
$P_1$ & 2 & 2 & 1 & 0 & 3 & 2 & 1 & 1 \\ \hline
$P_2$ & 3 & 1 & 2 & 1 & 3 & 3 & 2 & 1 \\ \hline
$P_3$ & 0 & 5 & 1 & 0 & 4 & 6 & 1 & 2 \\ \hline
$P_4$ & 4 & 2 & 1 & 2 & 6 & 3 & 2 & 5 \\ \hline
\end{tabular}
\caption{系统资源快照}
\end{table}

采用银行家算法,确定如下每个状态是否安全的。如果状态是安全的,那么说明进程可以完成
的顺序。否则,说明为什么状态是不安全的。

\begin{itemize}
    \item a.Available=(0,3,0,1)
    \item b.Available=(1,0,0,2)
\end{itemize}

\solution

Need矩阵为:

\[
\begin{array}{c|cccc}
\hline
   & A & B & C & D \\ \hline
P_0 & 2 & 1 & 0 & 3 \\
P_1 & 1 & 0 & 0 & 1 \\
P_2 & 0 & 2 & 0 & 0 \\
P_3 & 4 & 1 & 0 & 2 \\
P_4 & 2 & 1 & 1 & 3 \\ \hline
\end{array}
\]

按银行家算法的安全性检查步骤:

1. 找满足 $\text{Need}(P_i) \leq \text{Work}$ 的未完成进程。

   \[
   \text{Need}(P_2) = (0,2,0,0) \leq (0,3,0,1) \quad \checkmark
   \]
   \[
   \text{Work} = \text{Work} + \text{Allocation}(P_2) = (0,3,0,1) + (3,1,2,1) = (3,4,2,2)
   \]
   \[
   \text{Finish}[P_2] = T
   \]

2. 现在 $\text{Work} = (3,4,2,2)$,继续找下一个可执行的进程。

   \[
   \text{Need}(P_1) = (1,0,0,1) \leq (3,4,2,2) \quad \checkmark
   \]
   \[
   \text{Work} = (3,4,2,2) + (2,2,1,0) = (5,6,3,2)
   \]
   \[
   \text{Finish}[P_1] = T
   \]

3. 继续,$\text{Work} = (5,6,3,2)$。

   \[
   \text{Need}(P_3) = (4,1,0,2) \leq (5,6,3,2) \quad \checkmark
   \]
   \[
   \text{Work} = (5,6,3,2) + (0,5,1,0) = (5,11,4,2)
   \]
   \[
   \text{Finish}[P_3] = T
   \]

4. 现在 $\text{Work} = (5,11,4,2)$,剩下 $P_0$ 和 $P_4$ 未完成。

   \[
   \text{Need}(P_4) = (2,1,1,3) \leq (5,11,4,2)\,? 
   \]
   前三项满足,但最后一项 $3 \leq 2$ 不成立,无法执行 $P_4$。

\[
\text{Finish} = [F,T,T,T,F]
\]

仍有未完成进程($P_0,P_4$均无法得到所需资源),因此该状态不安全。

\textbf{故:当 $\text{Available}=(0,3,0,1)$ 时,状态不安全。}

\subsection*{(b) $\text{Available} = (1,0,0,2)$}

\[
\text{Work} = \text{Available} = (1,0,0,2),\quad \text{Finish}=[F,F,F,F,F]
\]

   
   \[
   \text{Need}(P_1)=(1,0,0,1) \leq (1,0,0,2) \quad \checkmark
   \]
   \[
   \text{Work} = (1,0,0,2)+(2,2,1,0) = (3,2,1,2)
   \]
   \[
   \text{Finish}[P_1]=T
   \]

2. $\text{Work}=(3,2,1,2)$,检查 $P_2$:
   \[
   \text{Need}(P_2)=(0,2,0,0) \leq (3,2,1,2) \quad \checkmark
   \]
   \[
   \text{Work}=(3,2,1,2)+(3,1,2,1)=(6,3,3,3)
   \]
   \[
   \text{Finish}[P_2]=T
   \]

3. $\text{Work}=(6,3,3,3)$,检查 $P_3$:
   \[
   \text{Need}(P_3)=(4,1,0,2) \leq (6,3,3,3) \quad \checkmark
   \]
   \[
   \text{Work}=(6,3,3,3)+(0,5,1,0)=(6,8,4,3)
   \]
   \[
   \text{Finish}[P_3]=T
   \]

4. $\text{Work}=(6,8,4,3)$,检查 $P_4$:
   \[
   \text{Need}(P_4)=(2,1,1,3) \leq (6,8,4,3) \quad \checkmark
   \]
   \[
   \text{Work}=(6,8,4,3)+(4,2,1,2)=(10,10,5,5)
   \]
   \[
   \text{Finish}[P_4]=T
   \]

   \[
   \text{Need}(P_0)=(2,1,0,3) \leq (10,10,5,5) \quad \checkmark
   \]
   \[
   \text{Work}=(10,10,5,5)+(3,0,1,4)=(13,10,6,9)
   \]
   \[
   \text{Finish}[P_0]=T
   \]

此时:
\[
\text{Finish}=[T,T,T,T,T]
\]

所有进程均能完成。

执行顺序为 $P_1 \rightarrow P_2 \rightarrow P_3 \rightarrow P_4 \rightarrow P_0$。

\textbf{结论:当 $\text{Available}=(1,0,0,2)$ 时,状态安全,可行的安全序列为:$(P_1, P_2, P_3, P_4, P_0)$。}


\section{7.13}

假设一个系统具有如下快照:

\begin{table}[H]
    \centering
    \renewcommand{\arraystretch}{1.5} % 设置行高
    \begin{tabular}{|c|c|c|c|c|c|c|c|c|c|}
    \hline
    \textbf{进程} & \multicolumn{4}{|c|}{\textbf{Allocation}} & \multicolumn{4}{|c|}{\textbf{Max}} & \textbf{Available} \\ \hline
     & A & B & C & D & A & B & C & D & A, B, C, D \\ \hline
    $P_0$ & 2 & 0 & 0 & 1 & 4 & 2 & 1 & 2 & 3, 3, 2, 1 \\ \hline
    $P_1$ & 3 & 1 & 2 & 1 & 5 & 2 & 5 & 2 & \\ \hline
    $P_2$ & 2 & 1 & 0 & 3 & 2 & 3 & 1 & 6 & \\ \hline
    $P_3$ & 1 & 3 & 1 & 2 & 1 & 4 & 2 & 4 & \\ \hline
    $P_4$ & 1 & 4 & 3 & 2 & 3 & 6 & 6 & 5 & \\ \hline
    \end{tabular}
    \caption{系统资源快照}
    \end{table}

采用银行家算法,回答下面的问题:

\begin{enumerate}
    \item[a.] 通过进程可以完成执行的顺序,说明系统处于安全状态。
    \item[b.] 当进程 $P_1$ 的请求为 $(1, 1, 0, 0)$ 时,能否立即允许这一请求?
    \item[c.] 当进程 $P_4$ 的请求为 $(0, 0, 2, 0)$ 时,能否立即允许这一请求?
\end{enumerate}

\solution

$\text{Need}$ 矩阵为:
\[
\begin{array}{c|cccc}
\hline
   & A & B & C & D \\ \hline
P_0 & 2 & 2 & 1 & 1 \\
P_1 & 2 & 1 & 3 & 1 \\
P_2 & 0 & 2 & 1 & 3 \\
P_3 & 0 & 1 & 1 & 2 \\
P_4 & 2 & 2 & 3 & 3 \\
\hline
\end{array}
\]


初始:
\[
\text{Work} = \text{Available} = (3,3,2,1), \quad \text{Finish} = [F,F,F,F,F]
\]

检查可满足的进程(Need $\leq$ Work):

   \[
   \text{Need}(P_0) = (2,2,1,1) \leq (3,3,2,1)\, \checkmark
   \]
   \[
   \text{Work} = (3,3,2,1) + \text{Allocation}(P_0) = (3,3,2,1)+(2,0,0,1) = (5,3,2,2)
   \]
   \[
   \text{Finish}[P_0] = T
   \]

2. 现在 $\text{Work}=(5,3,2,2)$,检查其他进程:
   
   \[
   \text{Need}(P_3)=(0,1,1,2) \leq (5,3,2,2)\, \checkmark
   \]
   \[
   \text{Work}=(5,3,2,2) + (1,3,1,2) = (6,6,3,4)
   \]
   \[
   \text{Finish}[P_3] = T
   \]

3. $\text{Work}=(6,6,3,4)$,继续:
   
   \[
   \text{Need}(P_2)=(0,2,1,3) \leq (6,6,3,4)\, \checkmark
   \]
   \[
   \text{Work}=(6,6,3,4) + (2,1,0,3) = (8,7,3,7)
   \]
   \[
   \text{Finish}[P_2] = T
   \]

4. $\text{Work}=(8,7,3,7)$,检查 $P_1$:
   \[
   \text{Need}(P_1)=(2,1,3,1) \leq (8,7,3,7)\, \checkmark
   \]
   \[
   \text{Work}=(8,7,3,7) + (3,1,2,1) = (11,8,5,8)
   \]
   \[
   \text{Finish}[P_1]=T
   \]

5. $\text{Work}=(11,8,5,8)$,最后检查 $P_4$:
   \[
   \text{Need}(P_4)=(2,2,3,3) \leq (11,8,5,8)\, \checkmark
   \]
   \[
   \text{Work}=(11,8,5,8) + (1,4,3,2)=(12,12,8,10)
   \]
   \[
   \text{Finish}[P_4]=T
   \]

所有进程均可完成:
\[
\text{Finish}=[T,T,T,T,T]
\]

安全序列为:
\[
(P_0 \rightarrow P_3 \rightarrow P_2 \rightarrow P_1 \rightarrow P_4).
\]

故系统处于安全状态。

\section*{(b) 当进程 $P_1$ 的请求为 $(1,1,0,0)$ 时,能否立即允许这一请求?}

请求向量为 $R_1=(1,1,0,0)$。

检查:
1. 请求是否不超过其需要:
   \[
   \text{Need}(P_1)=(2,1,3,1), \quad R_1=(1,1,0,0)
   \]
   \[
   (1 \leq 2,\,1 \leq 1,\,0 \leq 3,\,0 \leq 1)\, \checkmark
   \]

2. 请求是否不超过当前可用:
   \[
   \text{Available}=(3,3,2,1),\; R_1=(1,1,0,0)
   \]
   \[
   (1 \leq 3,\,1 \leq 3,\,0 \leq 2,\,0 \leq 1)\, \checkmark
   \]

可尝试分配:  
分配后更新:
\[
\text{Available}_{new} = (3,3,2,1)-(1,1,0,0)=(2,2,2,1)
\]
\[
\text{Allocation}(P_1)_{new}=(3+1,1+1,2+0,1+0)=(4,2,2,1)
\]
\[
\text{Need}(P_1)_{new}=(2-1,1-1,3-0,1-0)=(1,0,3,1)
\]

在此新状态下重复安全性检查:

\[
\text{Work}=(2,2,2,1)
\]

- 检查 $P_0$: Need$(P_0)=(2,2,1,1)\leq(2,2,2,1)$? 是的。执行 $P_0$:
  \[
  \text{Work}=(2,2,2,1)+(2,0,0,1)=(4,2,2,2)
  \]

- 检查 $P_3$: Need$(P_3)=(0,1,1,2) \leq (4,2,2,2)$? 是的。执行 $P_3$:
  \[
  \text{Work}=(4,2,2,2)+(1,3,1,2)=(5,5,3,4)
  \]

- 检查 $P_2$: Need$(P_2)=(0,2,1,3)\leq(5,5,3,4)$? 是的。执行 $P_2$:
  \[
  \text{Work}=(5,5,3,4)+(2,1,0,3)=(7,6,3,7)
  \]

- 检查更新后 $P_1$: Need$(P_1)=(1,0,3,1)\leq(7,6,3,7)$? 是的。执行 $P_1$:
  \[
  \text{Work}=(7,6,3,7)+(4,2,2,1)=(11,8,5,8)
  \]

- 检查 $P_4$: Need$(P_4)=(2,2,3,3)\leq(11,8,5,8)$? 是的。执行 $P_4$:
  \[
  \text{Work}=(11,8,5,8)+(1,4,3,2)=(12,12,8,10)
  \]

所有进程均可完成,状态仍然安全。

\textbf{结论:对于 $P_1$ 的请求 $(1,1,0,0)$ 可以立即满足。}

\section*{(c) 当进程 $P_4$ 的请求为 $(0,0,2,0)$ 时能否立即满足?}

检查 $P_4$ 的请求 $R_4=(0,0,2,0)$:

1. 请求是否不超过其需要:
   \[
   \text{Need}(P_4)=(2,2,3,3), \quad R_4=(0,0,2,0)
   \]
   \[
   (0 \leq 2,\,0 \leq 2,\,2 \leq 3,\,0 \leq 3)\, \checkmark
   \]

2. 请求是否不超过当前可用:
   \[
   \text{Available}=(3,3,2,1), \quad R_4=(0,0,2,0)
   \]
   \[
   (0 \leq 3,\,0 \leq 3,\,2 \leq 2,\,0 \leq 1) \, \checkmark
   \]

可暂时分配该请求:

更新后:
\[
\text{Available}_{new}=(3,3,2,1)-(0,0,2,0)=(3,3,0,1)
\]
\[
\text{Allocation}(P_4)_{new}=(1,4,3+2,2)=(1,4,5,2)
\]
\[
\text{Need}(P_4)_{new}=(2,2,3-2,3)=(2,2,1,3)
\]

再次进行安全性检查:

\[
\text{Work}=(3,3,0,1)
\]

依次检查进程:

- $P_0$: Need$(P_0)=(2,2,1,1)\leq(3,3,0,1)$?
  
  检查C资源:Need C=1 $\leq$ Work C=0? 不成立。$P_0$无法执行。

- $P_3$: Need$(P_3)=(0,1,1,2)\leq(3,3,0,1)$?
  
  Need C=1 $\leq$0? 不成立。$P_3$无法执行。

- $P_2$: Need$(P_2)=(0,2,1,3)\leq(3,3,0,1)$?
  
  Need C=1 $\leq$0? 不成立。$P_2$无法执行。

- $P_1$: Need$(P_1)=(2,1,3,1)\leq(3,3,0,1)$?
  
  Need C=3 $\leq$0? 不成立。$P_1$无法执行。

- $P_4$: Need$(P_4)_{new}=(2,2,1,3)\leq(3,3,0,1)$?
  
  Need C=1 $\leq$0? 不成立。$P_4$无法执行。

没有任何进程可以执行,系统进入死锁状态,故不安全。

\textbf{对于 $P_4$ 的请求 $(0,0,2,0)$ 不可立即满足,否则系统不安全。}



\end{homeworkProblem}

\bibliographystyle{abbrv}
\bibliography{references.bib}

\end{document}