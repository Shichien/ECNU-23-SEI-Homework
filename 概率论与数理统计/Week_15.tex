\documentclass[twoside]{article}
\usepackage[paperwidth=210mm, paperheight=297mm, margin=2cm]{geometry}
\usepackage[utf8]{inputenc}
\usepackage{ctex}

% Formatting Packages ——————————————————————————————————————
\usepackage{multicol}
\usepackage{multirow}
\usepackage{enumitem}
\usepackage{indentfirst}
\usepackage[toc]{multitoc}

% Math & Physics Packages ————————————————————————————
\usepackage{amsmath, amsthm, amsfonts, amssymb}
\usepackage{amssymb}
\usepackage{setspace}
\usepackage{physics}
\usepackage{cancel}
\usepackage{nicefrac}

% Image-related Packages —————————————————————————————
\usepackage{graphics, graphicx}
\usepackage{tikz}
\usetikzlibrary{arrows.meta}
\usepackage{pgfplots}
\pgfplotsset{compat=1.18}
\usepackage{xcolor}
\usepackage{fourier-orns}
\usepackage{lipsum}


%% Silakan otak-atik judul di sini
\title{\texorpdfstring{\vspace{-1.5em}}{}\textbf{概率论与数理统计}}
\author{张梓卫 10235101526}
\date{\the\day \ \monthyear}

% Kalau mau bab pertama nomornya 0, ganti 0 jadi -1
\setcounter{section}{0}

\renewcommand*\contentsname{第十三周概率论作业}
\renewcommand*{\multicolumntoc}{2}
\setlength{\columnseprule}{0.5pt}

% Layouting Packages ————————————————————————————————————————
\usepackage{titlesec}
\usepackage{fancyhdr}
\pagestyle{fancy}
\setlength{\headheight}{14.39996pt}
\fancyfoot[C]{\textit{By Deralive (10235101526)}}
\fancyfoot[R]{\thepage}

\fancyhead[LE, RO]{\textsl{\rightmark}}
\fancyhead[LO, RE]{\textsl{\leftmark}}

\renewcommand\headrule{
	\vspace{-6pt}
	\hrulefill
	\raisebox{-2.1pt}
	{\quad\floweroneleft\decoone\floweroneright\quad}
	\hrulefill}

\renewcommand\footrule{
	\hrulefill
	\raisebox{-2.1pt}
	{\quad\floweroneleft\decoone\floweroneright\quad}
	\hrulefill}
  
\newcommand{\fpb}[2]{\mathrm{FPB}(#1, #2)}
\newcommand{\kpk}[2]{\mathrm{KPK}(#1, #2)}

% Reference and Bibliography Packages ————————————————————
\usepackage{hyperref}
\hypersetup{
    colorlinks=true,
    linkcolor={black},
    citecolor={biru!70!black},
    urlcolor={biru!80!black}
}

\numberwithin{equation}{section}

% Silakan lihat dokumentasi package biblatex
% untuk format sitasi yang diperlukan
\usepackage[backend=biber]{biblatex}
\addbibresource{ref.bib}
\makeatletter
\def\@biblabel#1{}
\makeatother

\renewcommand{\mod}{\mathrm{mod} \ }

% Titling
\newcommand{\garis} [3] []{
	\begin{center}
		\begin{tikzpicture}
			\draw[#2-#3, ultra thick, #1] (0,0) to (1\linewidth,0);
		\end{tikzpicture}
	\end{center}
}

\newcommand{\monthyear}{
  \ifcase\month\or January\or February\or March\or April\or May\or June\or
  July\or August\or September\or October\or November\or
  December\fi\space\number\year
}

% Colour Palette ——————————————————————————————————————
\definecolor{merah}{HTML}{F4564E}
\definecolor{merahtua}{HTML}{89313E}
\definecolor{biru}{HTML}{60BBE5}
\definecolor{birutua}{HTML}{412F66}
\definecolor{hijau}{HTML}{59CC78}
\definecolor{hijautua}{HTML}{366D5B}
\definecolor{kuning}{HTML}{FFD56B}
\definecolor{jingga}{HTML}{FBA15F}
\definecolor{ungu}{HTML}{8C5FBF}
\definecolor{lavender}{HTML}{CBA5E8}
\definecolor{merjamb}{HTML}{FFB6E0}
\definecolor{mygray}{HTML}{E6E6E6}

% Theorems ————————————————————————————————————————————
\usepackage{tcolorbox}
\tcbuselibrary{skins,breakable,theorems}
\usepackage{changepage}

\newcounter{hitung}
\setcounter{hitung}{\thesection}

\makeatletter
	% Proof 证明如下
	\def\tcb@theo@widetitle#1#2#3{\hbox to \textwidth{\textsc{\large#1}\normalsize\space#3\hfil(#2)}}
	\tcbset{
		theorem style/theorem wide name and number/.code={ \let\tcb@theo@title=\tcb@theo@widetitle},
		proofbox/.style={skin=enhancedmiddle,breakable,parbox=false,boxrule=0mm,
			check odd page, toggle left and right, colframe=black!20!white!92!hijau,
			leftrule=8pt, rightrule=0mm, boxsep=0mm,arc=0mm, outer arc=0mm,
			left=3mm,right=3mm,top=0mm,bottom=0mm, toptitle=0mm,
			bottomtitle=0mm,colback=gray!3!white!98!biru, before skip=8pt, after skip=8pt,
			before={\par\vskip-2pt},after={\par\smallbreak},
		},
	}
	\newtcolorbox{ProofBox}{proofbox}
	\makeatother
	
	\let\realproof\proof
	\let\realendproof\endproof
	\renewenvironment{proof}[1][Prove:]{\ProofBox\strut\textsc{#1}\space}{\endProofBox}
        \AtEndEnvironment{proof}{\null\hfill$\blacksquare$}
        % Definition 定义环境
	\newtcbtheorem[use counter=hitung, number within=section]{dfn}{定义}
	{theorem style=theorem wide name and number,breakable,enhanced,arc=3.5mm,outer arc=3.5mm,
		boxrule=0pt,toprule=1pt,leftrule=0pt,bottomrule=1pt, rightrule=0pt,left=0.2cm,right=0.2cm,
		titlerule=0.5em,toptitle=0.1cm,bottomtitle=-0.1cm,top=0.2cm,
		colframe=white!10!biru,
		colback=white!90!biru,
		coltitle=white,
		shadow={1.3mm}{-1.3mm}{0mm}{gray!50!white}, % 添加阴影
        coltext=birutua!60!gray, title style={white!10!biru}, rbefoe skip=8pt, after skip=8pt,
		fonttitle=\bfseries,fontupper=\normalsize}{dfn}

	% 答题卡
	\newtcbtheorem[use counter=hitung, number within=section]{ans}{解答}
	{theorem style=theorem wide name and number,breakable,enhanced,arc=3.5mm,outer arc=3.5mm,
		boxrule=0pt,toprule=1pt,leftrule=0pt,bottomrule=1pt, rightrule=0pt,left=0.2cm,right=0.2cm,
		titlerule=0.5em,toptitle=0.1cm,bottomtitle=-0.1cm,top=0.2cm,
		colframe=white!10!biru,
		colback=white!90!biru,
		coltitle=white,
		shadow={1.3mm}{-1.3mm}{0mm}{gray!50!white}, % 添加阴影
        coltext=birutua!60!gray, title style={white!10!biru}, before skip=8pt, after skip=8pt,
		fonttitle=\bfseries,fontupper=\normalsize}{ans}

	% Axiom
	\newtcbtheorem[use counter=hitung, number within=section]{axm}{公理}
	{theorem style=theorem wide name and number,breakable,enhanced,arc=3.5mm,outer arc=3.5mm,
		boxrule=0pt,toprule=1pt,leftrule=0pt,bottomrule=1pt, rightrule=0pt,left=0.2cm,right=0.2cm,
		titlerule=0.5em,toptitle=0.1cm,bottomtitle=-0.1cm,top=0.2cm,
		colframe=white!10!biru,colback=white!90!biru,coltitle=white,
		shadow={1.3mm}{-1.3mm}{0mm}{gray!50!white!90}, % 添加阴影
        coltext=birutua!60!gray,title style={white!10!biru},before skip=8pt, after skip=8pt,
		fonttitle=\bfseries,fontupper=\normalsize}{axm}
 
	% Theorem
	\newtcbtheorem[use counter=hitung, number within=section]{thm}{定理}
	{theorem style=theorem wide name and number,breakable,enhanced,arc=3.5mm,outer arc=3.5mm,
		boxrule=0pt,toprule=1pt,leftrule=0pt,bottomrule=1pt, rightrule=0pt,left=0.2cm,right=0.2cm,
		titlerule=0.5em,toptitle=0.1cm,bottomtitle=-0.1cm,top=0.2cm,
		colframe=white!10!merah,colback=white!75!pink,coltitle=white, coltext=merahtua!80!merah,
		shadow={1.3mm}{-1.3mm}{0mm}{gray!50!white!90}, % 添加阴影
		title style={white!10!merah}, before skip=8pt, after skip=8pt,
		fonttitle=\bfseries,fontupper=\normalsize}{thm}
	
	% Proposition
	\newtcbtheorem[use counter=hitung, number within=section]{prp}{命题}
	{theorem style=theorem wide name and number,breakable,enhanced,arc=3.5mm,outer arc=3.5mm,
		boxrule=0pt,toprule=1pt,leftrule=0pt,bottomrule=1pt, rightrule=0pt,left=0.2cm,right=0.2cm,
		titlerule=0.5em,toptitle=0.1cm,bottomtitle=-0.1cm,top=0.2cm,
		colframe=white!10!hijau,colback=white!90!hijau,coltitle=white, coltext=hijautua!80!brown,
		shadow={1.3mm}{-1.3mm}{0mm}{gray!50!white}, % 添加阴影
		title style={white!10!hijau}, before skip=8pt, after skip=8pt,
		fonttitle=\bfseries,fontupper=\normalsize}{prp}


	% Example
	\newtcolorbox[use counter=hitung, number within=section]{cth}[1][]{breakable,
		colframe=white!10!jingga, coltitle=white!90!jingga, colback=white!85!jingga, coltext=black!10!brown!50!jingga, colbacktitle=white!10!jingga, enhanced, fonttitle=\bfseries,fontupper=\normalsize, attach boxed title to top left={yshift=-2mm}, before skip=8pt, after skip=8pt,
		title=Contoh~\thetcbcounter \ \ #1}

	% Catatan/Note
	\newtcolorbox{ctt}[1][]{enhanced, 
		left=4.1mm, borderline west={8pt}{0pt}{white!10!kuning}, 
		before skip=6pt, after skip=6pt, 
		colback=white!85!kuning, colframe= white!85!kuning, coltitle=orange!60!kuning!25!brown, coltext=orange!60!kuning!25!brown,
		fonttitle=\bfseries,fontupper=\normalsize, before skip=8pt, after skip=8pt,
		title=\underline{Catatan}  #1}
	
	% Komentar/Remark
	\newtcolorbox{rmr}[1][]{
		,arc=0mm,outer arc=0mm,
		boxrule=0pt,toprule=1pt,leftrule=0pt,bottomrule=5pt, rightrule=0pt,left=0.2cm,right=0.2cm,
		titlerule=0.5em,toptitle=0.1cm,bottomtitle=-0.1cm,top=0.2cm,
		colframe=white!10!kuning,colback=white!85!kuning,coltitle=white, coltext=orange!60!kuning,
		fonttitle=\bfseries,fontupper=\normalsize, before skip=8pt, after skip=8pt,
		title=Komentar  #1}

% ————————————————————————————————————————————————————————————————————————
\begin{document}

\maketitle
\vspace{-3.5em}
\garis{Kite}{Kite}

\tableofcontents

\section{第七章习题 17}


分别使用金球和铅球测定引力常数(以 $10^{-11} \, \text{m}^3 \cdot \text{kg}^{-1} \cdot \text{s}^{-2}$ 为单位),计算以下内容:
    
1. 用金球测定数据 $6.683, 6.681, 6.676, 6.678, 6.679, 6.672$ 的均值 $\mu$ 和置信水平为 $0.9$ 的置信区间;

2. 用铅球测定数据 $6.661, 6.664, 6.667, 6.667, 6.664$ 的均值 $\mu$ 和置信水平为 $0.9$ 的置信区间;

3. 分别计算两组数据的方差 $\sigma^2$ 及其置信区间。

\begin{ans}{17}{17}

    \subsection*{(1) 金球数据的均值 $\mu$ 和置信区间}
    
    数据量为 $n = 6$,样本均值 $\bar{x}$ 和样本标准差 $s$ 分别计算如下:
    \[
    \bar{x} = \frac{\sum x_i}{n} = \frac{6.683 + 6.681 + 6.676 + 6.678 + 6.679 + 6.672}{6} = 6.678
    \]
    \[
    s = \sqrt{\frac{\sum (x_i - \bar{x})^2}{n-1}} = 0.00387
    \]
    
    置信水平为 $0.9$,自由度 $n-1 = 5$,查 $t$ 分布表得 $t_{0.05, 5} = 2.015$。置信区间计算公式为:
    \[
    \bar{x} \pm t_{\alpha/2} \cdot \frac{s}{\sqrt{n}}
    \]
    代入得:
    \[
    6.678 \pm 2.015 \cdot \frac{0.00387}{\sqrt{6}} = 6.678 \pm 0.003
    \]
    即金球测定数据的置信区间为:
    \[
    (6.675, 6.681)
    \]
    
    \subsection*{(2) 铅球数据的均值 $\mu$ 和置信区间}
    
    与(1)同理,这里就把解题过程复制下来了。
    
    数据量为 $n = 5$,样本均值 $\bar{x}$ 和样本标准差 $s$ 分别计算如下:
    \[
    \bar{x} = \frac{\sum x_i}{n} = \frac{6.661 + 6.664 + 6.667 + 6.667 + 6.664}{5} = 6.664
    \]
    \[
    s = \sqrt{\frac{\sum (x_i - \bar{x})^2}{n-1}} = 0.003
    \]
    
    置信水平为 $0.9$,自由度 $n-1 = 4$,查 $t$ 分布表得 $t_{0.05, 4} = 2.1318$。置信区间计算公式为:
    \[
    \bar{x} \pm t_{\alpha/2} \cdot \frac{s}{\sqrt{n}}
    \]
    代入数据得:
    \[
    6.664 \pm 2.1318 \cdot \frac{0.003}{\sqrt{5}} = 6.664 \pm 0.003
    \]
    即铅球测定数据的置信区间为:
    \[
    (6.661, 6.667)
    \]
    
    \subsection*{(3) 两组数据方差的置信区间}
    
    根据题意,样本方差的置信区间计算公式为:
    \[
    \left( \frac{(n-1)s^2}{\chi^2_{1-\alpha/2}(n-1)}, \frac{(n-1)s^2}{\chi^2_{\alpha/2}(n-1)} \right)
    \]
    其中 $\chi^2_{1-\alpha/2}$ 和 $\chi^2_{\alpha/2}$ 为卡方分布的分位数。
    
    \textbf{金球:}

    \[
    \chi^2_{0.95}(5) = 1.145, \quad \chi^2_{0.05}(5) = 11.070
    \]
    \[
    \text{方差置信区间} = \left( \frac{(6-1) \cdot 0.00387^2}{11.070}, \frac{(6-1) \cdot 0.00387^2}{1.145} \right)
    \]
    \[
    = (6.8 \times 10^{-6}, 6.5 \times 10^{-5})
    \]
    
    \textbf{铅球:}

    \[
    \chi^2_{0.95}(4) = 0.711, \quad \chi^2_{0.05}(4) = 9.488
    \]
    \[
    \text{方差置信区间} = \left( \frac{(5-1) \cdot 0.003^2}{9.488}, \frac{(5-1) \cdot 0.003^2}{0.711} \right)
    \]
    \[
    = (3.8 \times 10^{-6}, 5.06 \times 10^{-5})
    \]
\end{ans}

\section{第七章习题 20}

在第 17 题中,设用金球和用铅球测定时测定值总方差相等,求两个测定值总体均值差的置信水平为 $0.9$ 的置信区间。

\begin{ans}{20}{20}

    由于假设 $\sigma_1^2 = \sigma_2^2 = \sigma^2$ 未知,因此:
    \[
    \frac{\bar{X}_1 - \bar{X}_2 - (\mu_1 - \mu_2)}{S_W \sqrt{\frac{1}{n_1} + \frac{1}{n_2}}} \sim t(n_1 + n_2 - 2), S_W = \sqrt{\frac{(n_1 - 1) S_1^2 + (n_2 - 1) S_2^2}{n_1 + n_2 - 2}}.
    \]

    
    根据公式,可得:
    \[
    P \left( -t_{\alpha/2}(n_1 + n_2 - 2) \leq \frac{\bar{X}_1 - \bar{X}_2 - (\mu_1 - \mu_2)}{S_W \sqrt{\frac{1}{n_1} + \frac{1}{n_2}}} \leq t_{\alpha/2}(n_1 + n_2 - 2) \right) = 1 - \alpha.
    \]
    
    不等式整理后得:
    \[
    \bar{X}_1 - \bar{X}_2 - t_{\alpha/2}(n_1 + n_2 - 2) S_W \sqrt{\frac{1}{n_1} + \frac{1}{n_2}} \leq \mu_1 - \mu_2 \leq \bar{X}_1 - \bar{X}_2 + t_{\alpha/2}(n_1 + n_2 - 2) S_W \sqrt{\frac{1}{n_1} + \frac{1}{n_2}}.
    \]
    
    即 $\mu_1 - \mu_2$ 的一个置信水平为 $1-\alpha$ 的置信区间为:
    \[
    \left( \bar{X}_1 - \bar{X}_2 \pm t_{\alpha/2}(n_1 + n_2 - 2) S_W \sqrt{\frac{1}{n_1} + \frac{1}{n_2}} \right).
    \]
    
    \subsection*{代入数据计算}
    
    \[
    \bar{X}_1 = 6.678, \quad S_1 = 0.00387, \quad n_1 = 6; \bar{X}_2 = 6.664, \quad S_2 = 0.003, \quad n_2 = 5.
    \]
    
    自由度:
    \(
    n_1 + n_2 - 2 = 6 + 5 - 2 = 9.
    \)
    
    查 $t$ 分布表得 $t_{0.05}(9) = 1.833$。
    
    计算 $S_W$:
    \[
    S_W = \sqrt{\frac{(n_1 - 1) S_1^2 + (n_2 - 1) S_2^2}{n_1 + n_2 - 2}} = \sqrt{\frac{(6-1) \cdot 0.00387^2 + (5-1) \cdot 0.003^2}{9}} = \sqrt{0.000012333} = 0.00351.
    \]

    计算置信区间:
    \[
    \bar{X}_1 - \bar{X}_2 = 6.678 - 6.664 = 0.014.
    \]
    
    \[
    \left( 0.014 \pm 1.833 \cdot 0.00351 \cdot \sqrt{\frac{1}{6} + \frac{1}{5}} \right).
    \]
    
    \[
    \left( 0.014 \pm 1.833 \cdot 0.00351 \cdot 0.6055 \right).
    \]
    
    所以,置信区间为:
    \[
    (0.014 - 0.0039, 0.014 + 0.0039) = (0.010, 0.018).
    \]
\end{ans}

\section{第七章习题 23}


设两位化验员 $A$ 和 $B$ 独立地对某种聚合物含氯量用相同的方法各做 10 次测定,其测定值的样本方差依次为:
\[
s_A^2 = 0.5419, \quad s_B^2 = 0.6065.
\]
设 $\sigma_A^2, \sigma_B^2$ 分别为 $A, B$ 所测定的测定值总体的方差。设总体均为正态,且两样本独立。求方差比 $\sigma_A^2 / \sigma_B^2$ 的置信水平为 $0.95$ 的置信区间。


\begin{ans}{23}{23}
    \[
    n_1 = n_2 = 10, \quad s_A^2 = 0.5419, \quad s_B^2 = 0.6065,
    \]
    \[
    1 - \alpha = 0.95, \quad \alpha = 0.05, \quad \alpha/2 = 0.025.
    \]
    
    统计量
    \(
    F = \frac{s_A^2}{s_B^2}
    \)
    服从自由度为 $(n_1 - 1, n_2 - 1)$ 的 $F$ 分布。此处自由度为 $(9, 9)$。
    
    查 $F$ 分布表得:
    \(
    F_{0.025}(9, 9) = 4.03, \quad F_{0.975}(9, 9) = 0.2481.
    \)
    
    方差比 $\sigma_A^2 / \sigma_B^2$ 的置信水平为 $0.95$ 的置信区间为:
    \(
    (0.222, 3.601).
    \)

\end{ans}

\section{第七章习题 24}

在一批货物的容量为 $100$ 的样本中,经检验发现有 $16$ 只次品。

试求这批货物次品率的置信水平为 $0.95$ 的置信区间。
    
\begin{ans}{24}{24}

    本题是 $(0,1)$ 分布总体 $X$ 的参数 $p$ 的区间估计问题,样本容量为 $n=100$。
    
    是一个大样本,$X$ 的分布律为:
    \[
    f(x, p) = p^x (1-p)^{1-x}, \quad x = 0, 1.
    \]
    
    设 $X_1, X_2, \ldots, X_n$ 是一个样本,由中心极限定理知,近似地有:
    \[
    \frac{\sum_{i=1}^n X_i - np}{\sqrt{np(1-p)}} \sim N(0, 1), 写为     \frac{n\bar{X} - np}{\sqrt{np(1-p)}} \sim N(0, 1).
    \]
    
    从而可以得到:
    \[
    P\left( -z_{\alpha/2} < \frac{n\bar{X} - np}{\sqrt{np(1-p)}} < z_{\alpha/2} \right) \approx 1 - \alpha.
    \]

    在这里面满足以下条件:
    \[
    a = n + z_{\alpha/2}^2, \quad b = -(2n\bar{X} + z_{\alpha/2}^2), \quad c = n\bar{X}^2.
    \]
    
    \[
    n = 100, \quad \bar{X} = \frac{16}{100} = 0.16, \quad 1-\alpha = 0.95, \quad \alpha/2 = 0.025, \quad z_{0.025} = 1.96.
    \]
    
    \[
    a = n + z_{\alpha/2}^2 = 100 + 1.96^2 = 100 + 3.8416 = 103.8416,
    \]
    \[
    b = -(2n\bar{X} + z_{\alpha/2}^2) = -(2 \cdot 100 \cdot 0.16 + 1.96^2) = -(32 + 3.8416) = -35.8416,
    \]
    \[
    c = n\bar{X}^2 = 100 \cdot 0.16^2 = 100 \cdot 0.0256 = 2.56.
    \]
    
    计算 $p_1$ 和 $p_2$:
    \[
    p_1 = \frac{-b - \sqrt{b^2 - 4ac}}{2a}, \quad p_2 = \frac{-b + \sqrt{b^2 - 4ac}}{2a}.
    \]
    
    先计算 $\sqrt{b^2 - 4ac}$:
    \[
    \sqrt{b^2 - 4ac} = \sqrt{221.7751} \approx 14.8933.
    \]
    
    继续计算 $p_1$ 和 $p_2$:
    \[
    p_1 = \frac{-(-35.8416) - 14.8933}{2 \cdot 103.8416} = \frac{35.8416 - 14.8933}{207.6832} = \frac{20.9483}{207.6832} \approx 0.101.
    \]
    \[
    p_2 = \frac{-(-35.8416) + 14.8933}{2 \cdot 103.8416} = \frac{35.8416 + 14.8933}{207.6832} = \frac{50.7349}{207.6832} \approx 0.244.
    \]
    
    
    故货物次品率的置信水平为 $0.95$ 的置信区间为:
    \(
    (0.101, 0.244).
    \)
    
\end{ans}

\section{第七章习题 26}

为研究某种汽车轮胎的磨损特性,随机地选择 16 只轮胎,每只轮胎行驶到磨坏为止,记录所有行驶的路程(以 km 计)如下:
\[
41250, \ 40187, \ 43175, \ 41010, \ 39265, \ 41872, \ 42654, \ 41287,
\]
\[
38970, \ 40200, \ 42550, \ 41095, \ 40680, \ 43500, \ 39775, \ 40400.
\]

假设这些数据来自正态总体 $N(\mu, \sigma^2)$,其中 $\mu, \sigma^2$ 未知,试求 $\mu$ 的置信水平为 $0.95$ 的单侧置信下限。

\begin{ans}{26}{26}

        \[
        P\left( \frac{\bar{X} - \mu}{S / \sqrt{n}} < t_\alpha(n-1) \right) = 1 - \alpha.
        \]
        
        整理得:
        \[
        P\left( \mu > \bar{X} - \frac{S}{\sqrt{n}} \cdot t_\alpha(n-1) \right) = 1 - \alpha.
        \]
        
        因此,$\mu$ 的置信水平为 $1-\alpha$ 的单侧置信下限为:
        \[
        \mu = \bar{X} - \frac{S}{\sqrt{n}} \cdot t_\alpha(n-1).
        \]
        
        已知样本量 $n=16$,样本均值 $\bar{X} = 41116.875$,
        
        样本标准差 $S = 1346.842$,置信水平 $1-\alpha = 0.95$,$\alpha = 0.05$。
        
        查 $t$ 分布表得 $t_{0.05}(15) = 1.7531$。
        
        将数据代入公式,计算单侧置信下限:
        \[
        \mu = \bar{X} - \frac{S}{\sqrt{n}} \cdot t_\alpha(n-1).
        \]
        
        $\frac{S}{\sqrt{n}}$:
        \[
        \frac{S}{\sqrt{n}} = \frac{1346.842}{\sqrt{16}} = \frac{1346.842}{4} = 336.7105.
        \]
        
        下限:
        \[
        \mu = 41116.875 - 336.7105 \cdot 1.7531 = 41116.875 - 590.4354 = 40526.4396.
        \]
\end{ans}

\end{document}