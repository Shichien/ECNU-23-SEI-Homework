% \date{May 14, 2024}
% \author{Deralive}
% \title{华东师范大学软件学院实验报告模板}
% 注意事项:编译两次,以确保目录、页码完整显示

\def\allfiles{}

%————————————多文件编译————————————%
% \ifx\allfiles\undefined
% 	    \begin{document}
% \else
% \fi

% Content

% \ifx\allfiles\undefined
% 	    \end{document}
% 	\else
% 	\fi
%—————————————————————————————————%

\documentclass[14pt,a4paper,UTF8,twoside]{article}

\usepackage{amsmath}
\usepackage{graphicx}
\usepackage{geometry} 
\usepackage{ctex}
\usepackage{booktabs} % 表格库
\usepackage{titlesec} % 标题库
\usepackage{fancyhdr} % 页眉页脚库
\usepackage{lastpage} % 页码数库
\usepackage{listings} % 代码块包
\usepackage{xcolor}
\usepackage[hidelinks]{hyperref}
\usepackage{tikz}
\usepackage{tikz-qtree}
\usepackage{fontspec} % 允许设置字体
\usepackage{unicode-math} % 允许数学公式使用特定字体
\usepackage{mwe}
\usepackage{zhlipsum} % 中文乱数文本
\usepackage{amsmath}
\usepackage{xcolor}
\usepackage{float} % 浮动体环境
\usepackage{subcaption} % 子图包
\usepackage{biblatex}
\addbibresource{references.bib} % 指定你的.bib文件名称

\definecolor{mygreen}{rgb}{0,0.6,0}
\definecolor{mygray}{rgb}{0.5,0.5,0.5}
\definecolor{mymauve}{rgb}{0.58,0,0.82}

\date{} % 留空,以让编译时去除日期

%———————————————注意事项—————————————————%

% 1、如果编译显示失败,但没有错误信息,就是 filename.pdf 正在被占用
% 2、在文件夹中的终端使用 Windows > xelatex filename.tex 也可编译

%—————————————华东师范大学———————————————%

% 论文制作时须加页眉,页眉从中文摘要开始至论文末
% 偶数页码内容为:华东师范大学硕士学位论文,奇数页码内容为学位论文题目

%————————定义 \section 的标题样式————————%

% 注意:\chapter 等命令,内部使用的是 \thispagestyle{plain} 的排版格式
% 若需要自己加上页眉,实际是在用 \thispagestyle{fancy} 的排版格式
% 加上下面这一段指令,就能够让 \section 也使用 fancy 的排版格式
% 本质就是让目录、第一页也能够显示页眉、页脚

\fancypagestyle{plain}{
  \pagestyle{fancy}
}

\title{华东师范大学软件学院实验报告} % 模板
\titleformat{\section}
    {\normalfont\bfseries\Large} % 字体大小、字体系列(\bfseries 为加粗)
    {\thesection}{1em}{}

% 设置章节的中文格式
\renewcommand\thesection{\chinese{section} \hspace{0pt}}
\renewcommand\thesubsection{\arabic{subsection} \hspace{0pt}}
% \renewcommand\thesubsubsection{\alph{subsubsection} \hspace{0pt}} % 字母编号
% \hspace{0pt} 是为了确保在章节编号和章节题目之间不要有空格,使得排版更为美观
    
%—————————————页面基础设置———————————————%

\geometry{left=10mm, right=10mm, top=20mm, bottom=20mm}

%————————————设置页眉、页脚——————————————%

\pagestyle{fancy} % 设置 plain style 的属性

% 设置页眉

\fancyhead[RE]{\leftmark} % Right Even 偶数页右侧显示章名 \leftmark 最高级别章名
\fancyhead[LO]{\rightmark} % Left Odd 奇数页左侧显示节名 \rightmark 第二级别节名
\fancyhead[C]{华东师范大学软件学院实验报告} % Center 居中显示
\fancyhead[LE,RO]{~\thepage~} % 在偶数页的左侧,奇数页的右侧显示页码
\renewcommand{\headrulewidth}{1.2pt} % 页眉与正文之间的水平线粗细

% 设置页脚:在每页的右下脚以斜体显示书名

\fancyfoot[RO,RE]{\it Lab Report By \LaTeX} % 使用意大利斜体显示
\renewcommand{\footrulewidth}{0.5pt} % 页脚水平线宽度

% 设置页码:在底部居中显示页码

\pagestyle{fancy}
\fancyfoot[C]{\kaishu 第 \thepage 页 \ 共 \pageref{LastPage} 页} % LastPage 需要二次编译以获取总页数

%——————————————代码块设置———————————————%

\lstset {
    backgroundcolor=\color{white},   % choose the background color; you must add \usepackage{color} or \usepackage{xcolor}
    basicstyle=\footnotesize,        % the size of the fonts that are used for the code
    breakatwhitespace=false,         % sets if automatic breaks should only happen at whitespace
    breaklines=true,                 % sets automatic line breaking
    captionpos=bl,                   % sets the caption-position to bottom
    commentstyle=\color{mygreen},    % comment style
    deletekeywords={...},            % if you want to delete keywords from the given language
    escapeinside={\%*}{*},           % if you want to add LaTeX within your code
    extendedchars=true,              % lets you use non-ASCII characters; for 8-bits encodings only, does not work with UTF-8
    frame=single,                    % adds a frame around the code
    keepspaces=true,                 % keeps spaces in text, useful for keeping indentation of code (possibly needs columns=flexible)
    keywordstyle=\color{blue},       % keyword style
    % language=Python,               % the language of the code
    morekeywords={*,...},            % if you want to add more keywords to the set
    numbers=left,                    % where to put the line-numbers; possible values are (none, left, right)
    numbersep=5pt,                   % how far the line-numbers are from the code
    numberstyle=\tiny\color{mygray}, % the style that is used for the line-numbers
    rulecolor=\color{black},         % if not set, the frame-color may be changed on line-breaks within not-black text (e.g. comments (green here))
    showspaces=false,                % show spaces everywhere adding particular underscores; it overrides 'showstringspaces'
    showstringspaces=false,          % underline spaces within strings only
    showtabs=false,                  % show tabs within strings adding particular underscores
    stepnumber=1,                    % the step between two line-numbers. If it's 1, each line will be numbered
    stringstyle=\color{orange},      % string literal style
    tabsize=2,                       % sets default tabsize to 2 spaces
    % title=Python Code              % show the filename of files included with \lstinputlisting; also try caption instead of title
}

% 注释掉的部分用于后续插入代码,参数可调整,格式如下:

% 1、直接插入
% \begin{lstlisting}[language = ? , title = { ? } ]
%       Your code here.
% \end{lstlisting}

% 2、文件插入
% \lstinputlisting[language = C , title = ?.c] {filename.c}

%———————————————字体设置————————————————%

% \setCJKmainfont{SimSun} % 设置正文罗马族的 CJK 字体
% \renewcommand{\normalsize}{\fontsize{12pt}{15pt}\selectfont} % 设置正文字号
\linespread{1.2}

%——————————————————————————————————————%

%———————————————超链接设置——————————————%

\hypersetup{
    pdfstartview=FitH, % 设置PDF文档打开时的初始视图为页面宽度适应窗口宽度(即页面水平适应)
    CJKbookmarks=true, % 用对CJK(中文、日文、韩文)字符的书签支持,确保这些字符在书签中正确显示
    bookmarksnumbered=true, % 书签带有章节编号。这对有章节编号的文档很有用
    bookmarksopen=true, % 文档打开时,书签树是展开的,方便查看所有书签
    colorlinks, % 启用彩色链接。这样,链接在PDF中会显示为彩色,而不是默认的方框
    pdfborder=001, % 设置PDF文档中链接的边框样式。001 表示链接周围没有边框,仅在单击时显示一个矩形
    linkcolor=blue, % 设置文档内部链接(如目录中的章节链接)的颜色为蓝色
    anchorcolor=blue, % 设置锚点链接(即目标在同一文档内的链接)的颜色为蓝色
    citecolor=blue, % 设置引用(如文献引用)的颜色为蓝色
}

\newcommand{\C}[2]{$ C_{#1}^{#2} $}

%——————————————导言区结束,进入正文部分———————————————%

%——————————————————————————————————————%

\begin{document}

\maketitle

\begin{center} % \extracolsep{\fill} 拉伸到页面最大宽度前,保证居中显示

  \begin{tabular*}{\textwidth}{@{\extracolsep{\fill}} l  l  l }
    \hline
    实验课程:计算机网络 &  年级:2023级本科  &  实验成绩: \\
    实验名称:Homework - 2 & 姓名:张梓卫 \\
    实验编号:(2) & 学号:10235101526 & 实验日期: \\
    指导老师: & 组号: \\
    \hline
  \end{tabular*}

\end{center}

% \tableofcontents % 目录也需要二次编译

\section{2-1}

\subsection{问题与翻译}

A noiseless 8-kHz channel is sampled every 1 msec. What is the maximum data rate?

一个无噪声的8千赫信道每1毫秒采样一次。最大数据传输速率是多少?

\subsection{解答与翻译}

Nyquist’s theorem states that for an ideal low-pass (no noise, finite bandwidth) channel, the maximum symbol transmission rate is 2W symbols per second, where W is the bandwidth of the channel (in Hz). If we use V to represent the number of discrete levels (i.e., how many distinct symbols there are), then the maximum data rate is:

\[
  \text{Maximum data rate in a noiseless channel} = 2W \log_2(V) \text{ b/s}
\]

In this case, the channel bandwidth is 8 kHz (8000 Hz), and if we assume each sample produces 16 bits, then there are \(2^{16}\) possible states. Substituting these values into the formula gives:

\[
C = 2 \times 8000 \times \log_2(2^{16}) = 256,000 \text{ bps}
\]

Thus, the maximum data rate is approximately 256 Kbps.

So, the maximum data rate of the noiseless channel is decided by how many bits there are in every sample.

\vspace{10pt}

奈奎斯特定理规定:在理想低通(没有噪声、带宽有限)信道中,为了避免码间串扰,极限码元传输速率为 2W 波特,其中W是信道的频率带宽(单位为Hz)。若用V表示每个码元的离散电平数目(码元的离散电平数目是指有多少种不同的码元),则极限数据传输速率为:

\[
  \text{理想低通信道下的极限数据传输速率} = 2W \log_2(V) \, \text{比特每秒}
\]

在此例中,信道带宽为8千赫(8000赫兹)。

假设每次采样产生16比特,则每次采样可以有\(2^{16}\)种不同的状态。将这些值代入公式,得到:

\[
C = 2 \times 8000 \times \log_2(2^{16}) = 256,000 \, \text{比特每秒}
\]

因此,无噪声信道的最大数据速率是由每个样本中有多少位决定的。在假设每次采样产生16比特时,得到的最大传输速率为 256 Kbps。

参考资料:\href{https://electronics.stackexchange.com/questions/94822/does-nyquist-rate-depend-on-the-sampling-rate}{StackOverflow}
\section{2-2}

\subsection{问题与翻译}

If a binary signal is sent over a 3-kHz channel whose signal-to-noise ratio is 20 dB, what is the maximum achievable data rate?

如果一个二进制信号通过信噪比为20分贝的3千赫信道传输,最大可实现的数据传输速率是多少?

\subsection{解答与翻译}

Using Shannon’s theorem, we can calculate the maximum achievable data rate for a noisy channel. The formula is:

\[
\text{Maximum Data Rate} = B \times \log_2(1 + SNR)
\]

Where B is the bandwidth of the channel, and SNR is the signal-to-noise ratio (in linear terms). First, we convert 20 dB into a linear value:

\[
\text{SNR (linear)} = 10^{\frac{20}{10}} = 100
\]

Then applying the Shannon formula:

\[
\text{Maximum Data Rate} = 3000 \times \log_2(1 + 100) = 3000 \times \log_2(101) \approx 3000 \times 6.6582 = 19974.6 \text{bps}
\]

So, the maximum achievable data rate is approximately 19.97 kbps.

\vspace{10pt}

使用香农公式,可以计算带噪信道的最大数据传输速率。公式如下:

\[
\text{最大数据速率} = B \times \log_2(1 + SNR)
\]

其中,B 是信道带宽,SNR 是信噪比(线性值)。首先将 20 dB 信噪比转换为线性值:

\[
\text{SNR (线性)} = 10^{\frac{20}{10}} = 100
\]

然后将其代入香农公式:

\[
\text{最大数据速率} = 3000 \times \log_2(1 + 100) = 3000 \times \log_2(101) \approx 3000 \times 6.6582 = 19974.6 \text{比特每秒}
\]

因此,最大可实现的数据传输速率约为 19.97 kbps。

\section{2-3}

\subsection{问题与翻译}

How much bandwidth is there in 0.1 microns of spectrum at a wavelength of 1 micron?

在波长为1微米的频谱中,0.1微米的频谱带宽是多少?

\subsection{解答与翻译}

The bandwidth \(\Delta f\) can be calculated using the formula:

\[
\Delta f = \frac{c \cdot \Delta \lambda}{\lambda^2}
\]

Assuming:
\begin{itemize}
    \item Speed of light \(c = 3 \times 10^8 \, \text{m/s}\),
    \item Wavelength range \(\Delta \lambda = 0.1 \times 10^{-6} \, \text{m}\),
    \item Wavelength \(\lambda = 1 \times 10^{-6} \, \text{m}\).
\end{itemize}

Substituting these values into the formula:

\[
\Delta f = \frac{3 \times 10^8 \, \text{m/s} \times 0.1 \times 10^{-6} \, \text{m}}{(1 \times 10^{-6} \, \text{m})^2} = 3 \times 10^{13} \, \text{Hz}
\]

Thus, the bandwidth is approximately 30 THz.

\vspace{10pt}

频谱带宽 \(\Delta f\) 可以通过以下公式计算:

\[
\Delta f = \frac{c \cdot \Delta \lambda}{\lambda^2}
\]

假设:
\begin{itemize}
    \item 光速 \(c = 3 \times 10^8 \, \text{米/秒}\),
    \item 波长范围 \(\Delta \lambda = 0.1 \times 10^{-6} \, \text{米}\),
    \item 波长 \(\lambda = 1 \times 10^{-6} \, \text{米}\)。
\end{itemize}

将这些值代入公式:

\[
\Delta f = \frac{3 \times 10^8 \, \text{米/秒} \times 0.1 \times 10^{-6} \, \text{米}}{(1 \times 10^{-6} \, \text{米})^2} = 3 \times 10^{13} \, \text{赫兹}
\]

因此,带宽约为 30 太赫兹(THz)。

\section{2-4}

\subsection{问题与翻译}

It is desired to send a sequence of computer screen images over an optical fiber. The screen is 1920x1200 pixels, each pixel being 24 bits. There are 50 screen images per second. How much bandwidth is needed?

希望通过光纤传送一系列计算机屏幕图像。屏幕分辨率为1920x1200像素,每个像素为24比特。每秒传输50幅图像。所需的带宽是多少?

\subsection{解答与翻译}

Each image contains:

\[
1920 \times 1200 \times 24 = 55296000 \text{ bits/image}
\]

With 50 images transmitted per second, the data rate is:

\[
55296000 \times 50 = 2764800000 \text{ bps} = 2.7648 \text{ Gbps}
\]

Thus, the bandwidth required is 2.7648 Gbps.

\vspace{10pt}

每张图像的比特数为:

\[
1920 \times 1200 \times 24 = 55296000 \text{ 比特/图像}
\]

每秒50张图像的传输速率为:

\[
55296000 \times 50 = 2764800000 \text{ 比特/秒} = 2.7648 \text{ 千兆比特每秒}
\]

因此,所需的带宽为 2.7648 Gbps。

\section{2-5}

\subsection{问题与翻译}

Radio antennas often work best when the diameter of the antenna is equal to the wavelength of the radio wave. Reasonable antennas range from 1 cm to 5 meters in diameter. What frequency range does this cover?

无线电天线的最佳工作状态往往是天线的直径等于无线电波的波长。合理的天线直径范围为1厘米到5米。这涵盖了什么频率范围?

\subsection{解答与翻译}

The frequency is related to the wavelength by the equation:

\[
f = \frac{c}{\lambda}
\]

For a wavelength of 1 cm (0.01 m):

\[
f = \frac{3 \times 10^8}{0.01} = 3 \times 10^{10} \text{ Hz} = 30 \text{ GHz}
\]

For a wavelength of 5 meters:

\[
f = \frac{3 \times 10^8}{5} = 6 \times 10^7 \text{ Hz} = 60 \text{ MHz}
\]

Thus, the frequency range is from 60 MHz to 30 GHz.

\vspace{10pt}

频率与波长的关系为:

\[
f = \frac{c}{\lambda}
\]

对于波长1厘米(0.01米):

\[
f = \frac{3 \times 10^8}{0.01} = 3 \times 10^{10} \text{ 赫兹} = 30 \text{ 千兆赫兹}
\]

对于波长5米:

\[
f = \frac{3 \times 10^8}{5} = 6 \times 10^7 \text{ 赫兹} = 60 \text{ 兆赫兹}
\]

因此,频率范围为60 MHz 至 30 GHz。

\section{2-6}

\subsection{问题与翻译}

Ten signals, each requiring 4000 Hz, are multiplexed onto a single channel using FDM. What is the minimum bandwidth required for the multiplexed channel? Assume that the guard bands are 400 Hz wide.

十个信号,每个需要4000赫兹,通过频分复用(FDM)复用到单一信道。复用信道所需的最小带宽是多少?假设保护带宽为400赫兹。

\subsection{解答与翻译}

Each signal requires 4000 Hz, and there are 10 signals. In addition, 400 Hz of guard band is needed between adjacent signals. There are 9 guard bands because only 9 are required to separate 10 signals.

The total bandwidth required is the sum of the bandwidth of the 10 signals and the 9 guard bands:

\[
\text{Total Bandwidth} = (10 \times 4000) + (9 \times 400) = 40000 + 3600 = 43600 \text{ Hz}
\]

Thus, the minimum bandwidth required for the multiplexed channel is 43.6 kHz.

\vspace{10pt}

每个信号需要4000赫兹,有10个信号。相邻信号之间的保护带宽为400赫兹。

一共有 10 个信号,因此需要为这 10 个信号分配带宽。
但是信号本身占用带宽之外,信号之间还需要设置保护带宽。
保护带宽的个数是 9 个,因为 10 个信号只需要 9 段保护带宽(类似于 10 个房间只需要 9 面墙来分隔)。

总带宽由两部分组成:

10 个信号,每个信号需要 4000 Hz,所以信号的总带宽是:
\[
10 \times 4000 = 40000 \, \text{Hz}
\]

9 段保护带宽,每段为 400 Hz,所以保护带宽的总宽度是:
\[
9 \times 400 = 3600 \, \text{Hz}
\]

将信号的带宽和保护带宽相加,得到所需的总带宽:
\[
\text{Total Bandwidth} = 40000 \, \text{Hz} + 3600 \, \text{Hz} = 43600 \, \text{Hz}
\]

即:43.6 kHz。


\section{2-7}

\subsection{问题与翻译}

Why has the PCM sampling time been set at 125 μsec?

为什么PCM采样时间被设置为125微秒?

\subsection{解答与翻译}

PCM (Pulse Code Modulation) is typically used in telephone communication, where the standard voice communication bandwidth is about 4 kHz. According to the Nyquist theorem, the sampling rate must be at least twice the bandwidth, which means sampling at a rate of 8000 samples per second. The sampling interval is:

\[
\text{Sampling Interval} = \frac{1}{8000} = 125 \, \text{microseconds}
\]

Thus, the PCM sampling time is set at 125 microseconds to meet the Nyquist sampling requirement for voice communication.

\vspace{10pt}

PCM(脉冲编码调制)通常用于电话通信中,标准语音通信带宽约为4 kHz。根据奈奎斯特定理,采样率必须至少是带宽的两倍,即每秒8000次采样。采样间隔为:

\[
Sampling Interval = \frac{1}{8000} = 125 \text{ 微秒}
\]

所以显然,PCM采样时间设置为125微秒,数据速率不会超过8KHz,因此没有必要更频繁地采样。
以满足语音通信的奈奎斯特采样要求。

\section{2-8}

\subsection{问题与翻译}

Compare the maximum data rate of a noiseless 4-kHz channel using
(a) Analog encoding (e.g., QPSK) with 2 bits per sample.
(b) The T1 PCM system.

比较无噪声的4千赫信道在以下两种情况下的最大数据速率:

(a) 模拟编码(例如QPSK)使用每个采样2比特。

(b) T1 PCM系统。

\subsection{解答与翻译}

For (a), using QPSK (Quadrature Phase Shift Keying), each symbol represents 2 bits, and there are 4 distinct phase shifts. The Nyquist formula for maximum data rate is:

\[
C = 2B \log_2 M
\]

For a 4-kHz bandwidth, and \(M = 4\) (since each symbol encodes 2 bits), the maximum data rate is:

\[
C = 2 \times 4000 \times \log_2 4 = 16,000 \text{ bps}
\]

For (b), in the T1 PCM system, the sampling rate is twice the bandwidth, and each sample is encoded using 8 bits. Thus, the maximum data rate is:

\[
C = 8000 \times 8 = 64,000 \text{ bps}
\]

Thus, the maximum data rate for QPSK is 16 kbps, and for the T1 PCM system, it is 64 kbps.

\vspace{10pt}

QPSK(正交相移键控)是一种相位调制技术,通过改变参考信号(载波)的相位来传输数据。在 QPSK 中,每个符号代表 2 个比特,因此有 4 种不同的相位变化(因为 \(2^2 = 4\))。

对于无噪声信道,Nyquist 最大数据速率公式为:

\[
C = 2B \log_2 M
\]

\[
\begin{aligned}
C &= 2 \times 4000 \times \log_2 4 \\
  &= 2 \times 4000 \times 2 \quad (\text{因为} \log_2 4 = 2) \\
  &= 16,000 \text{ 比特每秒(bps)}
\end{aligned}
\]

最大数据速率:16,000 bps(或 16 kbps)


T1 PCM(脉冲编码调制)系统是一种数字传输方法,常用于电信。它通过对模拟信号进行采样并将其转换为数字比特流。T1 系统以信号最高频率的两倍进行采样(根据 Nyquist 定理),并用 8 位来表示每个采样。

根据 Nyquist 定理,最低采样率 (\(f_s\)) 必须至少是信号最大频率 (\(f_{\text{max}}\)) 的两倍,以避免混叠:

\[
f_s = 2f_{\text{max}}
\]

\[
\text{数据速率} = \text{采样率} \times \text{每个样本的比特数}
\]
\[
= 8000 \text{ 样本/秒} \times 8 \text{ 比特/样本} = 64,000 \text{ 比特每秒(bps)}
\]

最大数据速率:64,000 bps(或 64 kbps)

\section{2-9}

\subsection{问题与翻译}

A CDMA receiver gets the following chips: (-1 +1 -3 +1 -1 -3 +1 +1). Assuming the chip sequences defined in Fig. 2-28(a), which stations transmitted, and which bits did each one send?

一个CDMA接收机接收到如下码片序列:(-1 +1 -3 +1 -1 -3 +1 +1)。假设码片序列如图2-28(a)所示,哪些站点发送了数据,每个站点发送了哪些比特?

\subsection{解答与翻译}

To determine which stations transmitted, we correlate the received chips with each station's chip sequence using the dot product. For each station, we compute the dot product of the received chips with the station's chip sequence. A positive result indicates a transmitted bit of 1, while a negative result indicates a transmitted bit of 0.

For example:
\[
S.A = (-1 +1 -3 +1 -1 -3 +1 +1) \cdot (+1 +1 +1 -1 -1 +1 -1 -1) = 1
\]
Thus, station A transmitted a bit of 1. Repeating this process for the other stations gives:

\[
S.B = (-1 +1 -3 +1 -1 -3 +1 +1) \cdot (+1 +1 -1 +1 -1 -1 +1 -1) = -1
\]
Thus, station B transmitted a bit of 0.

\[
S.C = (-1 +1 -3 +1 -1 -3 +1 +1) \cdot (+1 -1 +1 -1 +1 +1 -1 -1) = 0
\]
Thus, station C did not transmit.

So, the transmitted bits are: A: 1, B: 0, C: no transmission.

\vspace{10pt}

通过将接收到的码片序列与每个站点的码片序列进行点积运算,可以确定哪些站点发送了数据。正值表示发送了比特1,负值表示发送了比特0。

根据题目,每个站点在发送 0 比特时对应的码片序列为:
\begin{itemize}
    \item A 发送 0: \(+1 +1 +1 -1 -1 +1 -1 -1\)
    \item B 发送 0: \(+1 +1 -1 +1 -1 -1 +1 -1\)
    \item C 发送 0: \(+1 -1 +1 -1 +1 +1 -1 -1\)
    \item D 不发送: \(+0 +0 +0 +0 +0 +0 +0 +0\)
\end{itemize}

总的信号为各站点传输的信号叠加,接收端的信号为:
\[
(+3 +1 +1 -1 -3 -1 +1 +1)
\]
这表示A、B、C、D发送信号的组合。

通过计算接收到的信号和每个站点码片序列的内积,可以解码出各站点传输的比特。

\[
S.A = ( +3 +1 +1 -1 -3 -1 +1 +1 ) \cdot ( +1 +1 +1 -1 -1 +1 -1 -1 )
\]
计算得到:
\[
S.A = (1+1+1+1+3+1+1+1)/8 = 1
\]
因此,A 发送了比特 \textbf{1}。

\[
S.B = ( +3 +1 +1 -1 -3 -1 +1 +1 ) \cdot ( +1 +1 -1 +1 -1 -1 +1 -1 )
\]
计算得到:
\[
S.B = (1+1-1-1+3+1+1-1)/8 = -1
\]
因此,B 发送了比特 \textbf{0}。

\[
S.C = ( +3 +1 +1 -1 -3 -1 +1 +1 ) \cdot ( +1 -1 +1 -1 +1 +1 -1 -1 )
\]
计算得到:
\[
S.C = (1-1+1+1-3-1-1+1)/8 = 0
\]
因此,C \textbf{没有传输}。

\[
S.D = ( +3 +1 +1 -1 -3 -1 +1 +1 ) \cdot ( 0 +0 +0 +0 +0 +0 +0 +0 )
\]
计算得到:
\[
S.D = 1
\]
因此,D 发送了比特 \textbf{1}。

\section{2-10}

\subsection{问题与翻译}

A cable company decides to provide Internet access over cable in a neighborhood consisting of 5000 houses. The company uses a coaxial cable and spectrum allocation allowing 100 Mbps downstream bandwidth per cable. To attract customers, the company decides to guarantee at least 2 Mbps downstream bandwidth to each house at any time. Describe what the cable company needs to do to provide this guarantee.

一家有线公司决定在包含5000户家庭的社区提供有线互联网接入。公司使用同轴电缆和频谱分配,每条电缆允许100 Mbps的下行带宽。为了吸引客户,公司决定保证每户家庭在任何时候至少有2 Mbps的下行带宽。描述公司需要采取哪些措施来提供这一保证。

\subsection{解答与翻译}

To guarantee 2 Mbps per household, the total downstream bandwidth required is:

\[
5000 \times 2 \text{ Mbps} = 10000 \text{ Mbps} = 10 \text{ Gbps}
\]

Since each cable provides 100 Mbps, the company will need:

\[
\frac{10000 \text{ Mbps}}{100 \text{ Mbps/cable}} = 100 \text{ cables}
\]

The company must divide the neighborhood into smaller groups, with each group served by one cable. To guarantee 2 Mbps per house, each group should consist of no more than:

\[
\frac{100 \text{ Mbps}}{2 \text{ Mbps/house}} = 50 \text{ houses}
\]

Thus, the company needs to install 100 separate cables or split the network accordingly to provide sufficient bandwidth.

\vspace{10pt}

为了保证5000户每户都有2 Mbps的带宽,总所需的下行带宽为:

\[
5000 \times 2 \text{ Mbps} = 10000 \text{ Mbps} = 10 \text{ Gbps}
\]

由于每条电缆提供100 Mbps,公司的需求为:

\[
\frac{10000 \text{ Mbps}}{100 \text{ Mbps/cable}} = 100 \text{ cables}
\]

公司需要将社区划分为更小的组,每组由一条电缆提供服务。为了保证2 Mbps的带宽,每组的家庭数量不应超过:

\[
\frac{100 \text{ Mbps}}{2 \text{ Mbps/house}} = 50 \text{ households}
\]

公司应安装100条独立电缆或分割网络,以为每户提供足够的带宽。同时,
电缆公司需将每根电缆直接连接到光纤节点。

\end{document}