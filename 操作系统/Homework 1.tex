\documentclass[UTF8]{homework.cls}
% \date{May 14, 2024}
% \author{Deralive / Shichien}
% \title{华东师范大学软件学院实验报告模板}
% 注意事项:编译两次,以确保目录、页码完整显示
% 模板文件:https://github.com/Shichien/ECNU-LateX-Template

\def\allfiles{}

% \documentclass[14pt,a4paper,UTF8,twoside]{article}

\usepackage{amsmath}
\usepackage{graphicx}
\usepackage{longtable}
% \usepackage{geometry} 
\usepackage{ctex}
\usepackage{booktabs} % 表格库
\usepackage{titlesec} % 标题库
\usepackage{fancyhdr} % 页眉页脚库
\usepackage{lastpage} % 页码数库
\usepackage{listings} % 代码块包
\usepackage{xcolor}
\usepackage[hidelinks]{hyperref}
\usepackage{tikz}
\usepackage{tikz-qtree}
\usepackage{fontspec} % 允许设置字体
\usepackage{unicode-math} % 允许数学公式使用特定字体
\usepackage{mwe}
\usepackage{zhlipsum} % 中文乱数文本
\usepackage{amsmath}
\usepackage{xcolor}
\usepackage{float} % 浮动体环境
\usepackage{subcaption} % 子图包
\usepackage{forest}

\definecolor{mygreen}{rgb}{0,0.6,0}
\definecolor{mygray}{rgb}{0.5,0.5,0.5}
\definecolor{mymauve}{rgb}{0.58,0,0.82}

\date{} % 留空,以让编译时去除日期

%———————————————注意事项—————————————————%

% 1、如果编译显示失败,但没有错误信息,就是 filename.pdf 正在被占用
% 2、在文件夹中的终端使用 Windows > xelatex filename.tex 也可编译

%—————————————华东师范大学———————————————%

% 论文制作时须加页眉,页眉从中文摘要开始至论文末
% 偶数页码内容为:华东师范大学硕士学位论文,奇数页码内容为学位论文题目

%————————定义 \section 的标题样式————————%

% 注意:\chapter 等命令,内部使用的是 \thispagestyle{plain} 的排版格式
% 若需要自己加上页眉,实际是在用 \thispagestyle{fancy} 的排版格式
% 加上下面这一段指令,就能够让 \section 也使用 fancy 的排版格式
% 本质就是让目录、第一页也能够显示页眉、页脚

\fancypagestyle{plain}{
  \pagestyle{fancy}
}

\pagestyle{fancy} % 设置 plain style 的属性


\definecolor{mygreen}{rgb}{0,0.6,0}
\definecolor{mygray}{rgb}{0.5,0.5,0.5}
\definecolor{mymauve}{rgb}{0.58,0,0.82}

\newcommand{\hmwkTitle}{Homework\ \#1}
\newcommand{\hmwkDueDate}{October 04, 2021}
\newcommand{\hmwkClass}{Course Name}
\newcommand{\hmwkAuthorName}{Author Name}

\newcommand{\hmwkTitleCN}{作业\#1:\\操作系统的发展与十年后的操作系统}
\newcommand{\hmwkDueDateCN}{2024年09月22日}
\newcommand{\hmwkAuthorNameCN}{张梓卫}
\newcommand{\hmwkAuthorIDCN}{10235101526}

\if\hmwkCoverPage 1
    \title{
        \vspace{3in}
        \if\hmwkLanguage E
            \huge{\textbf{\hmwkClass}}\\
            \vspace{0.4in}
            \huge{\textbf{\hmwkTitle}}\\
            \vspace{0.4in}
            \normalsize{Due on \hmwkDueDate}\\
        \else
            \Huge{\textbf{\hmwkClassCN}}\\
            \vspace{0.3in}
            \Huge{\textbf{\hmwkTitleCN}}\\
            \vspace{0.4in}
            \normalsize{截止日期:\hmwkDueDateCN}\\
        \fi
        \vspace{2.4in}
    }

    \if\hmwkLanguage E
        \author{\hmwkAuthorName}
    \else
        \author{\hmwkAuthorNameCN (学号:\hmwkAuthorIDCN)}
    \fi
    \date{}
\else
    \title{
        \if\hmwkLanguage E
            \huge{\textbf{\hmwkClass}}\\
            \vspace{0.05in}
            \huge{\textbf{\hmwkTitle}}\\
            \vspace{0.2in}
            \normalsize{Due on \hmwkDueDate}\\
        \else
            \Huge{\textbf{\hmwkClassCN}}\\
            \vspace{0.05in}
            \Huge{\textbf{\hmwkTitleCN}}\\
            \vspace{0.2in}
            \normalsize{截止日期:\hmwkDueDateCN}\\
        \fi
    }

    \if\hmwkLanguage E
        \author{\hmwkAuthorName}
    \else
        \author{\hmwkAuthorNameCN (学号:\hmwkAuthorIDCN)}
    \fi
    \date{}
\fi

% 设置页脚:在每页的右下脚以斜体显示书名

\fancyfoot[RO,RE]{\it Lab Report By \LaTeX} % 使用意大利斜体显示
\renewcommand{\footrulewidth}{0.5pt} % 页脚水平线宽度

% 设置页码:在底部居中显示页码

\pagestyle{fancy}
\fancyfoot[C]{\kaishu 第 \thepage 页 \ 共 \pageref{LastPage} 页} % LastPage 需要二次编译以获取总页数

%——————————————代码块设置———————————————%

\lstset {
    backgroundcolor=\color{white},   % choose the background color; you must add \usepackage{color} or \usepackage{xcolor}
    basicstyle=\footnotesize,        % the size of the fonts that are used for the code
    breakatwhitespace=false,         % sets if automatic breaks should only happen at whitespace
    breaklines=true,                 % sets automatic line breaking
    captionpos=bl,                   % sets the caption-position to bottom
    commentstyle=\color{mygreen},    % comment style
    deletekeywords={...},            % if you want to delete keywords from the given language
    escapeinside={\%*}{*},           % if you want to add LaTeX within your code
    extendedchars=true,              % lets you use non-ASCII characters; for 8-bits encodings only, does not work with UTF-8
    frame=single,                    % adds a frame around the code
    keepspaces=true,                 % keeps spaces in text, useful for keeping indentation of code (possibly needs columns=flexible)
    keywordstyle=\color{blue},       % keyword style
    % language=Python,               % the language of the code
    morekeywords={*,...},            % if you want to add more keywords to the set
    numbers=left,                    % where to put the line-numbers; possible values are (none, left, right)
    numbersep=5pt,                   % how far the line-numbers are from the code
    numberstyle=\tiny\color{mygray}, % the style that is used for the line-numbers
    rulecolor=\color{black},         % if not set, the frame-color may be changed on line-breaks within not-black text (e.g. comments (green here))
    showspaces=false,                % show spaces everywhere adding particular underscores; it overrides 'showstringspaces'
    showstringspaces=false,          % underline spaces within strings only
    showtabs=false,                  % show tabs within strings adding particular underscores
    stepnumber=1,                    % the step between two line-numbers. If it's 1, each line will be numbered
    stringstyle=\color{orange},      % string literal style
    tabsize=2,                       % sets default tabsize to 2 spaces
    % title=Python Code              % show the filename of files included with \lstinputlisting; also try caption instead of title
}

\linespread{1.2}

%———————————————超链接设置——————————————%

\hypersetup{
    pdfstartview=FitH, % 设置PDF文档打开时的初始视图为页面宽度适应窗口宽度(即页面水平适应)
    CJKbookmarks=true, % 用对CJK(中文、日文、韩文)字符的书签支持,确保这些字符在书签中正确显示
    bookmarksnumbered=true, % 书签带有章节编号。这对有章节编号的文档很有用
    bookmarksopen=true, % 文档打开时,书签树是展开的,方便查看所有书签
    colorlinks, % 启用彩色链接。这样,链接在PDF中会显示为彩色,而不是默认的方框
    pdfborder=001, % 设置PDF文档中链接的边框样式。001 表示链接周围没有边框,仅在单击时显示一个矩形
    linkcolor=blue, % 设置文档内部链接(如目录中的章节链接)的颜色为蓝色
    anchorcolor=blue, % 设置锚点链接(即目标在同一文档内的链接)的颜色为蓝色
    citecolor=blue, % 设置引用(如文献引用)的颜色为蓝色
}

% 

\begin{document}

\maketitle
\if\hmwkCoverPage 1
    \pagebreak
\fi


%
% first problem, id is automatically set to 1
%
\begin{homeworkProblem}
\textbf{    请简述操作系统的发展史和预测10年后操作系统的样子。需要包含如下要点:}
    \begin{enumerate}
    	\item 列出3个操作系统发展历史上你认为最重要的事件,并给出理由(1000字)
    	\item 列出10年后操作系统3个可能的特征,给给出理由。(1000字)。
    \end{enumerate}
    
\solution

\section{操作系统发展史}

查阅资料,如下所示:

\begin{itemize}
    \item \href{https://www.geeksforgeeks.org/evolution-of-operating-system/}{\underline{https://www.geeksforgeeks.org/evolution-of-operating-system/}}
    \item \href{https://en.wikipedia.org/wiki/History\_of\_operating\_systems}{\underline{https://en.wikipedia.org/wiki/History\_of\_operating\_systems}}
\end{itemize}

\subsection{操作系统类型}

操作系统在过去几年中不断发展。它在得到最初的形式之前经历了几次变化。
操作系统中的这些变化被称为操作系统的进化。随着新技术的发明,操作系统不断改进。

如参考资料所示,操作系统的历史发展可大致划分为如下的阶段:

\begin{itemize}
    \item 无操作系统(0年代至40年代)
    \item 批处理系统(20世纪40年代至50年代)
    \item 多程序设计系统(1950年代至1960年代)
    \item 分时系统(20世纪60年代至70年代)
    \item GUI介绍(20世纪70年代至80年代)
    \item 网络系统(20世纪80年代至90年代)
    \item 移动操作系统(20世纪90年代末至21世纪初)
    \item 人工智能集成(2010年代至今)
\end{itemize}

在20世纪40年代之前,没有操作系统。因此早期的我们不得不用机器语言手动键入每个任务的指令。
当时,用户很难实现一个简单的任务。而且它非常耗时,也不方便用户使用。

\vspace{\baselineskip}

随着时间的推移,批处理系统进入市场。用户可以编写程序并将其加载给计算机操作员。
然后操作员制作不同批次的类似类型的任务,然后将不同批次(任务组)逐一提供给CPU。
CPU首先执行一批任务,然后以顺序方式跳到另一批任务。

\vspace{\baselineskip}

之后出现的多程序设计是第一个真正开始革命的操作系统。
它使用户能够将多个程序加载到内存中,并为每个程序提供特定的内存部分。
当一个程序正在等待任何I/O操作(需要很长时间)时,
操作系统允许CPU从之前的程序切换到另一个程序(在就绪队列中的第一个),以中断程序的连续执行。

\vspace{\baselineskip}

分时系统是多道程序设计系统的扩展版本。
随着时间的推移,图形用户界面(GUI)应运而生。
第一次操作系统变得更加用户友好,改变了我们与计算机交互的方式。

\vspace{\baselineskip}

20世纪80年代,计算机网络的热潮达到了顶峰。管理网络通信所需的一种特殊类型的操作系统。
Novell NetWare和Windows NT等操作系统是为了管理网络通信而开发的,
它为用户在协作环境中工作提供了便利。

\vspace{\baselineskip}

智能手机的发明在软件行业引发了一场巨大的革命,为了处理智能手机的操作,
开发了一种特殊类型的操作系统。其中我们熟知的就是iOS和Android等。
随着AI的跃进,操作系统集成了Siri、谷歌助手和Alexa等人工智能技术的功能,
在很多方面变得更加强大和高效。

\vspace{\baselineskip}

说起操作系统发展历史上我认为最重要的事件,肯定得从Unix的操作系统诞生说起,再说到如今智能手机的普及使用,操作系统深入千家万户。
我所认为的三件最重要的事件分别是:Unix 操作系统的诞生,Linux 内核的发布,以及移动操作系统的兴起。

\vspace{\baselineskip}

下面,让我逐一向各位娓娓道来。

\subsubsection{Unix 操作系统的诞生}

根据百科介绍,UNIX 操作系统由肯·汤普逊(Ken Thompson)和丹尼斯·里奇(Dennis Ritchie)在贝尔实验室开发,是操作系统史上的一个里程碑。
它具有多用户、多任务的设计理念:UNIX 是第一个真正支持多用户、多任务的操作系统,为现代操作系统奠定了基础。

它具有高度的可移植性(采用 C 语言编写),UNIX 可以方便地移植到不同硬件平台上,这在当时是革命性的。
并且遵循"一切皆文件"的设计美学,提供了简单而强大的工具集。这一设计思想影响了后来的众多操作系统,如 Linux、BSD、macOS 等,成为操作系统发展的基石。

随着 AT\&T 允许 Unix 在学术机构中使用,Unix 系统成为各大高校计算机教育的标准。
如同我们一样的学生,在学术环境中接触并学习 Unix,使其成为未来许多计算机科学家和程序员的工具选择。

\subsubsection{Linux 内核的发布(1991年)}

开源操作系统的新篇章,要从林纳斯·托瓦兹(Linus Torvalds)在 1991 年发布了 Linux 内核说起。
Linux 的源代码对公众开放,鼓励全球开发者共同参与,可以视作开源社区的重大里程碑。
从服务器、桌面到嵌入式设备,Linux 被广泛应用,成为互联网和云计算的基础。
而同时,由于其开放性,Linux 成为了许多新技术的试验平台,如Docker,

Linux 最早主要用于学术研究和爱好者开发环境中,
但随着社区的不断壮大和企业的参与,它逐渐成为了服务器操作系统的首选。
Apache HTTP 服务器的兴起进一步推动了 Linux 在互联网服务中的广泛应用。
而在 2000 年代,随着 Android 的推出,Linux 更是进入了数十亿台移动设备。
如今,Linux 还在数据中心、超级计算机、物联网设备中得到广泛应用,成为全球互联网和云计算的基石。

\subsubsection{移动操作系统的兴起(2007年-至今)}

2007 年,也就是我这个年代的学生们最幸福的童年里,我们接触到了诺基亚的屏幕,此时苹果公司发布了搭载 iOS 的第一代 iPhone,随后谷歌在 2008 年发布了 Android 操作系统,标志着移动操作系统时代的来临。

智能手机的普及,改变了我们的生活方式。移动操作系统成为我们日常生活中不可或缺的一部分,
影响了通讯、娱乐、购物等各个方面。移动应用商店的出现,推动了应用生态系统的发展,催生了大量的移动应用和开发者,形成了新的商业模式和产业链。

随着逐渐普及的智能手机,人手一部的移动操作系统,在全球范围内的广泛应用,加速了信息的传播和共享,对社会产生了深远影响。
随着 5G 技术的普及和移动计算能力的提升,
移动操作系统正在从单纯的通讯和娱乐设备扩展到更多领域,
如智能家居、物联网、增强现实和虚拟现实等。(如本人在创客实践课程中所作的智能家居项目)
同时,AI 的集成也将进一步增强移动设备的智能化,未来的移动操作系统有望更加个性化和自主化。

\end{homeworkProblem}

\begin{homeworkProblem}

\section{十年后操作系统的三个可能特征}

\subsection{深度融合人工智能的智能操作系统}

根据目前OpenAI的强劲势头,如O1大模型的博士级别的问答程度,可以显而易见他们的野心绝不会只用来做简单的Token对答。
根据微软在新的联想拯救者2024-Y9000P上添加的Copilot按键,
可以大胆预测未来的操作系统将深度融合人工智能技术,提供更智能、更个性化的用户体验。

操作系统将能够自适应用户界面,能够根据用户的使用习惯和偏好,
自动调整界面布局和功能,自动通过相关的(低功耗)算法优化排面、布局等等。

不仅如此,通过 AI 算法,操作系统可以优化 CPU、内存等资源的分配,
利用机器学习模型,操作系统可以实时检测并预防安全威胁,增强系统的整体安全性。

\subsection{基于云计算和边缘计算的分布式操作系统}

如同阿里云提供的Aliyun-Linux一样,你甚至可以在云端使用PentOS、CentOS等操作系统。
云计算和边缘计算的快速发展,未来的操作系统将可以采用分布式架构,\cite{satyanarayanan2017emergence}
实现更高效的资源利用和计算任务的优化分配。
阿里云的OSS云存储和本地的交互功能,也可能使得操作系统能够在本地设备和云端之间无缝协作,
根据需求动态调整计算任务的位置。

为了满足低延迟和实时性的需求,操作系统将支持在边缘节点进行数据处理和计算。

另一方面,通过虚拟化技术,操作系统可以更灵活地管理和分配计算资源,提高系统的可扩展性。
如Docker这样的被称为改变了程序员工作方式的技术,大大地提高了效率以及降低了很多环境配置的繁杂
要求。

\subsection{3D交互和虚拟现实的VR操作系统}

随着 VR 技术逐渐成熟,未来的操作系统也许将会融入这些技术,为用户提供沉浸式的交互体验。
例如横空出世的Vision Pro,高清、高帧率的现实世界3D渲染,让人们可以体验到无感触摸的空中点按交互。

不仅如此,三维操作界面的操作系统将支持手势、语音等多种交互方式,从纯视觉的物理交互会引申出更多的体验。
当然,正因如此,这样的操作系统的出现会催生出更小、更快的GPU、处理器等等,以适应其应用。

当这些技术真正成型时,很可能会逼迫某些技术发展,例如当年安卓系统出世时,很多现象级电脑应用纷纷推出手机版,
显然,当3D交互技术真正成熟、可以走进千家万户时,各种应用的3D开发也会普及,未来将带给用户更深层次的体验。

\end{homeworkProblem}

\bibliographystyle{abbrv}
\bibliography{references.bib}

\end{document}