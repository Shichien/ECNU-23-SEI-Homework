\documentclass[UTF8]{homework}
% \date{May 14, 2024}
% \author{Deralive / Shichien}
% \title{华东师范大学软件学院实验报告模板}
% 注意事项:编译两次,以确保目录、页码完整显示
% 模板文件:https://github.com/Shichien/ECNU-LateX-Template

\def\allfiles{}

% \documentclass[14pt,a4paper,UTF8,twoside]{article}

\usepackage{amsmath}
\usepackage{graphicx}
\usepackage{longtable}
% \usepackage{geometry} 
\usepackage{ctex}
\usepackage{booktabs} % 表格库
\usepackage{titlesec} % 标题库
\usepackage{fancyhdr} % 页眉页脚库
\usepackage{lastpage} % 页码数库
\usepackage{listings} % 代码块包
\usepackage{xcolor}
\usepackage[hidelinks]{hyperref}
\usepackage{tikz}
\usepackage{tikz-qtree}
\usepackage{fontspec} % 允许设置字体
\usepackage{unicode-math} % 允许数学公式使用特定字体
\usepackage{mwe}
\usepackage{zhlipsum} % 中文乱数文本
\usepackage{amsmath}
\usepackage{xcolor}
\usepackage{float} % 浮动体环境
\usepackage{subcaption} % 子图包
\usepackage{forest}
\usepackage{tikz}
\usepackage{pgfgantt}



\definecolor{mygreen}{rgb}{0,0.6,0}
\definecolor{mygray}{rgb}{0.5,0.5,0.5}
\definecolor{mymauve}{rgb}{0.58,0,0.82}

\date{} % 留空,以让编译时去除日期

%———————————————注意事项—————————————————%

% 1、如果编译显示失败,但没有错误信息,就是 filename.pdf 正在被占用
% 2、在文件夹中的终端使用 Windows > xelatex filename.tex 也可编译

%—————————————华东师范大学———————————————%

% 论文制作时须加页眉,页眉从中文摘要开始至论文末
% 偶数页码内容为:华东师范大学硕士学位论文,奇数页码内容为学位论文题目

%————————定义 \section 的标题样式————————%

% 注意:\chapter 等命令,内部使用的是 \thispagestyle{plain} 的排版格式
% 若需要自己加上页眉,实际是在用 \thispagestyle{fancy} 的排版格式
% 加上下面这一段指令,就能够让 \section 也使用 fancy 的排版格式
% 本质就是让目录、第一页也能够显示页眉、页脚

\fancypagestyle{plain}{
  \pagestyle{fancy}
}

\pagestyle{fancy} % 设置 plain style 的属性


\definecolor{mygreen}{rgb}{0,0.6,0}
\definecolor{mygray}{rgb}{0.5,0.5,0.5}
\definecolor{mymauve}{rgb}{0.58,0,0.82}

\newcommand{\hmwkTitle}{Homework\ \#1}
\newcommand{\hmwkDueDate}{October 04, 2021}
\newcommand{\hmwkClass}{Course Name}
\newcommand{\hmwkAuthorName}{Author Name}

\newcommand{\hmwkTitleCN}{作业\#3:\\ 241118 进程调度}
\newcommand{\hmwkDueDateCN}{2024年11月27日}
\newcommand{\hmwkAuthorNameCN}{张梓卫}
\newcommand{\hmwkAuthorIDCN}{10235101526}

\if\hmwkCoverPage 1
    \title{
        \vspace{3in}
        \if\hmwkLanguage E
            \huge{\textbf{\hmwkClass}}\\
            \vspace{0.4in}
            \huge{\textbf{\hmwkTitle}}\\
            \vspace{0.4in}
            \normalsize{Due on \hmwkDueDate}\\
        \else
            \Huge{\textbf{\hmwkClassCN}}\\
            \vspace{0.3in}
            \Huge{\textbf{\hmwkTitleCN}}\\
            \vspace{0.4in}
            \normalsize{截止日期:\hmwkDueDateCN}\\
        \fi
        \vspace{2.4in}
    }

    \if\hmwkLanguage E
        \author{\hmwkAuthorName}
    \else
        \author{\hmwkAuthorNameCN (学号:\hmwkAuthorIDCN)}
    \fi
    \date{}
\else
    \title{
        \if\hmwkLanguage E
            \huge{\textbf{\hmwkClass}}\\
            \vspace{0.05in}
            \huge{\textbf{\hmwkTitle}}\\
            \vspace{0.2in}
            \normalsize{Due on \hmwkDueDate}\\
        \else
            \Huge{\textbf{\hmwkClassCN}}\\
            \vspace{0.05in}
            \Huge{\textbf{\hmwkTitleCN}}\\
            \vspace{0.2in}
            \normalsize{截止日期:\hmwkDueDateCN}\\
        \fi
    }

    \if\hmwkLanguage E
        \author{\hmwkAuthorName}
    \else
        \author{\hmwkAuthorNameCN (学号:\hmwkAuthorIDCN)}
    \fi
    \date{}
\fi

% 设置页脚:在每页的右下脚以斜体显示书名

\fancyfoot[RO,RE]{\it Lab Report By \LaTeX} % 使用意大利斜体显示
\renewcommand{\footrulewidth}{0.5pt} % 页脚水平线宽度

% 设置页码:在底部居中显示页码

\pagestyle{fancy}
\fancyfoot[C]{\kaishu 第 \thepage 页 \ 共 \pageref{LastPage} 页} % LastPage 需要二次编译以获取总页数

%——————————————代码块设置———————————————%

\lstset {
    backgroundcolor=\color{white},   % choose the background color; you must add \usepackage{color} or \usepackage{xcolor}
    basicstyle=\footnotesize,        % the size of the fonts that are used for the code
    breakatwhitespace=false,         % sets if automatic breaks should only happen at whitespace
    breaklines=true,                 % sets automatic line breaking
    captionpos=bl,                   % sets the caption-position to bottom
    commentstyle=\color{mygreen},    % comment style
    deletekeywords={...},            % if you want to delete keywords from the given language
    escapeinside={\%*}{*},           % if you want to add LaTeX within your code
    extendedchars=true,              % lets you use non-ASCII characters; for 8-bits encodings only, does not work with UTF-8
    frame=single,                    % adds a frame around the code
    keepspaces=true,                 % keeps spaces in text, useful for keeping indentation of code (possibly needs columns=flexible)
    keywordstyle=\color{blue},       % keyword style
    % language=Python,               % the language of the code
    morekeywords={*,...},            % if you want to add more keywords to the set
    numbers=left,                    % where to put the line-numbers; possible values are (none, left, right)
    numbersep=5pt,                   % how far the line-numbers are from the code
    numberstyle=\tiny\color{mygray}, % the style that is used for the line-numbers
    rulecolor=\color{black},         % if not set, the frame-color may be changed on line-breaks within not-black text (e.g. comments (green here))
    showspaces=false,                % show spaces everywhere adding particular underscores; it overrides 'showstringspaces'
    showstringspaces=false,          % underline spaces within strings only
    showtabs=false,                  % show tabs within strings adding particular underscores
    stepnumber=1,                    % the step between two line-numbers. If it's 1, each line will be numbered
    stringstyle=\color{orange},      % string literal style
    tabsize=2,                       % sets default tabsize to 2 spaces
    % title=Python Code              % show the filename of files included with \lstinputlisting; also try caption instead of title
}

\linespread{1.2}

%———————————————超链接设置——————————————%

\hypersetup{
    pdfstartview=FitH, % 设置PDF文档打开时的初始视图为页面宽度适应窗口宽度(即页面水平适应)
    CJKbookmarks=true, % 用对CJK(中文、日文、韩文)字符的书签支持,确保这些字符在书签中正确显示
    bookmarksnumbered=true, % 书签带有章节编号。这对有章节编号的文档很有用
    bookmarksopen=true, % 文档打开时,书签树是展开的,方便查看所有书签
    colorlinks, % 启用彩色链接。这样,链接在PDF中会显示为彩色,而不是默认的方框
    pdfborder=001, % 设置PDF文档中链接的边框样式。001 表示链接周围没有边框,仅在单击时显示一个矩形
    linkcolor=blue, % 设置文档内部链接(如目录中的章节链接)的颜色为蓝色
    anchorcolor=blue, % 设置锚点链接(即目标在同一文档内的链接)的颜色为蓝色
    citecolor=blue, % 设置引用(如文献引用)的颜色为蓝色
}

% 

\begin{document}

\maketitle
\if\hmwkCoverPage 1
    \pagebreak
\fi


%
% first problem, id is automatically set to 1
%
\begin{homeworkProblem}

\section{5.6}

轮转调度程序的一个变种是回归轮转(regressiveround-robin)调度程序。这个调度程序为每个进
程分配时间片和优先级。时间片的初值为50ms。然而,如果一个进程获得CPU并用完它的整个
时间片(不会因I/0而阻塞),那么它的时间片会增加10ms并且它的优先级会提升。(进程的时
间片可以增加到最多100ms。)如果一个进程在用完它的整个时间片之前阻塞,那么它的时间片
会降低5ms而它的优先级不变。回归轮转调度程序会偏爱哪类进程(CPU密集型的或IO密集型
的)?请解释。

\solution

回归轮转调度程序会偏爱 CPU 密集型的进程,
这种调度器将有利于CPU受限的进程,因为它们会获得更长的时间量以及
每当它们消耗整个时间量时,优先级就会提高。因为它们能够连续完成分配的时间片,时间片和优先级会逐渐提高。
此调度程序不会惩罚I/O限制进程,因为它们在消耗全部时间之前可能会阻塞I/O,但它们的优先级保持不变。

\section{5.7}

假设有如下 1 组进程,它们的 CPU 执行时间可以毫秒来计算:
\[
\begin{array}{|c|c|c|}
\hline
\text{进程} & \text{执行时间} & \text{优先级} \\
\hline
P_1 & 2 & 2 \\
P_2 & 1 & 1 \\
P_3 & 8 & 4 \\
P_4 & 4 & 2 \\
P_5 & 5 & 3 \\
\hline
\end{array}
\]

假设进程按 $P_1$, $P_2$, $P_3$, $P_4$, $P_5$ 顺序在时刻 0 到达。

\begin{enumerate}
    \item[(a)] 画出 4 个 Gantt 图,分别演示采用每种调度算法(FCFS、SJF、非抢占优先级调度(一个较大优先级数值意味着更高优先级)和 RR(时间片 $=2$))的进程执行。
    \item[(b)] 每个进程在 (a) 里的每种调度算法下的周转时间是多少?
    \item[(c)] 每个进程在 (a) 里的每种调度算法下的等待时间是多少?
    \item[(d)] 哪一种调度算法的平均等待时间(对所有进程)最小?
\end{enumerate}

\solution

\begin{figure}[H]
    \begin{center}
    
    \begin{ganttchart}[y unit title=0.4cm,
    y unit chart=0.5cm,
    vgrid,hgrid, 
    title label anchor/.style={below=-1.6ex},
    title left shift=.05,
    title right shift=-.05,
    title height=1,
    progress label text={},
    bar height=0.7,
    group right shift=0,
    group top shift=.6,
    group height=.3]{1}{20}
    %labels
    \gantttitle{Cycle}{20} \\
    %tasks
    \ganttbar{FCFS}{1}{2}
    \ganttbar{FCFS}{3}{3}
    \ganttbar{FCFS}{4}{11}
    \ganttbar{FCFS}{12}{15}
    \ganttbar{FCFS}{16}{20} \\ 
    \ganttbar{SJF}{1}{1}
    \ganttbar{SJF}{2}{3}
    \ganttbar{SJF}{4}{7}
    \ganttbar{SJF}{8}{12}
    \ganttbar{SJF}{13}{20} \\
    \ganttbar{Priority}{1}{8}
    \ganttbar{Priority}{9}{13}
    \ganttbar{Priority}{14}{15}
    \ganttbar{Priority}{16}{19} 
    \ganttbar{Priority}{20}{20} \\
    \ganttbar{RR}{1}{2}
    \ganttbar{RR}{3}{3}
    \ganttbar{RR}{4}{5}
    \ganttbar{RR}{6}{7}
    \ganttbar{RR}{8}{9}
    \ganttbar{RR}{10}{11}
    \ganttbar{RR}{12}{13}
    \ganttbar{RR}{14}{15}
    \ganttbar{RR}{16}{17}
    \ganttbar{RR}{18}{18}
    \ganttbar{RR}{19}{20} \\
    \end{ganttchart}
    \end{center}
    \caption{Gantt Chart}

\end{figure}

\textbf{(b) 周转时间 = 完成时间 - 到达时间}

\begin{table}[H]
    \centering
    \begin{tabular}{@{}l|ccccc@{}}
    \toprule
              & \textbf{P1} & \textbf{P2} & \textbf{P3} & \textbf{P4} & \textbf{P5} \\ \midrule
    \textbf{FCFS}       & 2ms  & 3ms  & 11ms & 15ms & 20ms \\
    \textbf{SJF}        & 3ms  & 1ms  & 20ms & 7ms  & 12ms \\
    \textbf{非抢占优先级} & 15ms & 20ms & 8ms  & 19ms & 13ms \\
    \textbf{RR}         & 2ms  & 3ms  & 20ms & 13ms & 18ms \\ \bottomrule
    \end{tabular}
    \end{table}
    
    \vspace{1cm}
    
    \textbf{(c) 等待时间 = 周转时间 - 执行时间}
    
    \begin{table}[H]
    \centering
    \begin{tabular}{@{}l|ccccc@{}}
    \toprule
              & \textbf{P1} & \textbf{P2} & \textbf{P3} & \textbf{P4} & \textbf{P5} \\ \midrule
    \textbf{FCFS}       & 0ms  & 2ms  & 3ms  & 11ms & 15ms \\
    \textbf{SJF}        & 1ms  & 0ms  & 12ms & 3ms  & 7ms  \\
    \textbf{非抢占优先级} & 13ms & 19ms & 0ms  & 15ms & 8ms  \\
    \textbf{RR}         & 0ms  & 2ms  & 12ms & 9ms  & 13ms \\ \bottomrule
    \end{tabular}
    \end{table}

\vspace{1cm}

\begin{table}[H]
    \centering
    \begin{tabular}{@{}l|c@{}}
    \toprule
    \textbf{调度算法} & \textbf{平均等待时间} \\ \midrule
    \textbf{FCFS}       & 6.2ms \\
    \textbf{SJF}        & 4.6ms \\
    \textbf{非抢占优先级} & 11.0ms \\
    \textbf{RR}         & 7.2ms \\ \bottomrule
    \end{tabular}
    \caption{各调度算法的平均等待时间}
    \end{table}
    

\textbf{(d) Shortest Job First}
\begin{itemize}
    \item Shortest Job First (SJF) minimizes the average waiting time.
\end{itemize}

\section{5.12}

现有运行10个IO密集型任务和1个CPU密集型任务的一个系统。假设IO密集型任务每1ms
的 CPU计算就进行一次 I/0操作,并且每个I/0操作需要10ms来完成。另假设上下文切换开
销是0.1ms,所有进程都是长时间运行的任务。请讨论在下列条件下轮转调度程序的CPU利
用率:

\begin{itemize}
    \item (a) 时间片为 1ms
    \item (b) 时间片为 10ms
\end{itemize}

\solution

\subsection{a}

时间量为1毫秒:无论调度哪个进程,调度器每次切换上下文,都会产生0.1毫秒的上下文切换成本。
这导致CPU利用率为 $ \frac{1 ms}{1ms + 0.1 ms} \times 100 \% \approx 90.91 \% $. 

\subsection{b}

时间量为10毫秒:I/O受限的任务在仅使用1毫秒后就会发生上下文切换
毫秒的时间量。因此,循环完成所有流程所需的时间为 $ 10 \times 1.1 +
10.1 $(因为每个I/O绑定任务执行1毫秒,然后引发上下文切换任务,而
CPU限制的任务在引发上下文切换之前执行10毫秒)。CPU利用率为 $ \frac{20ms}{21.1ms} \times 100 \% \approx 94.79 \% $.

\end{homeworkProblem}

\bibliographystyle{abbrv}
\bibliography{references.bib}

\end{document}