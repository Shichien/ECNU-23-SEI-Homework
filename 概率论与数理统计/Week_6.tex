\documentclass[twoside]{article}
\usepackage[paperwidth=210mm, paperheight=297mm, margin=2cm]{geometry}
\usepackage[utf8]{inputenc}
\usepackage{ctex}

% Formatting Packages ——————————————————————————————————————
\usepackage{multicol}
\usepackage{multirow}
\usepackage{enumitem}
\usepackage{indentfirst}
\usepackage[toc]{multitoc}

% Math & Physics Packages ————————————————————————————
\usepackage{amsmath, amsthm, amsfonts, amssymb}
\usepackage{amssymb}
\usepackage{setspace}
\usepackage{physics}
\usepackage{cancel}
\usepackage{nicefrac}

% Image-related Packages —————————————————————————————
\usepackage{graphics, graphicx}
\usepackage{tikz}
\usetikzlibrary{arrows.meta}
\usepackage{pgfplots}
\pgfplotsset{compat=1.18}
\usepackage{xcolor}
\usepackage{fourier-orns}
\usepackage{lipsum}


%% Silakan otak-atik judul di sini
\title{\texorpdfstring{\vspace{-1.5em}}{}\textbf{概率论与数理统计}}
\author{张梓卫 10235101526}
\date{\the\day \ \monthyear}

% Kalau mau bab pertama nomornya 0, ganti 0 jadi -1
\setcounter{section}{0}

\renewcommand*\contentsname{第十三周概率论作业}
\renewcommand*{\multicolumntoc}{2}
\setlength{\columnseprule}{0.5pt}

% Layouting Packages ————————————————————————————————————————
\usepackage{titlesec}
\usepackage{fancyhdr}
\pagestyle{fancy}
\setlength{\headheight}{14.39996pt}
\fancyfoot[C]{\textit{By Deralive (10235101526)}}
\fancyfoot[R]{\thepage}

\fancyhead[LE, RO]{\textsl{\rightmark}}
\fancyhead[LO, RE]{\textsl{\leftmark}}

\renewcommand\headrule{
	\vspace{-6pt}
	\hrulefill
	\raisebox{-2.1pt}
	{\quad\floweroneleft\decoone\floweroneright\quad}
	\hrulefill}

\renewcommand\footrule{
	\hrulefill
	\raisebox{-2.1pt}
	{\quad\floweroneleft\decoone\floweroneright\quad}
	\hrulefill}
  
\newcommand{\fpb}[2]{\mathrm{FPB}(#1, #2)}
\newcommand{\kpk}[2]{\mathrm{KPK}(#1, #2)}

% Reference and Bibliography Packages ————————————————————
\usepackage{hyperref}
\hypersetup{
    colorlinks=true,
    linkcolor={black},
    citecolor={biru!70!black},
    urlcolor={biru!80!black}
}

\numberwithin{equation}{section}

% Silakan lihat dokumentasi package biblatex
% untuk format sitasi yang diperlukan
\usepackage[backend=biber]{biblatex}
\addbibresource{ref.bib}
\makeatletter
\def\@biblabel#1{}
\makeatother

\renewcommand{\mod}{\mathrm{mod} \ }

% Titling
\newcommand{\garis} [3] []{
	\begin{center}
		\begin{tikzpicture}
			\draw[#2-#3, ultra thick, #1] (0,0) to (1\linewidth,0);
		\end{tikzpicture}
	\end{center}
}

\newcommand{\monthyear}{
  \ifcase\month\or January\or February\or March\or April\or May\or June\or
  July\or August\or September\or October\or November\or
  December\fi\space\number\year
}

% Colour Palette ——————————————————————————————————————
\definecolor{merah}{HTML}{F4564E}
\definecolor{merahtua}{HTML}{89313E}
\definecolor{biru}{HTML}{60BBE5}
\definecolor{birutua}{HTML}{412F66}
\definecolor{hijau}{HTML}{59CC78}
\definecolor{hijautua}{HTML}{366D5B}
\definecolor{kuning}{HTML}{FFD56B}
\definecolor{jingga}{HTML}{FBA15F}
\definecolor{ungu}{HTML}{8C5FBF}
\definecolor{lavender}{HTML}{CBA5E8}
\definecolor{merjamb}{HTML}{FFB6E0}
\definecolor{mygray}{HTML}{E6E6E6}

% Theorems ————————————————————————————————————————————
\usepackage{tcolorbox}
\tcbuselibrary{skins,breakable,theorems}
\usepackage{changepage}

\newcounter{hitung}
\setcounter{hitung}{\thesection}

\makeatletter
	% Proof 证明如下
	\def\tcb@theo@widetitle#1#2#3{\hbox to \textwidth{\textsc{\large#1}\normalsize\space#3\hfil(#2)}}
	\tcbset{
		theorem style/theorem wide name and number/.code={ \let\tcb@theo@title=\tcb@theo@widetitle},
		proofbox/.style={skin=enhancedmiddle,breakable,parbox=false,boxrule=0mm,
			check odd page, toggle left and right, colframe=black!20!white!92!hijau,
			leftrule=8pt, rightrule=0mm, boxsep=0mm,arc=0mm, outer arc=0mm,
			left=3mm,right=3mm,top=0mm,bottom=0mm, toptitle=0mm,
			bottomtitle=0mm,colback=gray!3!white!98!biru, before skip=8pt, after skip=8pt,
			before={\par\vskip-2pt},after={\par\smallbreak},
		},
	}
	\newtcolorbox{ProofBox}{proofbox}
	\makeatother
	
	\let\realproof\proof
	\let\realendproof\endproof
	\renewenvironment{proof}[1][Prove:]{\ProofBox\strut\textsc{#1}\space}{\endProofBox}
        \AtEndEnvironment{proof}{\null\hfill$\blacksquare$}
        % Definition 定义环境
	\newtcbtheorem[use counter=hitung, number within=section]{dfn}{定义}
	{theorem style=theorem wide name and number,breakable,enhanced,arc=3.5mm,outer arc=3.5mm,
		boxrule=0pt,toprule=1pt,leftrule=0pt,bottomrule=1pt, rightrule=0pt,left=0.2cm,right=0.2cm,
		titlerule=0.5em,toptitle=0.1cm,bottomtitle=-0.1cm,top=0.2cm,
		colframe=white!10!biru,
		colback=white!90!biru,
		coltitle=white,
		shadow={1.3mm}{-1.3mm}{0mm}{gray!50!white}, % 添加阴影
        coltext=birutua!60!gray, title style={white!10!biru}, rbefoe skip=8pt, after skip=8pt,
		fonttitle=\bfseries,fontupper=\normalsize}{dfn}

	% 答题卡
	\newtcbtheorem[use counter=hitung, number within=section]{ans}{解答}
	{theorem style=theorem wide name and number,breakable,enhanced,arc=3.5mm,outer arc=3.5mm,
		boxrule=0pt,toprule=1pt,leftrule=0pt,bottomrule=1pt, rightrule=0pt,left=0.2cm,right=0.2cm,
		titlerule=0.5em,toptitle=0.1cm,bottomtitle=-0.1cm,top=0.2cm,
		colframe=white!10!biru,
		colback=white!90!biru,
		coltitle=white,
		shadow={1.3mm}{-1.3mm}{0mm}{gray!50!white}, % 添加阴影
        coltext=birutua!60!gray, title style={white!10!biru}, before skip=8pt, after skip=8pt,
		fonttitle=\bfseries,fontupper=\normalsize}{ans}

	% Axiom
	\newtcbtheorem[use counter=hitung, number within=section]{axm}{公理}
	{theorem style=theorem wide name and number,breakable,enhanced,arc=3.5mm,outer arc=3.5mm,
		boxrule=0pt,toprule=1pt,leftrule=0pt,bottomrule=1pt, rightrule=0pt,left=0.2cm,right=0.2cm,
		titlerule=0.5em,toptitle=0.1cm,bottomtitle=-0.1cm,top=0.2cm,
		colframe=white!10!biru,colback=white!90!biru,coltitle=white,
		shadow={1.3mm}{-1.3mm}{0mm}{gray!50!white!90}, % 添加阴影
        coltext=birutua!60!gray,title style={white!10!biru},before skip=8pt, after skip=8pt,
		fonttitle=\bfseries,fontupper=\normalsize}{axm}
 
	% Theorem
	\newtcbtheorem[use counter=hitung, number within=section]{thm}{定理}
	{theorem style=theorem wide name and number,breakable,enhanced,arc=3.5mm,outer arc=3.5mm,
		boxrule=0pt,toprule=1pt,leftrule=0pt,bottomrule=1pt, rightrule=0pt,left=0.2cm,right=0.2cm,
		titlerule=0.5em,toptitle=0.1cm,bottomtitle=-0.1cm,top=0.2cm,
		colframe=white!10!merah,colback=white!75!pink,coltitle=white, coltext=merahtua!80!merah,
		shadow={1.3mm}{-1.3mm}{0mm}{gray!50!white!90}, % 添加阴影
		title style={white!10!merah}, before skip=8pt, after skip=8pt,
		fonttitle=\bfseries,fontupper=\normalsize}{thm}
	
	% Proposition
	\newtcbtheorem[use counter=hitung, number within=section]{prp}{命题}
	{theorem style=theorem wide name and number,breakable,enhanced,arc=3.5mm,outer arc=3.5mm,
		boxrule=0pt,toprule=1pt,leftrule=0pt,bottomrule=1pt, rightrule=0pt,left=0.2cm,right=0.2cm,
		titlerule=0.5em,toptitle=0.1cm,bottomtitle=-0.1cm,top=0.2cm,
		colframe=white!10!hijau,colback=white!90!hijau,coltitle=white, coltext=hijautua!80!brown,
		shadow={1.3mm}{-1.3mm}{0mm}{gray!50!white}, % 添加阴影
		title style={white!10!hijau}, before skip=8pt, after skip=8pt,
		fonttitle=\bfseries,fontupper=\normalsize}{prp}


	% Example
	\newtcolorbox[use counter=hitung, number within=section]{cth}[1][]{breakable,
		colframe=white!10!jingga, coltitle=white!90!jingga, colback=white!85!jingga, coltext=black!10!brown!50!jingga, colbacktitle=white!10!jingga, enhanced, fonttitle=\bfseries,fontupper=\normalsize, attach boxed title to top left={yshift=-2mm}, before skip=8pt, after skip=8pt,
		title=Contoh~\thetcbcounter \ \ #1}

	% Catatan/Note
	\newtcolorbox{ctt}[1][]{enhanced, 
		left=4.1mm, borderline west={8pt}{0pt}{white!10!kuning}, 
		before skip=6pt, after skip=6pt, 
		colback=white!85!kuning, colframe= white!85!kuning, coltitle=orange!60!kuning!25!brown, coltext=orange!60!kuning!25!brown,
		fonttitle=\bfseries,fontupper=\normalsize, before skip=8pt, after skip=8pt,
		title=\underline{Catatan}  #1}
	
	% Komentar/Remark
	\newtcolorbox{rmr}[1][]{
		,arc=0mm,outer arc=0mm,
		boxrule=0pt,toprule=1pt,leftrule=0pt,bottomrule=5pt, rightrule=0pt,left=0.2cm,right=0.2cm,
		titlerule=0.5em,toptitle=0.1cm,bottomtitle=-0.1cm,top=0.2cm,
		colframe=white!10!kuning,colback=white!85!kuning,coltitle=white, coltext=orange!60!kuning,
		fonttitle=\bfseries,fontupper=\normalsize, before skip=8pt, after skip=8pt,
		title=Komentar  #1}

% ————————————————————————————————————————————————————————————————————————
\begin{document}

\maketitle
\vspace{-3.5em}
\garis{Kite}{Kite}

\tableofcontents
    
\section{第二章习题 21}

设随机变量 $X$ 的概率密度为

(1)
\[
f(x) =
\begin{cases} 
2 \left(1 - \frac{1}{x^2} \right), & 1 \leq x \leq 2, \\
0, & \text{其他}.
\end{cases}
\]

(2)
\[
f(x) =
\begin{cases} 
x, & 0 \leq x < 1, \\
2 - x, & 1 \leq x < 2, \\
0, & \text{其他}.
\end{cases}
\]

求 $X$ 的分布函数 $F(x)$, 并画出(2)中的 $f(x)$ 及 $F(x)$ 的图形。

\begin{ans}{21}{21}

    \textbf{解:}

由于概率密度函数 $f(x)$ 在 $x < 1$ 和 $x > 2$ 处等于零, 因此:

当 $x < 1$ 时,
  \[
  F(x) = \int_{-\infty}^x f(t) dt = \int_{-\infty}^x 0 dt = 0;
  \]
当 $x \geq 2$ 时,
  \[
  F(x) = \int_{-\infty}^x f(t) dt = 1 - \int_2^{\infty} 0 dt = 1;
  \]

当 $1 \leq x \leq 2$ 时,
  \[
  F(x) = \int_{-\infty}^x f(t) dt = \int_{-\infty}^1 0 dt + \int_1^x 2\left(1 - \frac{1}{t^2}\right) dt.
  \]
  对于 $1 \leq x \leq 2$, 通过计算:
  \[
  F(x) = 2\left(x + \frac{1}{x} - 2 \right).
  \]

因此,分布函数 $F(x)$ 为:
\[
F(x) =
\begin{cases} 
0, & x < 1, \\
2 \left(x + \frac{1}{x} - 2\right), & 1 \leq x \leq 2, \\
1, & x > 2.
\end{cases}
\]

(2) 概率密度 $f(x)$ 在 $x < 0$ 和 $x > 2$ 处等于零, 因此:

当 $x < 0$ 时, $F(x) = 0$;
当 $x \geq 2$ 时, $F(x) = 1$;
当 $0 \leq x < 1$ 时,
  \[
  F(x) = \int_{-\infty}^x f(t) dt = \int_0^x t dt = \frac{x^2}{2};
  \]
当 $1 \leq x < 2$ 时,
  \[
  F(x) = \int_{-\infty}^x f(t) dt = \frac{1}{2} + \left( 2x - \frac{x^2}{2} - 1 \right).
  \]

因此,分布函数 $F(x)$ 为:
\[
F(x) =
\begin{cases} 
0, & x < 0, \\
\frac{x^2}{2}, & 0 \leq x < 1, \\
2x - \frac{x^2}{2} - 1, & 1 \leq x < 2, \\
1, & x \geq 2.
\end{cases}
\]

\end{ans}

\section{第二章习题 24}

设顾客在某银行的窗口等候服务的时间 $X$(以分钟计)服从指数分布,其概率密度为
\[
f_X(x) =
\begin{cases} 
\frac{1}{5} e^{-x/5}, & x > 0, \\
0, & \text{其他}.
\end{cases}
\]
若顾客在窗口等候服务,如果超过10分钟就离开。设一个月内该顾客需要去银行5次。以 $Y$ 表示一个月内他未等到服务而离开的次数。写出 $Y$ 的分布律,并求 $P(Y \geq 1)$。

\begin{ans}{24}{24}

顾客在窗口等候服务超过 10 分钟的概率为:
\[
p = \int_{10}^{\infty} f_X(x) dx = \int_{10}^{\infty} \frac{1}{5} e^{-x/5} dx = e^{-2}.
\]

即 $Y$ 服从参数为 $n = 5$, $p = e^{-2}$ 的二项分布。即 $Y \sim \text{B}(5, e^{-2})$。

$Y$ 的分布律为:
\[
P(Y = k) =  C_{5}^{k} \times (e^{-2})^k (1 - e^{-2})^{5-k},
\]

要求的概率 $P(Y \geq 1)$ 为:

\[
P(Y \geq 1) = 1 - P(Y = 0) = 1 - (1 - e^{-2})^5 \approx 0.5167.
\]

\end{ans}

\section{第二章习题 26}

\textbf{题目26:} 设 $X \sim N(3, 2^2)$.

(1) 求 $P\{2 < X < 5\}$, $P\{-4 < X < 10\}$, $P\{|X| > 2\}$, $P\{X > 3\}$。

(2) 确定 $c$,使得 $P(X > c) = P(X < c)$。

(3) 设 $d$ 满足 $P(X > d) \geq 0.9$,问 $d$ 至多为多少?

\begin{ans}{26}{26}

首先,随机变量 $X$ 的正态分布为 $X \sim N(3, 2^2)$,即均值为 3,方差为 4。
\[
Z = \frac{X - 3}{2}, \quad Z \sim N(0, 1).
\]

    1.
    \[
    P(2 < X < 5) = P\left(\frac{2 - 3}{2} < Z < \frac{5 - 3}{2}\right) = P(-0.5 < Z < 1)。
    \]

    \[
    P(-0.5 < Z < 1) = \Phi(1) - \Phi(-0.5) = 0.8413 - 0.3085 = 0.5328.
    \]
    
    2.
    \[
    P(-4 < X < 10) = P\left(\frac{-4 - 3}{2} < Z < \frac{10 - 3}{2}\right) = P(-3.5 < Z < 3.5).
    \]
 
    \[
    P(-3.5 < Z < 3.5) = \Phi(3.5) - \Phi(-3.5) = 0.9998 - 0.0002 = 0.9996.
    \]
    
    3.
    \[
    P(|X| > 2) = P(X > 2) + P(X < -2).
    \]

    \[
    P(X > 2) = P\left(Z > \frac{2 - 3}{2}\right) = P(Z > -0.5) = 1 - \Phi(-0.5) = 1 - 0.3085 = 0.6915.
    \]

    \[
    P(|X| > 2) = 0.6915.
    \]
    
    4.
    \[
    P(X > 3) = P(Z > 0) = 1 - \Phi(0) = 0.5.
    \]
    
    (2) 
    
    $X$ 是对称分布的,解得 $c = 3$。
    
    (3)

    显然我们需要寻找的为:
    \[
    P\left(Z > \frac{d - 3}{2}\right) = 0.9,
    \]
    通过查表可得 $P(Z > 1.28) \approx 0.9$,所以:
    \[
    \frac{d - 3}{2} = 1.28, \quad d = 3 + 2 \times 1.28 = 5.56.
    \]
    
    因此,$d \approx 5.56$。


\end{ans}

\section{第二章习题 29}

一工厂生产的某种元件的寿命 $X$(以小时计)服从参数为 $\mu = 160$,$\sigma > 0$ 的正态分布。若要求 $P(120 < X < 200) \geq 0.80$,允许的 $\sigma$ 最大为多少?

\begin{ans}{29}{29}

    已知 $X \sim N(160, \sigma^2)$,现要求
\[
P(120 < X < 200) = \Phi\left(\frac{40}{\sigma}\right) - \Phi\left(\frac{-40}{\sigma}\right) = 2\Phi\left(\frac{40}{\sigma}\right) - 1 \geq 0.80.
\]

即要求:
\[
2\Phi\left(\frac{40}{\sigma}\right) - 1 \geq 0.80,
\]
解得:
\[
\Phi\left(\frac{40}{\sigma}\right) \geq 0.9 = \Phi(1.282).
\]

解不等式, $\sigma$ 最大约为 $31.20$。

\end{ans}

\section{第二章习题 35}

设随机变量 $X \sim N(0,1)$。

(1) 求 $Y = e^X$ 的概率密度。

(2) 求 $Y = 2X^2 + 1$ 的概率密度。

(3) 求 $Y = |X|$ 的概率密度。

\begin{ans}{35}{35}

(1)
由 $Y = e^X$,有 $X = \ln Y$。

此时,$X$ 的概率密度为标准正态分布,即:
\[
f_X(x) = \frac{1}{\sqrt{2\pi}} e^{-x^2 / 2}.
\]
根据变量变换公式,$Y$ 的概率密度函数为:
\[
f_Y(y) = f_X(\ln y) \cdot \left| \frac{d}{dy} \ln y \right|.
\]
可得:
\[
f_Y(y) = \frac{1}{\sqrt{2\pi}} e^{-(\ln y)^2 / 2} \cdot \frac{1}{y}, \quad y > 0.
\]
故 $Y = e^X$ 的概率密度为:
\[
f_Y(y) = \frac{1}{y \sqrt{2\pi}} e^{-(\ln y)^2 / 2}, \quad y > 0.
\]

(2)
使用变量变换法。令:
\[
Y = 2X^2 + 1 \quad \Rightarrow \quad X^2 = \frac{Y - 1}{2}.
\]
则 $X = \pm \sqrt{\frac{Y - 1}{2}}$。

此时概率密度函数为:
\[
f_Y(y) = \frac{1}{\sqrt{2\pi}} \cdot \frac{1}{\sqrt{y - 1}} e^{-\frac{y - 1}{2}}, \quad y \geq 1.
\]

(3)
根据正态分布的对称性,则由:
\[
P(Y \leq y) = P(-y \leq X \leq y) = 2P(0 \leq X \leq y), \quad y \geq 0.
\]
可得,$Y$ 的概率密度为:
\[
f_Y(y) = 2f_X(y) = \frac{2}{\sqrt{2\pi}} e^{-y^2 / 2}, \quad y \geq 0.
\]

\end{ans}

\section{第二章习题 38}

设电流 $I$ 是一个随机变量,它均匀分布在 $9 \sim 11$ A 之间。若此电流通过 $2 \Omega$ 的电阻,在其上消耗的功率为 $W = 2I^2$,求 $W$ 的概率密度。

\begin{ans}{38}{38}


\end{ans}

\end{document}
