\documentclass[twoside]{article}
\usepackage[paperwidth=210mm, paperheight=297mm, margin=2cm]{geometry}
\usepackage[utf8]{inputenc}
\usepackage{ctex}

% Formatting Packages ——————————————————————————————————————
\usepackage{multicol}
\usepackage{multirow}
\usepackage{enumitem}
\usepackage{indentfirst}
\usepackage[toc]{multitoc}

% Math & Physics Packages ————————————————————————————
\usepackage{amsmath, amsthm, amsfonts, amssymb}
\usepackage{amssymb}
\usepackage{setspace}
\usepackage{physics}
\usepackage{cancel}
\usepackage{nicefrac}

% Image-related Packages —————————————————————————————
\usepackage{graphics, graphicx}
\usepackage{tikz}
\usetikzlibrary{arrows.meta}
\usepackage{pgfplots}
\pgfplotsset{compat=1.18}
\usepackage{xcolor}
\usepackage{fourier-orns}
\usepackage{lipsum}


%% Silakan otak-atik judul di sini
\title{\texorpdfstring{\vspace{-1.5em}}{}\textbf{概率论与数理统计}}
\author{张梓卫 10235101526}
\date{\the\day \ \monthyear}

% Kalau mau bab pertama nomornya 0, ganti 0 jadi -1
\setcounter{section}{0}

\renewcommand*\contentsname{第十三周概率论作业}
\renewcommand*{\multicolumntoc}{2}
\setlength{\columnseprule}{0.5pt}

% Layouting Packages ————————————————————————————————————————
\usepackage{titlesec}
\usepackage{fancyhdr}
\pagestyle{fancy}
\setlength{\headheight}{14.39996pt}
\fancyfoot[C]{\textit{By Deralive (10235101526)}}
\fancyfoot[R]{\thepage}

\fancyhead[LE, RO]{\textsl{\rightmark}}
\fancyhead[LO, RE]{\textsl{\leftmark}}

\renewcommand\headrule{
	\vspace{-6pt}
	\hrulefill
	\raisebox{-2.1pt}
	{\quad\floweroneleft\decoone\floweroneright\quad}
	\hrulefill}

\renewcommand\footrule{
	\hrulefill
	\raisebox{-2.1pt}
	{\quad\floweroneleft\decoone\floweroneright\quad}
	\hrulefill}
  
\newcommand{\fpb}[2]{\mathrm{FPB}(#1, #2)}
\newcommand{\kpk}[2]{\mathrm{KPK}(#1, #2)}

% Reference and Bibliography Packages ————————————————————
\usepackage{hyperref}
\hypersetup{
    colorlinks=true,
    linkcolor={black},
    citecolor={biru!70!black},
    urlcolor={biru!80!black}
}

\numberwithin{equation}{section}

% Silakan lihat dokumentasi package biblatex
% untuk format sitasi yang diperlukan
\usepackage[backend=biber]{biblatex}
\addbibresource{ref.bib}
\makeatletter
\def\@biblabel#1{}
\makeatother

\renewcommand{\mod}{\mathrm{mod} \ }

% Titling
\newcommand{\garis} [3] []{
	\begin{center}
		\begin{tikzpicture}
			\draw[#2-#3, ultra thick, #1] (0,0) to (1\linewidth,0);
		\end{tikzpicture}
	\end{center}
}

\newcommand{\monthyear}{
  \ifcase\month\or January\or February\or March\or April\or May\or June\or
  July\or August\or September\or October\or November\or
  December\fi\space\number\year
}

% Colour Palette ——————————————————————————————————————
\definecolor{merah}{HTML}{F4564E}
\definecolor{merahtua}{HTML}{89313E}
\definecolor{biru}{HTML}{60BBE5}
\definecolor{birutua}{HTML}{412F66}
\definecolor{hijau}{HTML}{59CC78}
\definecolor{hijautua}{HTML}{366D5B}
\definecolor{kuning}{HTML}{FFD56B}
\definecolor{jingga}{HTML}{FBA15F}
\definecolor{ungu}{HTML}{8C5FBF}
\definecolor{lavender}{HTML}{CBA5E8}
\definecolor{merjamb}{HTML}{FFB6E0}
\definecolor{mygray}{HTML}{E6E6E6}

% Theorems ————————————————————————————————————————————
\usepackage{tcolorbox}
\tcbuselibrary{skins,breakable,theorems}
\usepackage{changepage}

\newcounter{hitung}
\setcounter{hitung}{\thesection}

\makeatletter
	% Proof 证明如下
	\def\tcb@theo@widetitle#1#2#3{\hbox to \textwidth{\textsc{\large#1}\normalsize\space#3\hfil(#2)}}
	\tcbset{
		theorem style/theorem wide name and number/.code={ \let\tcb@theo@title=\tcb@theo@widetitle},
		proofbox/.style={skin=enhancedmiddle,breakable,parbox=false,boxrule=0mm,
			check odd page, toggle left and right, colframe=black!20!white!92!hijau,
			leftrule=8pt, rightrule=0mm, boxsep=0mm,arc=0mm, outer arc=0mm,
			left=3mm,right=3mm,top=0mm,bottom=0mm, toptitle=0mm,
			bottomtitle=0mm,colback=gray!3!white!98!biru, before skip=8pt, after skip=8pt,
			before={\par\vskip-2pt},after={\par\smallbreak},
		},
	}
	\newtcolorbox{ProofBox}{proofbox}
	\makeatother
	
	\let\realproof\proof
	\let\realendproof\endproof
	\renewenvironment{proof}[1][Prove:]{\ProofBox\strut\textsc{#1}\space}{\endProofBox}
        \AtEndEnvironment{proof}{\null\hfill$\blacksquare$}
        % Definition 定义环境
	\newtcbtheorem[use counter=hitung, number within=section]{dfn}{定义}
	{theorem style=theorem wide name and number,breakable,enhanced,arc=3.5mm,outer arc=3.5mm,
		boxrule=0pt,toprule=1pt,leftrule=0pt,bottomrule=1pt, rightrule=0pt,left=0.2cm,right=0.2cm,
		titlerule=0.5em,toptitle=0.1cm,bottomtitle=-0.1cm,top=0.2cm,
		colframe=white!10!biru,
		colback=white!90!biru,
		coltitle=white,
		shadow={1.3mm}{-1.3mm}{0mm}{gray!50!white}, % 添加阴影
        coltext=birutua!60!gray, title style={white!10!biru}, rbefoe skip=8pt, after skip=8pt,
		fonttitle=\bfseries,fontupper=\normalsize}{dfn}

	% 答题卡
	\newtcbtheorem[use counter=hitung, number within=section]{ans}{解答}
	{theorem style=theorem wide name and number,breakable,enhanced,arc=3.5mm,outer arc=3.5mm,
		boxrule=0pt,toprule=1pt,leftrule=0pt,bottomrule=1pt, rightrule=0pt,left=0.2cm,right=0.2cm,
		titlerule=0.5em,toptitle=0.1cm,bottomtitle=-0.1cm,top=0.2cm,
		colframe=white!10!biru,
		colback=white!90!biru,
		coltitle=white,
		shadow={1.3mm}{-1.3mm}{0mm}{gray!50!white}, % 添加阴影
        coltext=birutua!60!gray, title style={white!10!biru}, before skip=8pt, after skip=8pt,
		fonttitle=\bfseries,fontupper=\normalsize}{ans}

	% Axiom
	\newtcbtheorem[use counter=hitung, number within=section]{axm}{公理}
	{theorem style=theorem wide name and number,breakable,enhanced,arc=3.5mm,outer arc=3.5mm,
		boxrule=0pt,toprule=1pt,leftrule=0pt,bottomrule=1pt, rightrule=0pt,left=0.2cm,right=0.2cm,
		titlerule=0.5em,toptitle=0.1cm,bottomtitle=-0.1cm,top=0.2cm,
		colframe=white!10!biru,colback=white!90!biru,coltitle=white,
		shadow={1.3mm}{-1.3mm}{0mm}{gray!50!white!90}, % 添加阴影
        coltext=birutua!60!gray,title style={white!10!biru},before skip=8pt, after skip=8pt,
		fonttitle=\bfseries,fontupper=\normalsize}{axm}
 
	% Theorem
	\newtcbtheorem[use counter=hitung, number within=section]{thm}{定理}
	{theorem style=theorem wide name and number,breakable,enhanced,arc=3.5mm,outer arc=3.5mm,
		boxrule=0pt,toprule=1pt,leftrule=0pt,bottomrule=1pt, rightrule=0pt,left=0.2cm,right=0.2cm,
		titlerule=0.5em,toptitle=0.1cm,bottomtitle=-0.1cm,top=0.2cm,
		colframe=white!10!merah,colback=white!75!pink,coltitle=white, coltext=merahtua!80!merah,
		shadow={1.3mm}{-1.3mm}{0mm}{gray!50!white!90}, % 添加阴影
		title style={white!10!merah}, before skip=8pt, after skip=8pt,
		fonttitle=\bfseries,fontupper=\normalsize}{thm}
	
	% Proposition
	\newtcbtheorem[use counter=hitung, number within=section]{prp}{命题}
	{theorem style=theorem wide name and number,breakable,enhanced,arc=3.5mm,outer arc=3.5mm,
		boxrule=0pt,toprule=1pt,leftrule=0pt,bottomrule=1pt, rightrule=0pt,left=0.2cm,right=0.2cm,
		titlerule=0.5em,toptitle=0.1cm,bottomtitle=-0.1cm,top=0.2cm,
		colframe=white!10!hijau,colback=white!90!hijau,coltitle=white, coltext=hijautua!80!brown,
		shadow={1.3mm}{-1.3mm}{0mm}{gray!50!white}, % 添加阴影
		title style={white!10!hijau}, before skip=8pt, after skip=8pt,
		fonttitle=\bfseries,fontupper=\normalsize}{prp}


	% Example
	\newtcolorbox[use counter=hitung, number within=section]{cth}[1][]{breakable,
		colframe=white!10!jingga, coltitle=white!90!jingga, colback=white!85!jingga, coltext=black!10!brown!50!jingga, colbacktitle=white!10!jingga, enhanced, fonttitle=\bfseries,fontupper=\normalsize, attach boxed title to top left={yshift=-2mm}, before skip=8pt, after skip=8pt,
		title=Contoh~\thetcbcounter \ \ #1}

	% Catatan/Note
	\newtcolorbox{ctt}[1][]{enhanced, 
		left=4.1mm, borderline west={8pt}{0pt}{white!10!kuning}, 
		before skip=6pt, after skip=6pt, 
		colback=white!85!kuning, colframe= white!85!kuning, coltitle=orange!60!kuning!25!brown, coltext=orange!60!kuning!25!brown,
		fonttitle=\bfseries,fontupper=\normalsize, before skip=8pt, after skip=8pt,
		title=\underline{Catatan}  #1}
	
	% Komentar/Remark
	\newtcolorbox{rmr}[1][]{
		,arc=0mm,outer arc=0mm,
		boxrule=0pt,toprule=1pt,leftrule=0pt,bottomrule=5pt, rightrule=0pt,left=0.2cm,right=0.2cm,
		titlerule=0.5em,toptitle=0.1cm,bottomtitle=-0.1cm,top=0.2cm,
		colframe=white!10!kuning,colback=white!85!kuning,coltitle=white, coltext=orange!60!kuning,
		fonttitle=\bfseries,fontupper=\normalsize, before skip=8pt, after skip=8pt,
		title=Komentar  #1}

% ————————————————————————————————————————————————————————————————————————
\begin{document}

\maketitle
\vspace{-3.5em}
\garis{Kite}{Kite}

\tableofcontents

\section{第七章习题 5}

\begin{ans}{5}{5}
(1) 最大似然估计计算:

概率密度函数为:
\[
f(t) = 
\begin{cases} 
\frac{1}{\theta} e^{-\frac{(t-c)}{\theta}}, & t \geq c, \\
0, & t < c,
\end{cases}
\]

由题意,失效时间为 \( x_1, x_2, \ldots, x_n \),因此似然函数为:
\[
L(\theta, c) = \prod_{i=1}^n f(x_i) = \prod_{i=1}^n \frac{1}{\theta} e^{-\frac{(x_i-c)}{\theta}} = \frac{1}{\theta^n} e^{-\frac{\sum_{i=1}^n (x_i-c)}{\theta}}, \quad c \leq x_1.
\]

取对数,得对数似然函数:
\[
\ln L(\theta, c) = -n \ln \theta - \frac{\sum_{i=1}^n (x_i-c)}{\theta}.
\]

对 \(\theta\) 和 \(c\) 分别求偏导并令其为 0:
\[
\frac{\partial \ln L}{\partial \theta} = -\frac{n}{\theta} + \frac{\sum_{i=1}^n (x_i-c)}{\theta^2} = 0,
\]
解得:
\[
\hat{\theta} = \frac{\sum_{i=1}^n (x_i-c)}{n}.
\]

似然函数在 \(c = x_1\) 处取得最大值:
\[
\hat{c} = x_1.
\]

将 \(\hat{c}\) 代入 \(\hat{\theta}\),得:
\[
\hat{\theta} = \frac{\sum_{i=1}^n x_i - n x_1}{n} = \bar{x} - x_1.
\]

故最大似然估计为:
\[
\hat{c} = x_1, \quad \hat{\theta} = \bar{x} - x_1.
\]

(2) 矩估计计算:

一阶矩为:
\[
\mu_1 = \int_{c}^\infty t f(t) dt = \int_{c}^\infty t \frac{1}{\theta} e^{-\frac{(t-c)}{\theta}} dt.
\]
令 \(u = \frac{t-c}{\theta}\),则 \(t = \theta u + c\),代入得:
\[
\mu_1 = \int_{0}^\infty (\theta u + c) e^{-u} du = c + \theta \Gamma(2) = c + \theta.
\]

二阶矩为:
\[
\mu_2 = \int_{c}^\infty t^2 f(t) dt = \int_{c}^\infty t^2 \frac{1}{\theta} e^{-\frac{(t-c)}{\theta}} dt.
\]
同样令 \(u = \frac{t-c}{\theta}\),代入得:
\[
\mu_2 = \int_{0}^\infty (\theta u + c)^2 e^{-u} du = \theta^2 \Gamma(3) + 2c\theta \Gamma(2) + c^2 \Gamma(1),
\]
化简得:
\[
\mu_2 = 2\theta^2 + 2c\theta + c^2.
\]

由矩估计公式,结合样本的中心矩:
\[
\mu_1 = \bar{x}, \quad \mu_2 = \frac{1}{n} \sum_{i=1}^n x_i^2,
\]
可解得:
\[
\theta = \sqrt{\mu_2 - \mu_1^2}, \quad c = \mu_1 - \sqrt{\mu_2 - \mu_1^2}.
\]

所以样本矩估计为:
\[
\hat{\theta} = \sqrt{\frac{1}{n} \sum_{i=1}^n x_i^2 - \bar{x}^2}, \quad \hat{c} = \bar{x} - \sqrt{\frac{1}{n} \sum_{i=1}^n x_i^2 - \bar{x}^2}.
\]

\end{ans}

\section{第七章习题 9}

\begin{ans}{9}{9}
(1) 验证 \( S_w^2 \) 是总体方差 \(\sigma^2\) 的无偏估计量:

由定义,合并估计公式为:
\[
S_w^2 = \frac{(n_1-1)S_1^2 + (n_2-1)S_2^2}{n_1 + n_2 - 2}.
\]

由 \(\mathbb{E}[S_1^2] = \mathbb{E}[S_2^2] = \sigma^2\) 得:
\[
\mathbb{E}[S_w^2] = \mathbb{E} \left[ \frac{(n_1-1)S_1^2 + (n_2-1)S_2^2}{n_1 + n_2 - 2} \right].
\]

将期望分配:
\[
\mathbb{E}[S_w^2] = \frac{(n_1-1)\mathbb{E}[S_1^2] + (n_2-1)\mathbb{E}[S_2^2]}{n_1 + n_2 - 2}.
\]

由 \(\mathbb{E}[S_1^2] = \sigma^2\) 且 \(\mathbb{E}[S_2^2] = \sigma^2\),代入得到:
\[
\mathbb{E}[S_w^2] = \frac{(n_1-1)\sigma^2 + (n_2-1)\sigma^2}{n_1 + n_2 - 2}.
\]

化简得:
\[
\mathbb{E}[S_w^2] = \sigma^2.
\]

故 \( S_w^2 \) 是总体方差 \(\sigma^2\) 的无偏估计量。

(2) 验证 \(\frac{\sum_{i=1}^n a_i X_i}{\sum_{i=1}^n a_i}\) 是总体均值 \(\mu\) 的无偏估计量:

令:
\[
\hat{\mu} = \frac{\sum_{i=1}^n a_i X_i}{\sum_{i=1}^n a_i},
\]
其中 \( a_i \) 是常数,且 \(\sum_{i=1}^n a_i \neq 0\)。

计算 \(\mathbb{E}[\hat{\mu}]\):
\[
\mathbb{E}[\hat{\mu}] = \mathbb{E} \left[ \frac{\sum_{i=1}^n a_i X_i}{\sum_{i=1}^n a_i} \right].
\]

将期望运算分配到求和内,得:
\[
\mathbb{E}[\hat{\mu}] = \frac{\sum_{i=1}^n a_i \mathbb{E}[X_i]}{\sum_{i=1}^n a_i}.
\]

由于每个 \( X_i \) 的期望为 \(\mu\),即 \(\mathbb{E}[X_i] = \mu\),所以:
\[
\mathbb{E}[\hat{\mu}] = \frac{\sum_{i=1}^n a_i \mu}{\sum_{i=1}^n a_i}.
\]

化简得:
\[
\mathbb{E}[\hat{\mu}] = \mu.
\]

因此 \(\frac{\sum_{i=1}^n a_i X_i}{\sum_{i=1}^n a_i}\) 是总体均值 \(\mu\) 的无偏估计量。


\end{ans}

\section{第七章习题 10}

\begin{ans}{10}{10}
    \section*{解答}

(1) 确定常数 \( c \),使 \( c \sum_{i=1}^{n-1}(X_{i+1} - X_i)^2 \) 是 \(\sigma^2\) 的无偏估计量:

\[
\mathbb{E} \left[ c \sum_{i=1}^{n-1} (X_{i+1} - X_i)^2 \right] = c \sum_{i=1}^{n-1} \mathbb{E}[(X_{i+1} - X_i)^2].
\]

\[
\mathbb{E}[(X_{i+1} - X_i)^2] = \text{Var}(X_{i+1}) + \text{Var}(X_i) = 2\sigma^2,
\]
因此:
\[
\mathbb{E} \left[ c \sum_{i=1}^{n-1} (X_{i+1} - X_i)^2 \right] = c \cdot 2\sigma^2 (n-1).
\]

令其等于 \(\sigma^2\),即:
\[
c \cdot 2\sigma^2 (n-1) = \sigma^2.
\]

\[
c = \frac{1}{2(n-1)}.
\]

(2) 确定常数 \( c \),使 \( \bar{X}^2 - cS^2 \) 是 \(\mu^2\) 的无偏估计量:

样本均值的平方的期望为:
\[
\mathbb{E}[\bar{X}^2] = \mathbb{E}\left[\left(\frac{1}{n} \sum_{i=1}^n X_i\right)^2\right] = \frac{\sigma^2}{n} + \mu^2.
\]

样本方差的期望为:
\[
\mathbb{E}[S^2] = \sigma^2.
\]

\[
\mathbb{E}[\bar{X}^2 - cS^2] = \mathbb{E}[\bar{X}^2] - c\mathbb{E}[S^2].
\]

\[
\mathbb{E}[\bar{X}^2 - cS^2] = \left(\frac{\sigma^2}{n} + \mu^2\right) - c\sigma^2.
\]

令其等于 \(\mu^2\),即:
\[
\frac{\sigma^2}{n} + \mu^2 - c\sigma^2 = \mu^2.
\]

\[
c = \frac{1}{n}.
\]

\end{ans}

\section{第七章习题 13}

\begin{ans}{13}{13}
    (1) 证明 \(\hat{\theta}^2 = (\hat{\theta})^2\) 不是 \(\theta^2\) 的无偏估计量:

已知 \(\hat{\theta}\) 是参数 \(\theta\) 的无偏估计量,且 \(\text{Var}(\hat{\theta}) > 0\),因此:
\[
\mathbb{E}[\hat{\theta}] = \theta.
\]

由平方的期望性质可得:
\[
\mathbb{E}[\hat{\theta}^2] = \text{Var}(\hat{\theta}) + \mathbb{E}[\hat{\theta}]^2.
\]

将 \(\mathbb{E}[\hat{\theta}] = \theta\) 代入,得:
\[
\mathbb{E}[\hat{\theta}^2] = \text{Var}(\hat{\theta}) + \theta^2.
\]

显然,当 \(\text{Var}(\hat{\theta}) > 0\) 时:
\[
\mathbb{E}[\hat{\theta}^2] > \theta^2.
\]

因此,\(\hat{\theta}^2\) 不是 \(\theta^2\) 的无偏估计量。

(2) 验证 \(\hat{\theta} = \max\{X_1, X_2, \dots, X_n\}\) 的最大似然估计量不是无偏估计量:

概率密度函数为:
\[
f(x; \theta) = 
\begin{cases} 
\frac{1}{\theta}, & 0 < x \leq \theta, \\
0, & \text{其他}.
\end{cases}
\]

对应的似然函数为:
\[
L(\theta) = 
\begin{cases} 
\frac{1}{\theta^n}, & \theta \geq x_{(n)}, \\
0, & \theta < x_{(n)},
\end{cases}
\]
其中 \(x_{(n)} = \max\{X_1, X_2, \dots, X_n\}\)。

当 \(\theta \geq x_{(n)}\) 时,\(L(\theta)\) 随 \(\theta\) 的增加而减小;当 \(\theta < x_{(n)}\) 时,\(L(\theta) = 0\)。因此,\(L(\theta)\) 在 \(\theta = x_{(n)}\) 处取得最大值。

由此可得,\(\hat{\theta} = x_{(n)}\) 是参数 \(\theta\) 的最大似然估计量。

现在计算 \(\mathbb{E}[\hat{\theta}]\):

累积分布函数为:
\[
F(x) = 
\begin{cases} 
0, & x < 0, \\
\frac{x}{\theta}, & 0 \leq x \leq \theta, \\
1, & x > \theta.
\end{cases}
\]

由最大值的分布性质,\(\hat{\theta} = \max\{X_1, X_2, \dots, X_n\}\) 的分布函数为:
\[
F_{\hat{\theta}}(z) = [F(z)]^n = \left(\frac{z}{\theta}\right)^n, \quad 0 \leq z \leq \theta.
\]

概率密度函数为:
\[
f_{\hat{\theta}}(z) = \frac{d}{dz} F_{\hat{\theta}}(z) = n \frac{z^{n-1}}{\theta^n}, \quad 0 \leq z \leq \theta.
\]

计算期望值:
\[
\mathbb{E}[\hat{\theta}] = \int_0^\theta z f_{\hat{\theta}}(z) \, dz = \int_0^\theta z \cdot n \frac{z^{n-1}}{\theta^n} \, dz.
\]

化简得:
\[
\mathbb{E}[\hat{\theta}] = n \int_0^\theta \frac{z^n}{\theta^n} \, dz = n \cdot \frac{1}{\theta^n} \cdot \frac{\theta^{n+1}}{n+1}.
\]

进一步化简:
\[
\mathbb{E}[\hat{\theta}] = \frac{n}{n+1} \theta.
\]

由于 \(\mathbb{E}[\hat{\theta}] \neq \theta\),因此 \(\hat{\theta}\) 不是 \(\theta\) 的无偏估计量。

\end{ans}

\end{document}