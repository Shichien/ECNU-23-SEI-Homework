\documentclass[twoside]{article}
\usepackage[paperwidth=210mm, paperheight=297mm, margin=2cm]{geometry}
\usepackage[utf8]{inputenc}
\usepackage{ctex}

% Formatting Packages ——————————————————————————————————————
\usepackage{multicol}
\usepackage{multirow}
\usepackage{enumitem}
\usepackage{indentfirst}
\usepackage[toc]{multitoc}

% Math & Physics Packages ————————————————————————————
\usepackage{amsmath, amsthm, amsfonts, amssymb}
\usepackage{amssymb}
\usepackage{setspace}
\usepackage{physics}
\usepackage{cancel}
\usepackage{nicefrac}

% Image-related Packages —————————————————————————————
\usepackage{graphics, graphicx}
\usepackage{tikz}
\usetikzlibrary{arrows.meta}
\usepackage{pgfplots}
\pgfplotsset{compat=1.18}
\usepackage{xcolor}
\usepackage{fourier-orns}
\usepackage{lipsum}


%% Silakan otak-atik judul di sini
\title{\texorpdfstring{\vspace{-1.5em}}{}\textbf{概率论与数理统计}}
\author{张梓卫 10235101526}
\date{\the\day \ \monthyear}

% Kalau mau bab pertama nomornya 0, ganti 0 jadi -1
\setcounter{section}{0}

\renewcommand*\contentsname{第十三周概率论作业}
\renewcommand*{\multicolumntoc}{2}
\setlength{\columnseprule}{0.5pt}

% Layouting Packages ————————————————————————————————————————
\usepackage{titlesec}
\usepackage{fancyhdr}
\pagestyle{fancy}
\setlength{\headheight}{14.39996pt}
\fancyfoot[C]{\textit{By Deralive (10235101526)}}
\fancyfoot[R]{\thepage}

\fancyhead[LE, RO]{\textsl{\rightmark}}
\fancyhead[LO, RE]{\textsl{\leftmark}}

\renewcommand\headrule{
	\vspace{-6pt}
	\hrulefill
	\raisebox{-2.1pt}
	{\quad\floweroneleft\decoone\floweroneright\quad}
	\hrulefill}

\renewcommand\footrule{
	\hrulefill
	\raisebox{-2.1pt}
	{\quad\floweroneleft\decoone\floweroneright\quad}
	\hrulefill}
  
\newcommand{\fpb}[2]{\mathrm{FPB}(#1, #2)}
\newcommand{\kpk}[2]{\mathrm{KPK}(#1, #2)}

% Reference and Bibliography Packages ————————————————————
\usepackage{hyperref}
\hypersetup{
    colorlinks=true,
    linkcolor={black},
    citecolor={biru!70!black},
    urlcolor={biru!80!black}
}

\numberwithin{equation}{section}

% Silakan lihat dokumentasi package biblatex
% untuk format sitasi yang diperlukan
\usepackage[backend=biber]{biblatex}
\addbibresource{ref.bib}
\makeatletter
\def\@biblabel#1{}
\makeatother

\renewcommand{\mod}{\mathrm{mod} \ }

% Titling
\newcommand{\garis} [3] []{
	\begin{center}
		\begin{tikzpicture}
			\draw[#2-#3, ultra thick, #1] (0,0) to (1\linewidth,0);
		\end{tikzpicture}
	\end{center}
}

\newcommand{\monthyear}{
  \ifcase\month\or January\or February\or March\or April\or May\or June\or
  July\or August\or September\or October\or November\or
  December\fi\space\number\year
}

% Colour Palette ——————————————————————————————————————
\definecolor{merah}{HTML}{F4564E}
\definecolor{merahtua}{HTML}{89313E}
\definecolor{biru}{HTML}{60BBE5}
\definecolor{birutua}{HTML}{412F66}
\definecolor{hijau}{HTML}{59CC78}
\definecolor{hijautua}{HTML}{366D5B}
\definecolor{kuning}{HTML}{FFD56B}
\definecolor{jingga}{HTML}{FBA15F}
\definecolor{ungu}{HTML}{8C5FBF}
\definecolor{lavender}{HTML}{CBA5E8}
\definecolor{merjamb}{HTML}{FFB6E0}
\definecolor{mygray}{HTML}{E6E6E6}

% Theorems ————————————————————————————————————————————
\usepackage{tcolorbox}
\tcbuselibrary{skins,breakable,theorems}
\usepackage{changepage}

\newcounter{hitung}
\setcounter{hitung}{\thesection}

\makeatletter
	% Proof 证明如下
	\def\tcb@theo@widetitle#1#2#3{\hbox to \textwidth{\textsc{\large#1}\normalsize\space#3\hfil(#2)}}
	\tcbset{
		theorem style/theorem wide name and number/.code={ \let\tcb@theo@title=\tcb@theo@widetitle},
		proofbox/.style={skin=enhancedmiddle,breakable,parbox=false,boxrule=0mm,
			check odd page, toggle left and right, colframe=black!20!white!92!hijau,
			leftrule=8pt, rightrule=0mm, boxsep=0mm,arc=0mm, outer arc=0mm,
			left=3mm,right=3mm,top=0mm,bottom=0mm, toptitle=0mm,
			bottomtitle=0mm,colback=gray!3!white!98!biru, before skip=8pt, after skip=8pt,
			before={\par\vskip-2pt},after={\par\smallbreak},
		},
	}
	\newtcolorbox{ProofBox}{proofbox}
	\makeatother
	
	\let\realproof\proof
	\let\realendproof\endproof
	\renewenvironment{proof}[1][Prove:]{\ProofBox\strut\textsc{#1}\space}{\endProofBox}
        \AtEndEnvironment{proof}{\null\hfill$\blacksquare$}
        % Definition 定义环境
	\newtcbtheorem[use counter=hitung, number within=section]{dfn}{定义}
	{theorem style=theorem wide name and number,breakable,enhanced,arc=3.5mm,outer arc=3.5mm,
		boxrule=0pt,toprule=1pt,leftrule=0pt,bottomrule=1pt, rightrule=0pt,left=0.2cm,right=0.2cm,
		titlerule=0.5em,toptitle=0.1cm,bottomtitle=-0.1cm,top=0.2cm,
		colframe=white!10!biru,
		colback=white!90!biru,
		coltitle=white,
		shadow={1.3mm}{-1.3mm}{0mm}{gray!50!white}, % 添加阴影
        coltext=birutua!60!gray, title style={white!10!biru}, rbefoe skip=8pt, after skip=8pt,
		fonttitle=\bfseries,fontupper=\normalsize}{dfn}

	% 答题卡
	\newtcbtheorem[use counter=hitung, number within=section]{ans}{解答}
	{theorem style=theorem wide name and number,breakable,enhanced,arc=3.5mm,outer arc=3.5mm,
		boxrule=0pt,toprule=1pt,leftrule=0pt,bottomrule=1pt, rightrule=0pt,left=0.2cm,right=0.2cm,
		titlerule=0.5em,toptitle=0.1cm,bottomtitle=-0.1cm,top=0.2cm,
		colframe=white!10!biru,
		colback=white!90!biru,
		coltitle=white,
		shadow={1.3mm}{-1.3mm}{0mm}{gray!50!white}, % 添加阴影
        coltext=birutua!60!gray, title style={white!10!biru}, before skip=8pt, after skip=8pt,
		fonttitle=\bfseries,fontupper=\normalsize}{ans}

	% Axiom
	\newtcbtheorem[use counter=hitung, number within=section]{axm}{公理}
	{theorem style=theorem wide name and number,breakable,enhanced,arc=3.5mm,outer arc=3.5mm,
		boxrule=0pt,toprule=1pt,leftrule=0pt,bottomrule=1pt, rightrule=0pt,left=0.2cm,right=0.2cm,
		titlerule=0.5em,toptitle=0.1cm,bottomtitle=-0.1cm,top=0.2cm,
		colframe=white!10!biru,colback=white!90!biru,coltitle=white,
		shadow={1.3mm}{-1.3mm}{0mm}{gray!50!white!90}, % 添加阴影
        coltext=birutua!60!gray,title style={white!10!biru},before skip=8pt, after skip=8pt,
		fonttitle=\bfseries,fontupper=\normalsize}{axm}
 
	% Theorem
	\newtcbtheorem[use counter=hitung, number within=section]{thm}{定理}
	{theorem style=theorem wide name and number,breakable,enhanced,arc=3.5mm,outer arc=3.5mm,
		boxrule=0pt,toprule=1pt,leftrule=0pt,bottomrule=1pt, rightrule=0pt,left=0.2cm,right=0.2cm,
		titlerule=0.5em,toptitle=0.1cm,bottomtitle=-0.1cm,top=0.2cm,
		colframe=white!10!merah,colback=white!75!pink,coltitle=white, coltext=merahtua!80!merah,
		shadow={1.3mm}{-1.3mm}{0mm}{gray!50!white!90}, % 添加阴影
		title style={white!10!merah}, before skip=8pt, after skip=8pt,
		fonttitle=\bfseries,fontupper=\normalsize}{thm}
	
	% Proposition
	\newtcbtheorem[use counter=hitung, number within=section]{prp}{命题}
	{theorem style=theorem wide name and number,breakable,enhanced,arc=3.5mm,outer arc=3.5mm,
		boxrule=0pt,toprule=1pt,leftrule=0pt,bottomrule=1pt, rightrule=0pt,left=0.2cm,right=0.2cm,
		titlerule=0.5em,toptitle=0.1cm,bottomtitle=-0.1cm,top=0.2cm,
		colframe=white!10!hijau,colback=white!90!hijau,coltitle=white, coltext=hijautua!80!brown,
		shadow={1.3mm}{-1.3mm}{0mm}{gray!50!white}, % 添加阴影
		title style={white!10!hijau}, before skip=8pt, after skip=8pt,
		fonttitle=\bfseries,fontupper=\normalsize}{prp}


	% Example
	\newtcolorbox[use counter=hitung, number within=section]{cth}[1][]{breakable,
		colframe=white!10!jingga, coltitle=white!90!jingga, colback=white!85!jingga, coltext=black!10!brown!50!jingga, colbacktitle=white!10!jingga, enhanced, fonttitle=\bfseries,fontupper=\normalsize, attach boxed title to top left={yshift=-2mm}, before skip=8pt, after skip=8pt,
		title=Contoh~\thetcbcounter \ \ #1}

	% Catatan/Note
	\newtcolorbox{ctt}[1][]{enhanced, 
		left=4.1mm, borderline west={8pt}{0pt}{white!10!kuning}, 
		before skip=6pt, after skip=6pt, 
		colback=white!85!kuning, colframe= white!85!kuning, coltitle=orange!60!kuning!25!brown, coltext=orange!60!kuning!25!brown,
		fonttitle=\bfseries,fontupper=\normalsize, before skip=8pt, after skip=8pt,
		title=\underline{Catatan}  #1}
	
	% Komentar/Remark
	\newtcolorbox{rmr}[1][]{
		,arc=0mm,outer arc=0mm,
		boxrule=0pt,toprule=1pt,leftrule=0pt,bottomrule=5pt, rightrule=0pt,left=0.2cm,right=0.2cm,
		titlerule=0.5em,toptitle=0.1cm,bottomtitle=-0.1cm,top=0.2cm,
		colframe=white!10!kuning,colback=white!85!kuning,coltitle=white, coltext=orange!60!kuning,
		fonttitle=\bfseries,fontupper=\normalsize, before skip=8pt, after skip=8pt,
		title=Komentar  #1}

% ————————————————————————————————————————————————————————————————————————
\begin{document}

\maketitle
\vspace{-3.5em}
\garis{Kite}{Kite}

\tableofcontents

\section{第四章习题 21}

\begin{ans}{21}{21}
    已知随机变量 \( X \sim U(0,2) \),表示矩形的长,矩形的周长为 20,因此矩形的宽为 \( 10 - X \)。矩形的面积 \( A \) 可表示为:
    \[
    A = X(10 - X).
    \]
    
    \subsection*{1. 数学期望 \( \mathbb{E}[A] \)}
    
    \[
    \mathbb{E}[A] = \mathbb{E}[X(10 - X)] = \int_{0}^{2} x(10 - x) f_X(x) \, dx,
    \]
    其中 \( f_X(x) = \frac{1}{2} \) 为均匀分布的概率密度函数。展开积分:
    \[
    \mathbb{E}[A] = \int_{0}^{2} \frac{1}{2} (10x - x^2) 
    \]
    
    \[
     = \frac{1}{2} \times \left[ 5x^2 - \frac{x^3}{3} \right]_{0}^{2} = 20 = \frac{8}{3}
    \]
    
    \subsection*{2. 二阶矩 \( \mathbb{E}[A^2] \)}
    
    面积的二阶矩为:
    \[
    \mathbb{E}[A^2] = \int_{0}^{2} x^2 (10 - x)^2 f_X(x) \, dx.
    \]
    
    展开 \( (10 - x)^2 \) 并计算:
    \[
    \mathbb{E}[A^2] = \frac{1}{2} \int_{0}^{2} (100x^2 - 20x^3 + x^4) \, dx.
    \]
    
    \[
    \mathbb{E}[A^2] = \frac{1448}{15} \approx 96.53.
    \]
    
    \subsection*{3. 方差 \( D(A) \)}
    \[
    D(A) = \mathbb{E}[A^2] - (\mathbb{E}[A])^2.
    \]
    
    \[
    D(A) = \frac{1448}{15} - \left(\frac{26}{3}\right)^2 = \frac{1448}{15} - \frac{676}{9} \approx 21.42.
    \]
\end{ans}

\section{第四章习题 24}

\begin{ans}{24}{24}
    已知每袋水泥质量 \( X \sim N(50, 2.5^2) \),问卡车最多能装多少袋水泥,使得总质量超过 2000 kg 的概率不大于 0.05。

    \subsection*{解题步骤}
    
    设卡车所装水泥的总质量为:
    \[
    W = X_1 + X_2 + \cdots + X_n,
    \]
    其中 \( X_i \sim N(50, 2.5^2) \) 且相互独立,故:
    \[
    W \sim N(50n, 2.5^2 n).
    \]
    
    按题意要求:
    \[
    P(W \geq 2000) \leq 0.05.
    \]
    
    标准化 \( W \):
    \[
    P(W \geq 2000) = 1 - \Phi \left( \frac{2000 - 50n}{2.5\sqrt{n}} \right),
    \]
    其中 \(\Phi(z)\) 表示标准正态分布的累积分布函数。即:

    \[
    \Phi \left( \frac{2000 - 50n}{2.5\sqrt{n}} \right) \geq 0.95.
    \]
    
    \[
    \frac{2000 - 50n}{2.5\sqrt{n}} \geq 1.645.
    \]

    \[
    \sqrt{n} \leq 6.2836.
    \]
    
    因此,\( n \) 的最大整数为 39.
    
\end{ans}

\section{习题 25}
设随机变量 \( X \) 和 \( Y \) 相互独立,且均服从 \( U(0,1) \)。现计算以下量:

\subsection*{(1) 求 \( \mathbb{E}(XY), \mathbb{E}(\frac{X}{Y}), \mathbb{E}(\ln(XY)), \mathbb{E}(|Y - X|) \)}

\begin{ans}{25(1)}{25(1)}

\[
\mathbb{E}(XY) = \mathbb{E}(X) \cdot \mathbb{E}(Y) = \frac{1}{2} \cdot \frac{1}{2} = \frac{1}{4}.
\]


由于 \( \frac{X}{Y} \) 发散,故 \( \mathbb{E}(\frac{X}{Y}) \) 不存在。

\[
\mathbb{E}[\ln(XY)] = \mathbb{E}[\ln(X)] + \mathbb{E}[\ln(Y)] = -1 + (-1) = -2.
\]

\[
\mathbb{E}(|Y - X|) = 2 \int_{0}^{1} \int_{x}^{1} (y - x) \, dy \, dx = \frac{2}{3}.
\]

\end{ans}

\subsection*{(2) 令 \( A = XY \), \( C = 2(X + Y) \),求 \( \text{Cov}(A, C) \) 和 \( \rho_{AC} \)}

\begin{ans}{25(2)}{25(2)}
\[
\text{Cov}(A, C) = \mathbb{E}(AC) - \mathbb{E}(A)\mathbb{E}(C) = \frac{1}{6}.
\]

\[
D(A) = \frac{7}{144}, \quad D(C) = \frac{2}{3}.
\]

\[
\rho_{AC} = \frac{\text{Cov}(A, C)}{\sqrt{D(A) D(C)}} = \sqrt{\frac{6}{7}}.
\]

\end{ans}

\section{习题 26}

\begin{ans}{26}{26}

    已知:
    \[
    X_1 \sim b\left(4, \frac{1}{2}\right), \quad X_2 \sim b\left(6, \frac{1}{3}\right), \quad X_3 \sim b\left(6, \frac{1}{3}\right).
    \]
    
    1. 
    \[
    P(X_1 = 2, X_2 = 2, X_3 = 5) = P(X_1 = 2) \cdot P(X_2 = 2) \cdot P(X_3 = 5).
    \]
    
    由二项分布:
    \[
    P(X_1 = 2) = \binom{4}{2} \left(\frac{1}{2}\right)^2 \left(\frac{1}{2}\right)^2 = \frac{3}{8},
    \]
    \[
    P(X_2 = 2) = \binom{6}{2} \left(\frac{1}{3}\right)^2 \left(\frac{2}{3}\right)^4 = \frac{240}{729},
    \]
    \[
    P(X_3 = 5) = \binom{6}{5} \left(\frac{1}{3}\right)^5 \left(\frac{2}{3}\right) = \frac{12}{729}.
    \]
    
    故有:
    \[
    P(X_1 = 2, X_2 = 2, X_3 = 5) = \frac{3}{8} \cdot \frac{240}{729} \cdot \frac{12}{729} = 0.00203.
    \]
    
    2. 计算 \( \mathbb{E}(X_1 X_2 X_3) \):
    \[
    \mathbb{E}(X_1 X_2 X_3) = \mathbb{E}(X_1) \cdot \mathbb{E}(X_2) \cdot \mathbb{E}(X_3).
    \]
    
    由二项分布:
    \[
    \mathbb{E}(X_1) = 2, \quad \mathbb{E}(X_2) = 2, \quad \mathbb{E}(X_3) = 2.
    \]
    
    故有:
    \[
    \mathbb{E}(X_1 X_2 X_3) = 2 \cdot 2 \cdot 2 = 8.
    \]
    
    3. 计算 \( \mathbb{E}(X_1 - X_2) \):
    \[
    \mathbb{E}(X_1 - X_2) = \mathbb{E}(X_1) - \mathbb{E}(X_2) = 2 - 2 = 0.
    \]
    
    4. 计算 \( \mathbb{E}(X_1 - 2X_2) \):
    \[
    \mathbb{E}(X_1 - 2X_2) = \mathbb{E}(X_1) - 2 \cdot \mathbb{E}(X_2) = 2 - 4 = -2.
    \]
    
    \subsection*{(2) 随机变量 \( X, Y \) 的计算}
    
    已知:
    \[
    \mathbb{E}(X) = 3, \quad \mathbb{E}(Y) = 1, \quad D(X) = 4, \quad D(Y) = 9, \quad Z = 5X - Y + 15.
    \]
    
    \[
    \mathbb{E}(Z) = 5 \mathbb{E}(X) - \mathbb{E}(Y) + 15 = 5 \cdot 3 - 1 + 15 = 29.
    \]
    
    2. 情况 (i):\( X, Y \) 独立
    \[
    D(Z) = D(5X - Y) = D(5X) + D(-Y) = 25D(X) + D(Y) = 25 \cdot 4 + 9 = 109.
    \]
    
    3. 情况 (ii):\( X, Y \) 不相关
    \[
    D(Z) = 109 \quad (\text{与独立相同}).
    \]
    
    4. 情况 (iii):\( \rho_{XY} = 0.25 \)
    \[
    \text{Cov}(X, Y) = \rho_{XY} \sqrt{D(X) D(Y)} = 0.25 \cdot \sqrt{4 \cdot 9} = 1.5.
    \]
    \[
    D(Z) = 25D(X) + D(Y) - 10 \text{Cov}(X, Y) = 100 + 9 - 15 = 94.
    \]
\end{ans}

\section{习题 27}

\begin{ans}{27}{27}
\subsection*{(1) \( X \sim U(0, 1), \, Y = X^2 \)}

\[
\mathbb{E}(X) = \frac{1}{2}, \quad \mathbb{E}(Y) = \frac{1}{3}.
\]

\[
\text{Cov}(X, Y) = \mathbb{E}(XY) - \mathbb{E}(X)\mathbb{E}(Y) = \frac{1}{3} - \frac{1}{4} \cdot \frac{1}{2} = \frac{1}{4}.
\]

结论:\( X, Y \) 不相互独立,也不相互不相关。

\subsection*{(2) \( X \sim U(-1, 1), \, Y = X^2 \)}

\[
\mathbb{E}(X) = 0, \quad \mathbb{E}(Y) = \frac{1}{3}.
\]

\[
\text{Cov}(X, Y) = 0.
\]

结论:\( X, Y \) 不相互独立,但相互不相关。

\subsection*{(3) \( X = \cos V, \, Y = \sin V, \, V \sim U(0, 2\pi) \)}

1. 计算 \( \mathbb{E}(X) \) 和 \( \mathbb{E}(Y) \):
\[
\mathbb{E}(X) = \int_{0}^{2\pi} \frac{1}{2\pi} \cos v \, dv = 0, \quad \mathbb{E}(Y) = \int_{0}^{2\pi} \frac{1}{2\pi} \sin v \, dv = 0.
\]

2. 计算 \( \mathbb{E}(XY) \):
\[
\mathbb{E}(XY) = \mathbb{E}(\sin V \cos V) = \frac{1}{2} \mathbb{E}(\sin(2V)).
\]
由于 \( \sin(2V) \) 在 \( [0, 2\pi] \) 上对称,故:
\[
\mathbb{E}(\sin(2V)) = \int_{0}^{2\pi} \frac{1}{2\pi} \sin(2V) \, dv = 0.
\]

因此:
\[
\mathbb{E}(XY) = 0.
\]

3. 计算协方差:
\[
\text{Cov}(X, Y) = \mathbb{E}(XY) - \mathbb{E}(X) \mathbb{E}(Y) = 0 - 0 \cdot 0 = 0.
\]

结论:\( X \) 和 \( Y \) 不相互独立,但相互不相关。

\subsection*{(4) \( f(x, y) = x + y \), \( 0 < x < 1, 0 < y < 1 \)}

1. 计算边缘分布:
\[
f_X(x) = \int_{0}^{1} (x + y) \, dy = x + \frac{1}{2}, \quad f_Y(y) = \int_{0}^{1} (x + y) \, dx = y + \frac{1}{2}.
\]

2. 检查独立性:
联合分布 \( f(x, y) \) 与 \( f_X(x) f_Y(y) \) 不相等,因此 \( X \) 和 \( Y \) 不独立。

3. 计算 \( \mathbb{E}(X) \) 和 \( \mathbb{E}(Y) \):
\[
\mathbb{E}(X) = \int_{0}^{1} x \left(x + \frac{1}{2}\right) \, dx = \frac{7}{12},
\]
\[
\mathbb{E}(Y) = \int_{0}^{1} y \left(y + \frac{1}{2}\right) \, dy = \frac{7}{12}.
\]

4. 计算 \( \mathbb{E}(XY) \):
\[
\mathbb{E}(XY) = \int_{0}^{1} \int_{0}^{1} xy (x + y) \, dx \, dy = \frac{1}{3}.
\]

5. 计算协方差:
\[
\text{Cov}(X, Y) = \mathbb{E}(XY) - \mathbb{E}(X) \mathbb{E}(Y) = \frac{1}{3} - \frac{7}{12} \cdot \frac{7}{12}.
\]

结论:\( X \) 和 \( Y \) 不独立,且不不相关。

\subsection*{(5) \( f(x, y) = 2y \), \( 0 < x < 1, 0 < y < 1 \)}

1. 计算边缘分布:
\[
f_X(x) = \int_{0}^{1} 2y \, dy = 1, \quad f_Y(y) = \int_{0}^{1} 2y \, dx = 2y.
\]

2. 检查独立性:
联合分布可分解为:
\[
f(x, y) = f_X(x) f_Y(y).
\]

结论:\( X \) 和 \( Y \) 相互独立。

\end{ans}

\section{习题 34}

\begin{ans}{34}{34}
    \subsection*{(1) 求常数 \( a \) 使 \( \mathbb{E}(W) \) 最小,并求最小值}

    \[
    W = (aX + 3Y)^2, \quad \mathbb{E}(X) = \mathbb{E}(Y) = 0, \quad D(X) = 4, \quad D(Y) = 16, \quad \rho_{XY} = -0.5.
    \]
    
    \[
    \mathbb{E}(W) = \text{Var}(aX + 3Y).
    \]
    
    由方差公式:
    \[
    \text{Var}(aX + 3Y) = a^2 \text{Var}(X) + 9 \text{Var}(Y) + 2a \cdot 3 \cdot \text{Cov}(X, Y).
    \]
    
    \[
    \text{Var}(X) = 4, \quad \text{Var}(Y) = 16, \quad \text{Cov}(X, Y) = \rho_{XY} \sqrt{D(X)D(Y)} = -4.
    \]
    
    \[
    \text{Var}(aX + 3Y) = 4a^2 + 144 - 24a.
    \]
    
    令 \( \mathbb{E}(W) = 4a^2 - 24a + 144 \),求极小值,对其求导:
    \[
    f'(a) = 8a - 24.
    \]
    
    令 \( f'(a) = 0 \),得:$a = 3$.
    
    \[
    \min \mathbb{E}(W) = 4(3)^2 - 24(3) + 144 = 108.
    \]
    
    \subsection*{(2) 证明 \( W = X - aY \) 与 \( V = X + aY \) 相互独立}
    
    \[
    W = X - aY, \quad V = X + aY.
    \]
    
    由协方差的线性性质:
    \[
    \text{Cov}(W, V) = \text{Cov}(X - aY, X + aY) = \text{Var}(X) - a^2 \text{Var}(Y).
    \]
    
    当 \( \text{Cov}(W, V) = 0 \) 时:
    \[
    \sigma_X^2 - a^2 \sigma_Y^2 = 0.
    \]
    
    \[
    a^2 = \frac{\sigma_X^2}{\sigma_Y^2}.
    \]
    
    因此,当 \( a^2 = \sigma_X^2 / \sigma_Y^2 \) 时,\( W \) 和 \( V \) 相互独立。
\end{ans}

\section{习题 36}

\begin{ans}{36}{36}
    已知每毫升正常男性成人血液中白细胞数 \( X \) 的分布满足:
    \[
    \mathbb{E}(X) = \mu = 7300, \quad \sqrt{\text{Var}(X)} = \sigma = 700.
    \]
    
    目标:利用切比雪夫不等式估计白细胞数在 \( 5200 \sim 9400 \) 之间的概率 \( p \)。
    
    
    将区间改写为关于均值 \( \mu \) 和标准差 \( \sigma \) 的形式:
    \[
    p = P(-2100 < X - 7300 < 2100) = P(|X - \mu| < 2100).
    \]
    
    2. 估计 
    \[
    P(|X - \mu| < \epsilon) \geq 1 - \frac{\sigma^2}{\epsilon^2}.
    \]
    
    在本题中,取 \( \epsilon = 2100 \)
    \[
    1 - \frac{\sigma^2}{\epsilon^2} = 1 - \frac{700^2}{2100^2}.
    \]
    
    \[
    \frac{\sigma^2}{\epsilon^2} = \frac{700^2}{2100^2} = \frac{490000}{4410000} = \frac{1}{9}.
    \]
    
    \[
    P(|X - \mu| < 2100) \geq 1 - \frac{1}{9} = \frac{8}{9}.
    \]
    
\end{ans}


\end{document}