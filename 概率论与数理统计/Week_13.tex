\documentclass[twoside]{article}
\usepackage[paperwidth=210mm, paperheight=297mm, margin=2cm]{geometry}
\usepackage[utf8]{inputenc}
\usepackage{ctex}

% Formatting Packages ——————————————————————————————————————
\usepackage{multicol}
\usepackage{multirow}
\usepackage{enumitem}
\usepackage{indentfirst}
\usepackage[toc]{multitoc}

% Math & Physics Packages ————————————————————————————
\usepackage{amsmath, amsthm, amsfonts, amssymb}
\usepackage{amssymb}
\usepackage{setspace}
\usepackage{physics}
\usepackage{cancel}
\usepackage{nicefrac}

% Image-related Packages —————————————————————————————
\usepackage{graphics, graphicx}
\usepackage{tikz}
\usetikzlibrary{arrows.meta}
\usepackage{pgfplots}
\pgfplotsset{compat=1.18}
\usepackage{xcolor}
\usepackage{fourier-orns}
\usepackage{lipsum}


%% Silakan otak-atik judul di sini
\title{\texorpdfstring{\vspace{-1.5em}}{}\textbf{概率论与数理统计}}
\author{张梓卫 10235101526}
\date{\the\day \ \monthyear}

% Kalau mau bab pertama nomornya 0, ganti 0 jadi -1
\setcounter{section}{0}

\renewcommand*\contentsname{第十三周概率论作业}
\renewcommand*{\multicolumntoc}{2}
\setlength{\columnseprule}{0.5pt}

% Layouting Packages ————————————————————————————————————————
\usepackage{titlesec}
\usepackage{fancyhdr}
\pagestyle{fancy}
\setlength{\headheight}{14.39996pt}
\fancyfoot[C]{\textit{By Deralive (10235101526)}}
\fancyfoot[R]{\thepage}

\fancyhead[LE, RO]{\textsl{\rightmark}}
\fancyhead[LO, RE]{\textsl{\leftmark}}

\renewcommand\headrule{
	\vspace{-6pt}
	\hrulefill
	\raisebox{-2.1pt}
	{\quad\floweroneleft\decoone\floweroneright\quad}
	\hrulefill}

\renewcommand\footrule{
	\hrulefill
	\raisebox{-2.1pt}
	{\quad\floweroneleft\decoone\floweroneright\quad}
	\hrulefill}
  
\newcommand{\fpb}[2]{\mathrm{FPB}(#1, #2)}
\newcommand{\kpk}[2]{\mathrm{KPK}(#1, #2)}

% Reference and Bibliography Packages ————————————————————
\usepackage{hyperref}
\hypersetup{
    colorlinks=true,
    linkcolor={black},
    citecolor={biru!70!black},
    urlcolor={biru!80!black}
}

\numberwithin{equation}{section}

% Silakan lihat dokumentasi package biblatex
% untuk format sitasi yang diperlukan
\usepackage[backend=biber]{biblatex}
\addbibresource{ref.bib}
\makeatletter
\def\@biblabel#1{}
\makeatother

\renewcommand{\mod}{\mathrm{mod} \ }

% Titling
\newcommand{\garis} [3] []{
	\begin{center}
		\begin{tikzpicture}
			\draw[#2-#3, ultra thick, #1] (0,0) to (1\linewidth,0);
		\end{tikzpicture}
	\end{center}
}

\newcommand{\monthyear}{
  \ifcase\month\or January\or February\or March\or April\or May\or June\or
  July\or August\or September\or October\or November\or
  December\fi\space\number\year
}

% Colour Palette ——————————————————————————————————————
\definecolor{merah}{HTML}{F4564E}
\definecolor{merahtua}{HTML}{89313E}
\definecolor{biru}{HTML}{60BBE5}
\definecolor{birutua}{HTML}{412F66}
\definecolor{hijau}{HTML}{59CC78}
\definecolor{hijautua}{HTML}{366D5B}
\definecolor{kuning}{HTML}{FFD56B}
\definecolor{jingga}{HTML}{FBA15F}
\definecolor{ungu}{HTML}{8C5FBF}
\definecolor{lavender}{HTML}{CBA5E8}
\definecolor{merjamb}{HTML}{FFB6E0}
\definecolor{mygray}{HTML}{E6E6E6}

% Theorems ————————————————————————————————————————————
\usepackage{tcolorbox}
\tcbuselibrary{skins,breakable,theorems}
\usepackage{changepage}

\newcounter{hitung}
\setcounter{hitung}{\thesection}

\makeatletter
	% Proof 证明如下
	\def\tcb@theo@widetitle#1#2#3{\hbox to \textwidth{\textsc{\large#1}\normalsize\space#3\hfil(#2)}}
	\tcbset{
		theorem style/theorem wide name and number/.code={ \let\tcb@theo@title=\tcb@theo@widetitle},
		proofbox/.style={skin=enhancedmiddle,breakable,parbox=false,boxrule=0mm,
			check odd page, toggle left and right, colframe=black!20!white!92!hijau,
			leftrule=8pt, rightrule=0mm, boxsep=0mm,arc=0mm, outer arc=0mm,
			left=3mm,right=3mm,top=0mm,bottom=0mm, toptitle=0mm,
			bottomtitle=0mm,colback=gray!3!white!98!biru, before skip=8pt, after skip=8pt,
			before={\par\vskip-2pt},after={\par\smallbreak},
		},
	}
	\newtcolorbox{ProofBox}{proofbox}
	\makeatother
	
	\let\realproof\proof
	\let\realendproof\endproof
	\renewenvironment{proof}[1][Prove:]{\ProofBox\strut\textsc{#1}\space}{\endProofBox}
        \AtEndEnvironment{proof}{\null\hfill$\blacksquare$}
        % Definition 定义环境
	\newtcbtheorem[use counter=hitung, number within=section]{dfn}{定义}
	{theorem style=theorem wide name and number,breakable,enhanced,arc=3.5mm,outer arc=3.5mm,
		boxrule=0pt,toprule=1pt,leftrule=0pt,bottomrule=1pt, rightrule=0pt,left=0.2cm,right=0.2cm,
		titlerule=0.5em,toptitle=0.1cm,bottomtitle=-0.1cm,top=0.2cm,
		colframe=white!10!biru,
		colback=white!90!biru,
		coltitle=white,
		shadow={1.3mm}{-1.3mm}{0mm}{gray!50!white}, % 添加阴影
        coltext=birutua!60!gray, title style={white!10!biru}, rbefoe skip=8pt, after skip=8pt,
		fonttitle=\bfseries,fontupper=\normalsize}{dfn}

	% 答题卡
	\newtcbtheorem[use counter=hitung, number within=section]{ans}{解答}
	{theorem style=theorem wide name and number,breakable,enhanced,arc=3.5mm,outer arc=3.5mm,
		boxrule=0pt,toprule=1pt,leftrule=0pt,bottomrule=1pt, rightrule=0pt,left=0.2cm,right=0.2cm,
		titlerule=0.5em,toptitle=0.1cm,bottomtitle=-0.1cm,top=0.2cm,
		colframe=white!10!biru,
		colback=white!90!biru,
		coltitle=white,
		shadow={1.3mm}{-1.3mm}{0mm}{gray!50!white}, % 添加阴影
        coltext=birutua!60!gray, title style={white!10!biru}, before skip=8pt, after skip=8pt,
		fonttitle=\bfseries,fontupper=\normalsize}{ans}

	% Axiom
	\newtcbtheorem[use counter=hitung, number within=section]{axm}{公理}
	{theorem style=theorem wide name and number,breakable,enhanced,arc=3.5mm,outer arc=3.5mm,
		boxrule=0pt,toprule=1pt,leftrule=0pt,bottomrule=1pt, rightrule=0pt,left=0.2cm,right=0.2cm,
		titlerule=0.5em,toptitle=0.1cm,bottomtitle=-0.1cm,top=0.2cm,
		colframe=white!10!biru,colback=white!90!biru,coltitle=white,
		shadow={1.3mm}{-1.3mm}{0mm}{gray!50!white!90}, % 添加阴影
        coltext=birutua!60!gray,title style={white!10!biru},before skip=8pt, after skip=8pt,
		fonttitle=\bfseries,fontupper=\normalsize}{axm}
 
	% Theorem
	\newtcbtheorem[use counter=hitung, number within=section]{thm}{定理}
	{theorem style=theorem wide name and number,breakable,enhanced,arc=3.5mm,outer arc=3.5mm,
		boxrule=0pt,toprule=1pt,leftrule=0pt,bottomrule=1pt, rightrule=0pt,left=0.2cm,right=0.2cm,
		titlerule=0.5em,toptitle=0.1cm,bottomtitle=-0.1cm,top=0.2cm,
		colframe=white!10!merah,colback=white!75!pink,coltitle=white, coltext=merahtua!80!merah,
		shadow={1.3mm}{-1.3mm}{0mm}{gray!50!white!90}, % 添加阴影
		title style={white!10!merah}, before skip=8pt, after skip=8pt,
		fonttitle=\bfseries,fontupper=\normalsize}{thm}
	
	% Proposition
	\newtcbtheorem[use counter=hitung, number within=section]{prp}{命题}
	{theorem style=theorem wide name and number,breakable,enhanced,arc=3.5mm,outer arc=3.5mm,
		boxrule=0pt,toprule=1pt,leftrule=0pt,bottomrule=1pt, rightrule=0pt,left=0.2cm,right=0.2cm,
		titlerule=0.5em,toptitle=0.1cm,bottomtitle=-0.1cm,top=0.2cm,
		colframe=white!10!hijau,colback=white!90!hijau,coltitle=white, coltext=hijautua!80!brown,
		shadow={1.3mm}{-1.3mm}{0mm}{gray!50!white}, % 添加阴影
		title style={white!10!hijau}, before skip=8pt, after skip=8pt,
		fonttitle=\bfseries,fontupper=\normalsize}{prp}


	% Example
	\newtcolorbox[use counter=hitung, number within=section]{cth}[1][]{breakable,
		colframe=white!10!jingga, coltitle=white!90!jingga, colback=white!85!jingga, coltext=black!10!brown!50!jingga, colbacktitle=white!10!jingga, enhanced, fonttitle=\bfseries,fontupper=\normalsize, attach boxed title to top left={yshift=-2mm}, before skip=8pt, after skip=8pt,
		title=Contoh~\thetcbcounter \ \ #1}

	% Catatan/Note
	\newtcolorbox{ctt}[1][]{enhanced, 
		left=4.1mm, borderline west={8pt}{0pt}{white!10!kuning}, 
		before skip=6pt, after skip=6pt, 
		colback=white!85!kuning, colframe= white!85!kuning, coltitle=orange!60!kuning!25!brown, coltext=orange!60!kuning!25!brown,
		fonttitle=\bfseries,fontupper=\normalsize, before skip=8pt, after skip=8pt,
		title=\underline{Catatan}  #1}
	
	% Komentar/Remark
	\newtcolorbox{rmr}[1][]{
		,arc=0mm,outer arc=0mm,
		boxrule=0pt,toprule=1pt,leftrule=0pt,bottomrule=5pt, rightrule=0pt,left=0.2cm,right=0.2cm,
		titlerule=0.5em,toptitle=0.1cm,bottomtitle=-0.1cm,top=0.2cm,
		colframe=white!10!kuning,colback=white!85!kuning,coltitle=white, coltext=orange!60!kuning,
		fonttitle=\bfseries,fontupper=\normalsize, before skip=8pt, after skip=8pt,
		title=Komentar  #1}

% ————————————————————————————————————————————————————————————————————————
\begin{document}

\maketitle
\vspace{-3.5em}
\garis{Kite}{Kite}

\tableofcontents

\section{第六章习题 2}

\begin{ans}{2}{2}
    已知总体 \( X \sim N(12, 4) \),从中随机抽取容量为5的样本 \( X_1, X_2, X_3, X_4, X_5 \)。

    \subsection*{(1) 求样本均值与总体均值之差的绝对值大于1的概率}
    
    样本均值:
    \[
    \bar{X} = \frac{1}{5} \sum_{i=1}^{5} X_i, \bar{X} \sim N\left( \mu, \frac{\sigma^2}{n} \right) = N\left( 12, \frac{4}{5} \right),
    \]
    
    其标准差为:
    \[
    \sqrt{\frac{\sigma^2}{n}} = \sqrt{\frac{4}{5}} = \frac{2}{\sqrt{5}}.
    \]
    
    目标概率为:
    \[
    P(|\bar{X} - \mu| > 1).
    \]
    
    \[
    P(|\bar{X} - \mu| > 1) = 2P\left(Z > \frac{\sqrt{5}}{2}\right).
    \]
    
    查正态分布表得:
    \[
    P(Z > \frac{\sqrt{5}}{2}) = P(Z > 1.118) \approx 0.131.
    \]
    
    \[
    P(|\bar{X} - \mu| > 1) = 2 \cdot 0.131 = 0.262.
    \]
    
    \subsection*{(2) 求 \( P\{\max(X_1, X_2, X_3, X_4, X_5) > 15\} \) 和 \( P\{\min(X_1, X_2, X_3, X_4, X_5) < 10\} \)}
    
    1. 
    
    记 \( Y = \max(X_1, X_2, X_3, X_4, X_5) \),则:
    \[
    P(Y \leq 15) = [P(X \leq 15)]^5.
    \]
    
    计算单个样本:
    \[
    P(X \leq 15) = P\left(Z \leq \frac{15 - 12}{2}\right) = P(Z \leq 1.5) \approx 0.9332.
    \]
    
    \[
    P(Y \leq 15) = (0.9332)^5 \approx 0.705.
    \]
    
    \[
    P(Y > 15) = 1 - P(Y \leq 15) = 1 - 0.705 = 0.295.
    \]
    
    2.
    
    记 \( Z = \min(X_1, X_2, X_3, X_4, X_5) \),则:
    \[
    P(Z \geq 10) = [P(X \geq 10)]^5.
    \]
    
    计算单个样本:
    \[
    P(X \geq 10) = 1 - P(Z \leq \frac{10 - 12}{2}) = 1 - P(Z \leq -1) = 1 - 0.1587 = 0.8413.
    \]
    
    \[
    P(Z \geq 10) = (0.8413)^5 \approx 0.418.
    \]
    
    \[
    P(Z < 10) = 1 - P(Z \geq 10) = 1 - 0.418 = 0.582.
    \]

\end{ans}

\section{第六章习题 3}

\begin{ans}{3}{3}

已知总体 \( N(20, 3) \),抽取容量分别为 10 和 15 的两个独立样本,样本均值分别记为 \( \bar{X} \) 和 \( \bar{Y} \)。

\subsection*{1. 样本均值的分布}

\[
\bar{X} \sim N\left( 20, \frac{3}{10} \right), \quad \bar{Y} \sim N\left( 20, \frac{3}{15} \right).
\]

由于 \( \bar{X} \) 和 \( \bar{Y} \) 独立,样本均值差的分布为:
\[
\bar{X} - \bar{Y} \sim N\left( 0, \frac{1}{2} \right).
\]

所求概率为:
\[
p = 2P\left( \bar{X} - \bar{Y} > 0.3 \right).
\]

\[
P\left( \bar{X} - \bar{Y} > 0.3 \right) = P\left( Z > \frac{0.3}{\sqrt{\frac{1}{2}}} \right) = P(Z > 0.42).
\]

查正态分布表得:
\[
P(Z > 0.42) = 1 - P(Z \leq 0.42) \approx 1 - 0.6628 = 0.3372.
\]

\[
p = 2 \cdot 0.3372 = 0.6744.
\]

\end{ans}

\section{第六章习题 4(2)}


设样本 \( X_1, X_2, \dots, X_5 \) 来自总体 \( N(0, 1) \),随机变量 \( Y \) 定义为:
\[
Y = \frac{C(X_1 + X_2)}{\sqrt{X_3^2 + X_4^2 + X_5^2}}.
\]

要求确定常数 \( C \),使得 \( Y \) 服从 \( t \) 分布。

\begin{ans}{4(2)}{4(2)}

1. \( X_1, X_2 \) 是来自 \( N(0, 1) \) 的独立同分布随机变量,则:
\[
X_1 + X_2 \sim N(0, 2).
\]

因此,\( \frac{X_1 + X_2}{\sqrt{2}} \sim N(0, 1) \)。

2. \( X_3^2 + X_4^2 + X_5^2 \) 是 3 个来自 \( N(0, 1) \) 的独立同分布随机变量平方和,因此:
\[
X_3^2 + X_4^2 + X_5^2 \sim \chi_3^2.
\]

3. \( \sqrt{X_3^2 + X_4^2 + X_5^2} \sim \sqrt{\chi_3^2} \),即自由度为 3 的卡方分布的平方根。

\subsection*{2. 构造 \( t \) 分布}

根据 \( t \) 分布的定义:
\[
t = \frac{Z}{\sqrt{V / k}}, \quad Z \sim N(0, 1), \quad V \sim \chi_k^2,
\]
其中 \( k \) 为自由度。

比较题目中的 \( Y \) 和 \( t \) 分布的形式,将 \( Y \) 写为:
\[
Y = \frac{C (X_1 + X_2)}{\sqrt{X_3^2 + X_4^2 + X_5^2}}.
\]

当自由度为 3 时,可以得到:$C = \sqrt{\frac{3}{2}} $.
\end{ans}

\section{第六章习题 6}
\begin{ans}{6}{6}

    已知总体 \( X \sim b(1, p) \),即 \( X \) 服从参数为 \( p \) 的伯努利分布。随机样本为 \( X_1, X_2, \dots, X_n \)。

\subsection*{(1) 求 \( (X_1, X_2, \dots, X_n) \) 的分布律}

由于 \( X_1, X_2, \dots, X_n \) 是从总体 \( X \) 中独立抽取的样本,且 \( X_i \sim b(1, p) \, (i = 1, 2, \dots, n) \),因此每个样本的分布为:
\[
P(X_i = x_i) = p^{x_i}(1 - p)^{1 - x_i}, \quad x_i = 0 \text{ 或 } 1.
\]

由于样本是相互独立的,因此联合分布为:
\[
P(X_1 = x_1, X_2 = x_2, \dots, X_n = x_n) = \prod_{i=1}^n P(X_i = x_i).
\]

将每个样本的概率代入得:
\[
P(X_1 = x_1, X_2 = x_2, \dots, X_n = x_n) = \prod_{i=1}^n p^{x_i}(1 - p)^{1 - x_i}.
\]

将幂次化简后得:
\[
P(X_1 = x_1, X_2 = x_2, \dots, X_n = x_n) = p^{\sum_{i=1}^n x_i}(1 - p)^{n - \sum_{i=1}^n x_i}.
\]

\subsection*{(2) 求 \( \sum_{i=1}^n X_i \) 的分布律}

由于 \( X_1, X_2, \dots, X_n \) 是独立同分布的随机变量,且 \( X_i \sim b(1, p) \),因此其和:
\[
\sum_{i=1}^n X_i \sim b(n, p).
\]

即:
\[
P\left(\sum_{i=1}^n X_i = k\right) = \binom{n}{k} p^k (1 - p)^{n - k}, \quad k = 0, 1, 2, \dots, n.
\]

\subsection*{最终结果}

1. \( (X_1, X_2, \dots, X_n) \) 的联合分布为:
\[
P(X_1 = x_1, X_2 = x_2, \dots, X_n = x_n) = p^{\sum_{i=1}^n x_i}(1 - p)^{n - \sum_{i=1}^n x_i}.
\]

2. \( \sum_{i=1}^n X_i \sim b(n, p) \),其分布律为:
\[
P\left(\sum_{i=1}^n X_i = k\right) = \binom{n}{k} p^k (1 - p)^{n - k}, \quad k = 0, 1, 2, \dots, n.
\]

3. 样本均值和样本方差的期望与方差为:
\[
\mathbb{E}(\bar{X}) = p, \quad D(\bar{X}) = \frac{p(1 - p)}{n}, \quad \mathbb{E}(S^2) = p(1 - p).
\]

\end{ans}

\section{第六章习题 9}

已知总体 \( N(\mu, \sigma^2) \),样本容量 \( n = 16 \),样本方差记为 \( S^2 \),目标是:

\begin{enumerate}
    \item 求 \( P\left(\frac{S^2}{\sigma^2} \leq 2.041\right) \)。
    \item 求 \( D(S^2) \)。
\end{enumerate}

\begin{ans}{9}{9}

\subsection*{(1) 求 \( P\left(\frac{S^2}{\sigma^2} \leq 2.041\right) \)}

由样本方差的统计性质可知:
\[
\frac{(n-1)S^2}{\sigma^2} \sim \chi^2(n-1),
\]
其中自由度为 \( n-1 = 15 \)。

因此,有:
\[
P\left(\frac{S^2}{\sigma^2} \leq 2.041\right) = P\left(\frac{15S^2}{\sigma^2} \leq 15 \cdot 2.041\right) = P\left(\frac{15S^2}{\sigma^2} \leq 30.615\right).
\]

定义变量:
\[
Y = \frac{15S^2}{\sigma^2}, \quad Y \sim \chi^2(15).
\]

\[
P\left(\frac{S^2}{\sigma^2} \leq 2.041\right) = P\left(Y \leq 30.615\right).
\]

查 \( \chi^2 \) 分布表得:
\[
P(Y \leq 30.615) = 1 - P(Y > 30.615).
\]

\[
P(Y \leq 30.615) \approx 1 - 0.01 = 0.99.
\]

\[
P\left(\frac{S^2}{\sigma^2} \leq 2.041\right) \approx 0.99.
\]

\subsection*{(2) 求 \( D(S^2) \)}

\[
\frac{(n-1)S^2}{\sigma^2} \sim \chi^2(n-1).
\]

记:
\[
Y = \frac{(n-1)S^2}{\sigma^2}, \quad Y \sim \chi^2(15).
\]

根据 \( \chi^2 \) 分布的性质,有:
\[
D(Y) = 2 \cdot (n-1) = 2 \cdot 15 = 30.
\]

结合 \( Y = \frac{15S^2}{\sigma^2} \),得:
\[
D\left(\frac{15S^2}{\sigma^2}\right) = 30.
\]

由方差的缩放性质:
\[
D(S^2) = \frac{\sigma^4}{15^2} \cdot D\left(\frac{15S^2}{\sigma^2}\right).
\]

\[
D(S^2) = \frac{\sigma^4}{15^2} \cdot 30 = \frac{2\sigma^4}{15}.
\]

\end{ans}

\end{document}